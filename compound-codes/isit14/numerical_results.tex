\label{section:numerical_results}
Table \ref{table:succ_enc} shows the number of attempts required to encode $50$ source sequences in $\mathcal{C}(s^k)$, with $s^k=0^k$, over different block lengths. 
For example, at a block length of $9000$, $21$ sequences encoded in the first attempt and $9$ sequences did not encode in four attempts.
Without removing $4$-cycles, only $5$ sequences encoded at first and $35$ failed after 4 attempts.
%the number of attempts for the same parameters are $5/3/5/2/35$.
At a block length of $81000$, all $50$ sequences encoded in the first attempt.

Tables \ref{table:wyner_ziv} and \ref{table:gelfand_pinsker} provide the simulation results with SC compound codes with message-passing algorithms.
In the LDGM part of the code, $1\%$ of the check-nodes have degree 1.
This is to ensure that the decoding gets started for the channel coding problem.
Encoding for the WZ problem is relatively easy, since this is performed using the codebook $\mathcal{SC}'$, which does not have LDPC check constraints.
The optimal thresholds are calculated based on the rate of the code.
For example, if a code of rate $R$ is used to encode a $\mathsf{Ber}(\tfrac{1}{2})$, then $D_{*}=h^{-1}(1-R)$.
The reported distortion and the optimal thresholds correspond to the saturated section of the SC system uneffected by the boundary condition.

For the GP problem, half of all the LDPC check-nodes are chosen randomly to carry message bits, except in the $w-1$ sections at the boundary where no message bits are placed.
Since the bit-nodes in the left-most $w-1$ sections are initialized to $0$, the LDPC check nodes cannot carry any messages in the left-most $w-1$ sections.
Also, since the channel code for the GP problem uses fewer parity checks for the right boundary LDPC bit-nodes, there is a small error floor if we place message bits in the right-most $w-1$ sections.
To avoid this floor, we do not place message bits in these sections and instead incur a small rate loss.

%\vspace{-0.2cm}
\begin{table}[thb]
\begin{center}
\caption{No. of attempts for succ. encoding for $50$ codewords. 
  $(d_v,d_c,d'_v,d'_c)=(6,3,3,6)$, $(L,w)=(15,3)$, $(\beta,T)=(0.65,10)$.}
\label{table:succ_enc}
\vspace{-1mm}
\begin{tabular}{|c|c|c|}
\hline
Block length ($n$) & 4-cycles & $1/2/3/4/\geq 5$  \\
\hline
$9000$ & yes & $5/3/5/2/35$ \\
$9000$ & no & $21/12/5/3/9$ \\
$27000$ & no & $35/15/0/0/0$ \\
$45000$ & no & $40/9/0/0/1$ \\
$63000$ & no & $44/6/0/0/0$ \\
$81000$ & no & $50/0/0/0/0$\\ 
\hline  
\end{tabular}
\end{center}
\vspace{-0.65cm}
\end{table}

\begin{table}[thb]
\begin{center}
\caption{Thresholds for Wyner-Ziv problem with $(n\approx 140000,\beta=1.04,T=10)$.}
\label{table:wyner_ziv}
\vspace{-1mm}
\begin{tabular}{|p{0.8cm}|p{0.8cm}|p{0.7cm}|p{1.6cm}|p{1.65cm}|}
\hline
LDGM & LDPC & $(L,w)$ & $(D_{*},\delta_{*})$ & $(D,\delta)$ \\
$(d_v,d_c)$ & $(d'_v,d'_c)$ &  &  & \\
\hline
$(6,3)$ & (3,6) & (20,4)  & (0.111,0.134)  & (0.1174,0.122) \\
$(8,4)$ & (3,6) & (20,4)  & (0.111,0.134)  & (0.1149,0.120) \\
$(10,5)$ & (3,6) & (20,4)  & (0.111,0.134)  & (0.1139,0.122) \\
\hline  
\end{tabular}
\end{center}
\vspace{-0.65cm}
\end{table}

\begin{table}[thb]
\begin{center}
\caption{Thresholds for Gelfand-Pinsker problem with $(n\approx 140000,\beta=0.65,T=10)$.} 
\label{table:gelfand_pinsker}
\vspace{-1mm}
\begin{tabular}{|p{1.4cm}|p{0.8cm}|p{0.8cm}|p{1.4cm}|p{1.4cm}|}
\hline
LDGM & LDPC & $(L,w)$ & $(p_{*},\delta_{*})$ & $(p,\delta)$ \\
$(d_v,d_c)$ & $(d'_v,d'_c)$ &  &  & \\
\hline
$(6,3)$ & (3,6) & (20,4)  & (0.215,0.157)  & (0.220,0.152) \\
$(8,4)$ & (3,6) & (20,4)  & (0.215,0.157)  & (0.223,0.151) \\
$(10,5)$ & (3,6) & (20,4)  & (0.215,0.157)  & (0.220,0.151) \\
\hline
\end{tabular}
\end{center}
\vspace{-0.5cm}
\end{table}

%%% Local Variables: 
%%% mode: latex
%%% TeX-master: "isit14"
%%% End: 

\label{section:coupling_mp}
In this section, we describe coding schemes for the side-information problems based on spatially-coupled compound LDGM/LDPC codes.
The encoding and decoding are based on practically implementable, polynomial time message-passing algorithms.
In essence, we describe new codebooks $\mathcal{SC}$, $\mathcal{SC}(s^k)$, $\mathcal{SC}'$, $\mathcal{SC}'(s^k)$, analogous to $\mathcal{C}$, $\mathcal{C}(s^k)$, $\mathcal{C}'$, $\mathcal{C}'(s^k)$, that appear to be good for rate distortion and/or channel coding when encoding and decoding is done with message-passing algorithms  \cite{Aref-arxiv13}, \cite{Aref-isit12}, \cite{Obata-isit13}, \cite{Sakaniwa-arxiv11}.
Given that these are good for rate distortion and/or channel coding, the coding scheme for the side-information problems is same as in the previous section.
We begin with the construction of spatially-coupled compound codes.

\subsection{Spatially-Coupled Compound LDGM/LDPC Codes}
\begin{figure}[!tb]
  \centering
  \setlength\tikzheight{5cm}
  \setlength\tikzwidth{6cm} 
  \begin{tikzpicture}
  [yscale=0.8,
  node distance = 12mm, draw=black, thick, >=stealth',
  bitnode/.style={circle, inner sep = 0pt, minimum size = 4mm, draw=black},
  bitnode2/.style={circle, inner sep = 0pt, minimum size = 4mm, draw=black, fill=gray},
  checknode/.style={rectangle, inner sep = 0pt, minimum size = 4mm, draw=black},
  ]

  \foreach \x in {1,...,6} {
    \node[bitnode2] (bu\x) at (\x,3) {};
  }

  \foreach \x in {1,...,6} {
    \node[checknode] (cu\x) at (\x,2) {};
  }

  \foreach \x in {1,2} {
    \node[bitnode,pattern=north east lines] (b\x) at (\x,0) {$0$};
  }

  \foreach \x in {3,...,8} {
    \node[bitnode] (b\x) at (\x,0) {};
  }

  \foreach \x in {1,...,6} {
    \node[checknode] (cl\x) at (\x+2,-2) {};
  }

  \foreach \x in {1,...,6} {
    \draw (bu\x) -- (cu\x);
  }

  \foreach \x/\y/\z in {1/2/3,2/3/4,3/4/5,4/5/6,5/6/7,6/7/8} {
    \draw (cu\x) -- (b\x);
    \draw (cu\x) -- (b\y);
    \draw (cu\x) -- (b\z);
  }

  \foreach \x/\y/\z in {1/2/3,2/3/4,3/4/5,4/5/6,5/6/7,6/7/8} {
    \draw (cl\x) -- (b\x);
    \draw (cl\x) -- (b\y);
    \draw (cl\x) -- (b\z);
  }

  \draw[<->] (1,-2.5) -- node[above]{\footnotesize{$w-1$}} (2,-2.5);
  \draw[<->] (1,-2.85) -- node[midway,fill=white]{\footnotesize{$L$}} (6,-2.85);
  \draw[<->] (1,-3.3) -- node[midway,fill=white]{\footnotesize{$L+w-1$}} (8,-3.3);

\end{tikzpicture}

%%% Local Variables: 
%%% mode: latex
%%% TeX-master: "../isit14"
%%% End: 

  \vspace{-2.5mm}
  \caption{Illustration of edge connections in a spatially-coupled compound LDGM/LDPC code.
    LDPC bit-nodes in the first $w-1$ groups are set to $0$.
  }
  \label{figure:protograph_coupled_compound}
\end{figure}
For convenience, we assume regular degrees in both LDGM and LDPC components of the compound code.
The construction can be easily generalized to irregular degrees.
Suppose $d_c$ and $d_v$ denote the degrees of the check- and bit-nodes, respectively, in the LDGM component.
Similarly, define $d'_v$ and $d'_c$ for the LDPC component.
See Fig.~\ref{figure:tanner_graph_compound_code} for an illustration.
Consider $L+w-1$ groups of LDPC bit-nodes at positions $\mathcal{N}_v=\{1,2,\ldots,L+w-1\}$, $L$ groups of LDGM check-nodes at positions $\mathcal{N}_c=\{1,\ldots,L\}$, $L$ groups of LDPC check-nodes at positions $\mathcal{N}'_c=\{w,2,\ldots,L+w-1\}$.
See Fig.~\ref{figure:protograph_coupled_compound}.

Choose $N$ large enough so that $N d_v/d_c, N d_v /w$, $N d'_v/d'_c, N d_v'/w \in \mathbb{N}$.
We first show the coupling structure in the LDGM part.
At each position $i \in \mathcal{N}_v$, place $N$ LDPC bit-nodes each with $d_v$ edge sockets.
Similarly, at each position $j \in \mathcal{N}_c$, place $N d_v/d_c$ LDGM check-nodes each with $d_c$ edge sockets.
At each bit- and check-node group, partition the $N d_v$ edge sockets into $w$ groups using a uniform random permutation, and denote these partitions, respectively, by $\mathcal{N}_{i,k}^v$, $\mathcal{N}^c_{j,k}$ where $1 \leq i \leq L+w-1$, $1 \leq j \leq L$, $1 \leq k \leq w$.
The coupled LDGM component is constructed by connecting the sockets in $\mathcal{N}^c_{j,k}$ to sockets in $\mathcal{N}^v_{j+k-1,k}$.
The coupling structure in the LDPC part can be constructed similarly.
These connections are depicted in Fig.~\ref{figure:protograph_coupled_compound} for $L=6$ and $w=3$.

We note that such a construction leaves some edge sockets of the LDPC bit-nodes at the boundary unconnected.
We also shorten the LDPC bit-nodes in the first $w-1$ groups to $0$.
This shortening and the unconnected edge sockets at the boundary are necessary for the spatial-coupling phenomenon to take effect \cite{Kudekar-it11}.
It is implicit that the LDPC check-nodes at each position in $\mathcal{N}'_c$ are partitioned into two groups $\mathcal{P}_1$ and $\mathcal{P}_2$ so as to have the desired design rate.
To avoid small error floors when decoding, each LDPC bit-node should have some connections to the checks in $\mathcal{P}_2$.
The characterization of the codebooks $\mathcal{SC}$, $\mathcal{SC}(s^k)$, $\mathcal{SC}'$, $\mathcal{SC}'(s^k)$ follows implicitly.

\subsection{Message-Passing Algorithms for Encoding}
\begin{algorithm}[!t]
\begin{algorithmic}
  \caption{Belief-Propagation Guided Decimation}
  \label{algorithm:bpgd}
  \REQUIRE{Sequence $x^n \in \{0,1\}^n$ to encode, parameters ($T$, $\beta$), graph $G(V,U,C)$.}
  \STATE Set $m_{i \to a}\!=\!\wh{m}_{a \to i}\!=\!0$ for $i \in V \cup U$, $a \in C$ and $(i,a) \in G$.
  \STATE Initialize $V_{\mathrm{dec}}$ to be LDPC bit-nodes in first $w-1$ sections.
  \WHILE{$V_{\mathrm{dec}} \neq V$}
    \FOR {$t=1$ to $T$}
      \STATE $m_{i \to a} = (-1)^{x_i} \tanh(\beta)$ for $i \in U$ and $a \in C$.
      \STATE $m_{i \to a} = (-1)^{u_i} \cdotp \infty$ for $i \in V_{\mathrm{dec}}$ and $a \in C$.
      \STATE $m_{i \to a} = \sum\limits_{b \in \partial i \setminus \{a\}} \wh{m}_{b \to i}$ for $i \in V \setminus V_{\mathrm{dec}}$ and $a \in C$.
      \STATE $\wh{m}_{a \to i} = \tanh^{-1} \prod\limits_{j \in \partial a \setminus \{ i \} } \tanh m_{j \to a}$ for $i \in V \setminus V_{\mathrm{dec}}$ and $a \in C$.
    \ENDFOR
    \STATE Evaluate $m_{i}=\sum_{a \in \partial i} \wh{m}_{a \to i} $ for all $i \in V\setminus V_{\mathrm{dec}}$.
    \STATE Set $B$ to be max. of $|m_i|$ when $i$ varies over left-most $w$ sections of $V \setminus V_{\mathrm{dec}}$; denote the resulting bit-node by $i^*$.
    \IF{$B=0$}
      \STATE Pick a bit-node $i^*$ uniformly in left-most $w$ sections of $V \setminus V_{\mathrm{dec}}$ and set $u_{i^*}$ to be $0$ or $1$ uniformly randomly.
    \ELSE
      \STATE Set $u_{i^*}$ to $0$ or $1$ with prob. $\frac{1+\tanh m_{i^*}}{2}$ or $\frac{1-\tanh m_{i^*}}{2}$.
    \ENDIF
    \STATE Set $V_{\mathrm{dec}} = V_{\mathrm{dec}} \cup \{i^*\}$.
  \ENDWHILE
  \STATE If $\{u_{i}\}$ fail to satisfy LDPC checks, then \textbf{re-encode}.
\end{algorithmic}
\end{algorithm}

Message-passing algorithms for channel coding are now a standard part of the coding theory literature.
We refer the reader to \cite{RU-2008} for their description.
It has been recently shown \cite{Obata-isit13}, \cite{Sakaniwa-arxiv11} that SC compound LDGM/LDPC codes are good for channel coding under message-passing.
As such, $\mathcal{SC}$, $\mathcal{SC}(s^k)$, $\mathcal{SC}'(s^k)$ are good for channel coding under message-passing.
In the following, we focus exclusively on the message-passing algorithm for rate distortion.
In particular, we describe a variation of the so-called BPGD algorithm \cite{Aref-arxiv13}, \cite{Aref-isit12}.

Consider an instance of the spatially-coupled compound LDGM/LDPC described above.
Denote its Tanner graph by $G(V,U,C)$, where $V$ denotes the LDPC bit-nodes, $U$ denotes the LDGM bit-nodes, $C$ denotes the check-nodes (both LDGM and LDPC).
Place a sequence $x^n \in \{0,1\}^n$ at the top of LDGM bit-nodes as in Fig.~\ref{figure:tanner_graph_compound_code}.
The message-passing rules here are same as in the channel coding setup by assuming that $x_i$ have come through a BSC channel (parameterized by $\beta$).
However, every $T$ iterations, an LDPC bit-node is decimated (shortened based on the current LLR).
This encoding procedure for the codebook $\mathcal{C}(s^k)$ is described in Algorithm \ref{algorithm:bpgd} assuming $s^k = 0^k$.
For $s^k \neq 0^k$, the update for each check node in $\mathcal{P}_1$ is modified to include the appropriate $s_k$ hard-decision message.
Also, the decimated sequence $u^n$ may not satisfy all the LDPC check constraints.
In this case, successive encoding attempts often result in a valid codeword due to the randomization in Algorithm \ref{algorithm:bpgd}.
In practice, we found that removing double edges and 4-cycles from the code essentially eliminated this problem at moderate block lengths.
For example, see the results in Table \ref{table:succ_enc}.

One variation in Algorithm \ref{algorithm:bpgd} from \cite{Aref-arxiv13} is the choice of the LDPC bit-node for decimation.
In \cite{Aref-arxiv13}, bit-node with maximum bias over entire graph is selected, but we restrict the search for maximum biased bit to only left-most $w$ sections of the bits that are not already decimated.
We observed that this change increases the chances of encoding to a valid codeword.
\begin{remark}
  The BPGD algorithm, when applied to the \emph{uncoupled} compound LDGM/LDPC code, always failed to satisfy the LDPC check constraints and the spatial-coupling structure is required to overcome this problem.
  Thus, spatial coupling in compound codes not only helps to reduce the distortion, but allows the BPGD algorithm to encode to a valid codeword.
\end{remark}
It is also possible to create a coupling structure in a circular fashion and decimate the bit-nodes in a fixed section as done in \cite{Aref-isit12}.
However, this leads to a \emph{wave} of decimations from both ends and will result in a failure to encode to a codeword, as the constraints imposed by the LDPC check-nodes are unlikely to match when the two waves meet.
From a physics point of view, this is akin to growing a crystal on a torus from a single seed.
When the two growth interfaces meet, they are very unlikely to mesh nicely and form a pure crystal.

%%% Local Variables: 
%%% mode: latex
%%% TeX-master: "isit14"
%%% End: 

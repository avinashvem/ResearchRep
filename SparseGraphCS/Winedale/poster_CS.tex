\documentclass[final]{beamer}
\usepackage[orientation=portrait, scale=1.24,10pt]{beamerposter}

%Extra 1st line tesing comments

%\usepackage{amsmath,amssymb,amsthm}
%\usepackage{stmaryrd}
\usepackage{amsmath,amssymb,amsthm}
\usepackage{enumerate}
\usepackage{stfloats}


\usepackage{graphics} % for pdf, bitmapped graphics files
\usepackage{epsfig} % for postscript graphics files
\usepackage{mathptmx} % assumes new font selection scheme installed
\usepackage[mathscr]{euscript}
\usepackage{algorithm}
\usepackage[noend]{algpseudocode}
\makeatletter
\def\BState{\State\hskip-\ALG@thistlm}
\makeatother

\usepackage{tikz}
\usetikzlibrary{arrows,shapes,chains,matrix,positioning,scopes,patterns}
\usepackage{pgfplots}
\usepgflibrary{shapes}

\newtheorem{theorem}{Theorem}
\newtheorem{lemma}[theorem]{Lemma}
\newtheorem{proposition}[theorem]{Proposition}
\newtheorem{definition}[theorem]{Definition}
\newtheorem{example}[theorem]{Example}
\newtheorem{remark}[theorem]{Remark}
\newtheorem{corollary}[theorem]{Corollary}

\renewcommand{\epsilon}{\varepsilon}

\newcommand{\h}{\texttt{h}}
\newcommand{\hbp}{\h^{\mathrm{BP}}}
\newcommand{\hmap}{\h^{\mathrm{MAP}}}
\newcommand{\hstab}{\h^{\mathrm{stab}}}
\newcommand{\harea}{\h^{A}}

\newcommand{\expt}{\mathbb{E}}
\newcommand{\indicator}[1]{\mathbbm{1}_{\left\{ {#1} \right\} }}
\newcommand{\abs}[1]{\left\lvert#1\right\rvert}

\newcommand{\mb}[1]{\mathbf{#1}}
\newcommand{\mbb}[1]{\mathbb{#1}}
\newcommand{\mr}[1]{\mathrm{#1}}
\newcommand{\mc}[1]{\mathcal{#1}}
\newcommand{\ms}[1]{\mathsf{#1}}
\newcommand{\msc}[1]{\mathscr{#1}}
\newcommand{\mf}[1]{\mathfrak{#1}}

\newcommand{\RNum}[1]{\uppercase\expandafter{\romannumeral #1\relax}}

\newcommand{\mse}{\mathsf{e}}
\newcommand{\msx}{\mathsf{x}}
\newcommand{\msxvn}{\tilde{\mathsf{x}}}
\newcommand{\msy}{\mathsf{y}}
\newcommand{\msz}{\mathsf{z}}
\newcommand{\msa}{\mathsf{a}}
\newcommand{\msb}{\mathsf{b}}
\newcommand{\msbx}{\underline{\mathsf{x}}}
\newcommand{\msby}{\underline{\mathsf{y}}}
\newcommand{\msbxvn}{\tilde{\underline{\mathsf{x}}}}
\newcommand{\msbz}{\underline{\mathsf{z}}}
\newcommand{\msba}{\underline{\mathsf{a}}}
\newcommand{\msbb}{\underline{\mathsf{b}}}
\newcommand{\msbc}{\underline{\mathsf{c}}}

%\newcommand{\bv}{\underline{\mathrm{b}}}
%\newcommand{\xv}{\underline{\mathrm{x}}}
%\newcommand{\yv}{\underline{\mathrm{y}}}
%\newcommand{\zv}{\underline{\mathrm{z}}}
%\newcommand{\rv}{\underline{\mathrm{r}}}
%\newcommand{\wv}{\underline{\mathrm{w}}}

\newcommand{\bv}{\overrightarrow{\mathrm{b}}}
\newcommand{\xv}{\overrightarrow{\mathrm{x}}}
\newcommand{\yv}{\overrightarrow{\mathrm{y}}}
\newcommand{\zv}{\underline{\mathrm{z}}}
\newcommand{\rv}{\underline{\mathrm{r}}}
\newcommand{\wv}{\underline{\mathrm{w}}}


\newcommand{\Xv}{\underline{\mathrm{X}}}
\newcommand{\Yv}{\underline{\mathrm{Y}}}
\newcommand{\Zv}{\underline{\mathrm{Z}}}
\newcommand{\Rv}{\underline{\mathrm{R}}}
\newcommand{\RXYv}{\underline{\mathrm{R}_{XY}}}

\newcommand{\wh}{\widehat}
\newcommand{\bop}{\ast}
\newcommand{\vnop}{\varoast}
\newcommand{\disth}{d_{\mathrm{H}}}
\newcommand{\cnop}{\boxast}
\newcommand{\diff}[1]{d#1}
\newcommand{\deri}[1]{\mathrm{d}_{ #1 }\hspace{0.05cm}}
\newcommand{\dderi}[1]{\mathrm{d}_{ #1 }^2\hspace{0.05cm}}
\newcommand{\bvert}[1]{\,\Big{\vert}_{ #1  }}
\newcommand{\degr}{\succ}
\newcommand{\degreq}{\succeq}
\newcommand{\upgr}{\prec}
\newcommand{\upgreq}{\preceq}
\newcommand{\extR}{\overline{\mathbb{R}}}

\newcommand{\des}{\mathsf{T}_\mathrm{s}}
\newcommand{\dec}{\mathsf{T}_\mathrm{c}}
\newcommand{\pots}{U_\mathrm{s}}
\newcommand{\potc}{U_\mathrm{c}}
\newcommand{\shft}{\mathsf{S}}

\newcommand{\vnunit}{\Delta_0}
\newcommand{\cnunit}{\Delta_\infty}

\newcommand{\ent}[1]{ \mathrm{H} \left( #1 \right) }

\newcommand{\meass}{\mathcal{M}}
\newcommand{\probs}{\mathcal{X}}
\newcommand{\dpros}{\mathcal{X}_{\mathrm{d}}}
\newcommand{\chend}{N_{w}}

\newcommand{\minf}{\mathsf{a}_{0}}
\newcommand{\minfb}{\underline{\minf}}

\DeclareMathOperator*{\argmin}{\,arg\ min}
\DeclareMathOperator*{\argmax}{\,arg\ max}

\newlength\tikzwidth
\newlength\tikzheight

\textfloatsep=0.05in

\newcommand{\coleq}{\mathrel{\mathop:}=}
\newcommand{\defeq}{\triangleq}

%%% Local Variables:
%%% mode: plain-tex
%%% TeX-master: "isit14"
%%% End:


\newlength{\columnheight}
\newlength{\onecolwid}
\newlength{\twocolwid}
\newlength{\threecolwid}
\newlength{\halfcolwid}
\newlength{\subitemspace}
\newlength{\blockskip}
\newlength{\paraskip}
\newlength{\negativeskip}

\setlength{\columnheight}{105cm}
\setlength{\onecolwid}{0.32\paperwidth}
\setlength{\twocolwid}{0.32\paperwidth}
\setlength{\threecolwid}{0.32\paperwidth}
\setlength{\halfcolwid}{0.12\paperwidth}
\setlength{\topmargin}{0.5in}
\setlength{\subitemspace}{0.1cm}
\setlength{\blockskip}{2cm}
\setlength{\paraskip}{1cm}
\setlength{\negativeskip}{-6cm}
\usetheme{confposter}


\xpatchcmd{\itemize}
  {\def\makelabel}
  {\setlength{\itemsep}{7mm}\def\makelabel}
  {}
  {}
  
\definecolor{jblue}{rgb}{0,0.2,0.8}

\setbeamercolor{block title}{fg=ngreen,bg=white}
\setbeamercolor{block body}{fg=black,bg=white}
\setbeamercolor{block alerted title}{fg=white,bg=dblue!70}
\setbeamercolor{block alerted body}{fg=black,bg=dblue!10}

\setbeamertemplate{itemize items}[ball]

\newlength\figureheight 
\newlength\figurewidth

\newcommand{\mc}[1]{\mathcal{#1}}
\newcommand{\ms}[1]{\mathscr{#1}}
\newcommand{\mbb}[1]{\mathbb{#1}}
\newcommand{\mbf}[1]{\mathbf{#1}}
\newcommand{\tit}[1]{\textit{#1}}
\newcommand{\tbf}[1]{\textbf{#1}}
\newcommand{\tsc}[1]{\textsc{#1}}

\newcommand{\defeq}{\triangleq}
\newcommand{\coleq}{\mathrel{\mathop:}=}
\newcommand{\Pp}{\mathbb{P}}
\newcommand{\E}{\mathbb{E}}
\newcommand{\N}{\mathbb{N}}
\newcommand{\Z}{\mathbb{Z}}
\newcommand{\Zp}{\mathbb{Z}_{+}}
\newcommand{\R}{\mathbb{R}}
\newcommand{\Rp}{\R_{+}}
\newcommand{\g}{\mathbf{g}_}
\newcommand{\cp}{\times}
\newcommand{\Lmb}{\Lambda}
\newcommand{\lmb}{\lambda}
\newcommand{\tx}[1]{\text{#1}}

\newcommand{\Q}{\mathbb{Q}}
\newcommand{\F}{\mathbb{F}}
\newcommand{\Zw}{\mathbb{Z}[\omega]}
\newcommand{\Zi}{\mathbb{Z}[i]}
\newcommand{\C}{\mathcal{C}}

\newcommand{\abs}[1]{\lvert{#1}\rvert}
\newcommand{\card}[1]{\abs{#1}}
\newcommand{\norm}[1]{\lVert{#1}\rVert}
\newcommand{\iid}{i.\@i.\@d.\ }

\newcommand{\ceil}[1]{\lceil{#1}\rceil}
\newcommand{\floor}[1]{\lfloor{#1}\rfloor}

\DeclareMathOperator*{\argmax}{arg\,max}
\DeclareMathOperator*{\argmin}{arg\,min}

\theoremstyle{definition}\newtheorem{lemma}{Lemma}
\theoremstyle{definition}\newtheorem{proposition}[lemma]{Proposition}
\theoremstyle{definition}\newtheorem{theorem}[lemma]{Theorem}
\theoremstyle{definition}\newtheorem{corollary}[lemma]{Corollary}
\newtheorem{definition}[lemma]{Definition}
\newtheorem{Example}[lemma]{Example}
\newtheorem{Remark}[lemma]{Remark}
\newtheorem*{Discussion}{Discussion}
%\newtheorem{example}[theorem]{Example}



\newcommand{\ind}{{\rm 1\hspace*{-0.4ex}\rule{0.1ex}{1.52ex}\hspace*{0.2ex}}}
\newcommand{\dbar}[1]{\bar{\bar{#1}}}
\newcommand{\T}{\msf{T}}
\newcommand{\dotleq}{\mathrel{\dot{\leq}}}
\newcommand{\dotgeq}{\mathrel{\dot{\geq}}}
\newcommand{\dotl}{\mathrel{\dot{<}}}
\newcommand{\dotg}{\mathrel{\dot{>}}}
\newcommand{\from}{\colon}

\DeclareMathOperator{\rank}{rank}
\DeclareMathOperator{\tr}{tr}
\DeclareMathOperator{\cl}{cl}
\DeclareMathOperator{\diag}{diag}
\DeclareMathOperator{\conv}{conv}
\DeclareMathOperator{\Bernoulli}{Bernoulli}
\DeclareMathOperator{\Ei}{Ei}
\DeclareMathOperator{\OR}{OR}
\DeclareMathOperator{\Geom}{Geom}
\DeclareMathOperator{\var}{var}

\title{Sub-linear Time Compressed Sensing and Group Testing via Left-and-Right-Regular Sparse-Graph Codes}
\author{\textbf{Avinash Vem}, Nagaraj T. Janakiraman, Krishna R. Narayanan}
\institute{ECE Dept.,Texas A\&M University}

\begin{document}
\begin{frame}
\begin{columns}[t]
  % First Column
  \begin{column}{\onecolwid}
\vspace{\negativeskip}
    % Introduction
  \begin{block}{\Large Compressed Sensing(CS)}
Recover a sparse signal $\mbf{x}$ from $\mbf{y}$:  $~\mathbf{y=Ax +w}$
	 \begin{itemize}
		\item $\mbf{x}$ -$N \times 1$ sparse signal
		\item $\mbf{A}$ -$M \times N$ measurement matrix
		\item $\mbf{w}$ -additive Gaussian noise
		\item $\mbf{y}$ -$M \times 1$ measurement vector
		\item $\card{\{i: x_i\neq 0\}}=K. ~~ K<<N$
	\end{itemize}
	\vspace{\paraskip}

Metric of interest:
		\begin{itemize}
			\item Prob. of failure of support recovery $\mbb{P}_{F}\coleq \text{Pr}(\text{supp}(\hat{\mbf{x}})\neq \text{supp}(\mbf{x}))$
		\end{itemize}     
    \end{block}
\vspace{\blockskip}    

\begin{block}{\Large Group Testing(GT)}
	Recover a sparse signal $\mbf{x}$ from $\mbf{y}$:  $~\mathbf{y=A\odot x}$
	 \begin{itemize}
		\item $\mbf{x, y, A}$ are binary vectors/matrix respectively
		\item $\odot$: Matrix multiplication with \textbf{``binary} OR'' instead of addition
		\item $\mbf{y}$- $M \times 1$ measurement vector
	\end{itemize}
	\vspace{\paraskip}
	
Metric of interest:   
    \begin{itemize}
 		\item Prob. of miss: $\mbb{P}_{m}\coleq \text{Pr}(\hat{x}_i=0, x_i=1)$
	\end{itemize}     
	
 \end{block}

 
    \begin{block}{\Large Known Bounds for CS}
	\begin{itemize}
%	\item In 2007, Wainwright gave information theoretic limits for compressed sensing: support recovery
	 \item For sub-linear sparsity, $K=o(N)$, $M=\Theta\left( K\log(\frac{N}{K	})\right)$ is shown to be necessary and sufficient.
	  \item In the linear sparse regime, $K=\alpha N$, it was shown that $M=\Theta(N)$ measurements are sufficient. 
	  \item In \cite{li2015subisit}, Li \textit{et al} using left-regular sparse graph codes proposed a scheme that achieves \textcolor{red}{$O(K\log N)$} sample and decoding complexity
	 \end{itemize}
	 
   \begin{alertblock}{\Large Main Result (CS)} 
	    \textbf{Sub-linear sparsity: }For a given SNR, our scheme has 
			\begin{itemize}
			\itemsep10pt
				\item Sample complexity of $M=c_1$ \textcolor{red}{$K\log (\frac{c_2 N}{K})$}
				\item Decoding complexity of \textcolor{red}{$O\left(K\log(\frac{N}{K})\right)$} 
				\item $\mbb{P}_{\text{F}}\rightarrow 0$ asymptotically in $K$ and $N$
			\end{itemize} 
\vspace{\paraskip}    
  
   \textbf{Linear sparsity: }Our scheme has 
		\begin{itemize}
		\itemsep10pt
			\item Sample complexity of $M=c_3 K\log K$
			\item Decoding complexity of \textcolor{red}{$O\left(K\log(\frac{N}{K})\right)$}
			\item $\mbb{P}_{\text{F}}\rightarrow 0$ asymptotically in $K$ and $N$
		\end{itemize} 
    \end{alertblock}
 \end{block}
  \vspace{\blockskip}   
    
    \begin{block}{\Large Bounds for Group Testing}
		\begin{itemize}
			 \item We assume all the $\binom{N}{K}$ $K$-sparse sets are equi-probable
			 \item At least $\log_2 \binom{N}{K}$ tests are necessary
			 \item For large $K$ and $N$, $\log_2 \binom{N}{K}\approx K\log_2(\frac{N}{K})$
	 \end{itemize}
	 
	 
   \begin{alertblock}{\Large Main Result (GT)} 
	    \textbf{Sub-linear sparsity: }Let $K\in o(N^{\frac{p}{p+1}})$, for some $p\in\mathbb{Z}$
			\begin{itemize}
			\itemsep10pt
				\item Number of tests  $M=2(p+1)c(\epsilon)$ \textcolor{red}{$K\log_2 (\frac{c_1N}{K})$}
				\item Decoding complexity of $O\left(K\log(\frac{N}{K})\right)$ 
				\item $\mbb{P}_{m}\leq \epsilon$ w.h.p. asymptotically in $K$ and $N$
				\item e.g. for $\epsilon=10^{-5}$, $c(\epsilon)=9.63$
			\end{itemize}
    	\end{alertblock}
	 \end{block}
  \end{column}

  % Second Column
  \begin{column}{\twocolwid}
    \vspace{\negativeskip}

    \begin{block}{\Large Idea: Divide-and-Conquer}
    \begin{itemize}
		\item Original problem is $K$-sparse CS/GT  
    	\item Divide $N$ nodes into non-disjoint bins
	    \item Can you solve for 1-sparse CS/GT at a bin?
    \end{itemize}

    \begin{figure}
       \center
		\scalebox{1.3}{\begin{tikzpicture}
\def\horzgap{1.25in}; %Horizontal gap between nodes/levels
\def \gapVN{0.75in}; %vertical gap between nodes
\def \gapCN{1.0in}; %Horizontal gap between nodes



\def \textoffs{0.6in}; %Offset for writing text above a node
\def\nodewidth{0.25in};
\def\nodewidthA{0.25in};
\def \edgewidth{0.1in};
\def\ext{1.0in};
\def\linewidth{0.2mm}

\def \n {8};
\def\ldeg{3};
\def \m {4};
\def\rdeg{6};
\def\langle{40};%120 degrees/3
\def\langle{20};%120 degrees/6
\def\fsize{\small};

\tikzstyle{check} = [rectangle, draw,line width=0.05mm,  inner sep=0mm, fill=black, minimum height=\nodewidthA, minimum width=\nodewidthA]
\tikzstyle{checksm} = [rectangle, draw, line width=0.05mm, inner sep=0mm, fill=blue,minimum height=\edgewidth,minimum width=\edgewidth]

\tikzstyle{bit} = [circle, draw,line width=0.05mm, inner sep=0mm, fill=red, minimum size=\nodewidthA]
\tikzstyle{bitsm} = [circle, draw, very thin, inner sep=0mm,fill=red, minimum size=\edgewidth]
\tikzstyle{edgesock} = [circle, inner sep=0mm, minimum size=\edgewidth,draw, fill=white]     

                          
\foreach \vn in {1,...,6}{
 \node[bit] (vn\vn) at (0,\vn*\gapVN) {};
}

\foreach \cn in {1,...,4}{
\node[check] (cn\cn) at (\horzgap,\cn*\gapCN) {};
}

\node at (0.5*\horzgap,0){\fsize{$K$-sparse CS/GT}};
\node[anchor=west,above right](cn2text) at (cn2.east) {\fsize{Identify and decode}};
\path (cn2text) ++(0,-2*\nodewidth) node() {\fsize{a $1$-sparse CS/GT?}};

\draw[line width=\linewidth] (vn6.east)--(cn4.west);
\draw[line width=\linewidth] (vn3.east)--(cn4.west);
\draw[line width=\linewidth] (vn1.east)--(cn4.west);

\draw[line width=\linewidth] (vn2.east)--(cn3.west);
\draw[line width=\linewidth] (vn4.east)--(cn3.west);
\draw[line width=\linewidth] (vn5.east)--(cn3.west);

\draw[line width=\linewidth] (vn2.east)--(cn3.west);
\draw[line width=\linewidth] (vn4.east)--(cn3.west);
\draw[line width=\linewidth] (vn5.east)--(cn3.west);

\draw[line width=\linewidth] (vn6.east)--(cn2.west);
\draw[line width=\linewidth] (vn5.east)--(cn2.west);
\draw[line width=\linewidth] (vn3.east)--(cn2.west);

\draw[line width=\linewidth] (vn1.east)--(cn1.west);
\draw[line width=\linewidth] (vn2.east)--(cn1.west);
\draw[line width=\linewidth] (vn4.east)--(cn1.west);
\end{tikzpicture}

%
}
	\end{figure}

    \end{block}
    \vspace{\blockskip}

    \begin{block}{\Large Divide: ($\ell,r$) Bipartite Graph}
    \begin{itemize}
	    \item $N$ Variable(left) nodes: $x_i$. Each node has (left) degree: $\ell$
	    \item Bin(right) nodes: Choose $M_1=cK$ bins (sub problems)
	    \item  Each bin has (right) degree $r$. Gives $r=\frac{N\ell}{cK}$
	    \item Connections between $N\ell$ edges on each side are \textcolor{red}{random}
    \end{itemize}
    \end{block}  
\vspace{\blockskip}

    \begin{block}{\Large Conquer: 1-sparse CS}
	    \begin{itemize}
    		\item At each bin, use code words of an error control code $\mathcal{C}$:
		    	\begin{equation*}
				    \mbf{y}_i=x_{i1}\mbf{c}_1 +x_{i2}\mbf{c}_{2}+\ldots x_{ir}\mbf{c}_{r}+\mbf{w}_i
			    \end{equation*}
    		\item If it is 1-sparse i.e. only one $x_{ik}\neq 0$: (call it \textcolor{red}{singleton}):
		        \begin{equation*}
				    \mbf{y}_i=x_{ik}\mbf{c}_k +\mbf{w}_i
		        \end{equation*}
		    \item A \textcolor{red}{channel coding problem} if sign($x_{ik}$) is known
		    \item  From channel coding, dim($\mbf{c}_j$)>$\frac{\log r}{C_{\text{Sh}}}\approx \Theta(\log \frac{N}{K})$ 
%   		    \item $x\in\mathcal{X}\coleq\{Ae^{j\theta}: A\in \mc{A},\theta\in \Theta\}$ be a large but finite set
           \item After decoding index $k$, need to decode the value of $x_{ik}$
		    \item Let $x\in\mathcal{X}\coleq\{\pm A: A\in \mc{A}\}$ be a \textcolor{red}{discrete and finite set}
		    \item We decode $\hat{x}_{ik}\in\mathcal{X}$ to be the value that best fits $\mbf{y}_i\mbf{c}_{k}^{\dagger}/||\mbf{c}_{k}||^{2}$
		    %$\frac{\mbf{y}_i\mbf{c}_{k}^{\dagger}}{||\mbf{c}_{k}||^{2}}$
    	\end{itemize}
    \end{block}      
\vspace{\blockskip}


 \begin{block}{\Large Reconstruction via Peeling}
    \begin{itemize}
	   \item Assume we can conquer the 1-sparse CS/GT at a bin
	   \item If a singleton bin is found, decode the index and the value
	   \item \textcolor{red}{Peel off} the decoded variable node value from other bins
   \end{itemize}
    
       \begin{figure}
	       \centering
	       \begin{subfigure}{.5\textwidth}
			  \centering
   			\scalebox{1.1}{\begin{tikzpicture}
\def\horzgap{1.5in}; %Horizontal gap between nodes/levels
\def \gapVN{0.9in}; %vertical gap between nodes
\def \gapCN{1.2in}; %Horizontal gap between nodes


\def\nodewidth{0.3in};
\def\nodewidthA{0.3in};
\def \edgewidth{0.12in};
\def\ext{1.2in};

\def\fsize{\small};

\tikzstyle{check} = [rectangle, draw,line width=0.2mm,  inner sep=0mm, fill=black, minimum height=\nodewidthA, minimum width=\nodewidthA]
\tikzstyle{bit} = [circle, draw, line width=0.2mm, inner sep=0mm, fill=ProcessBlue, minimum size=\nodewidthA]
\tikzstyle{bituncover} = [circle, draw=none, line width=0.2mm, inner sep=0mm, fill=gray, minimum size=\nodewidthA]

\tikzstyle{edgesock} = [circle, inner sep=0mm, minimum size=\edgewidth,draw, fill=white]     

                          
\foreach \vn in {1,...,6}{
 \node[bit] (vn\vn) at (0,-\vn*\gapVN) {};
}

\foreach \vn in {2,4,5}{
\path (vn\vn) ++(-\nodewidth,0) node()[inner sep=0mm] {\fsize{0}};
}

\node[anchor=east, left] at (vn1.west){\fsize{$x_{1}$}};
\node[anchor=east, left] at (vn3.west){\fsize{$x_{3}$}};
\node[anchor=east, left] at (vn6.west){\fsize{$x_{6}$}};

\foreach \cn in {1,...,4}{
\node[check] (cn\cn) at (\horzgap,-\cn*\gapCN) {};
}

\draw[line width=0.05mm] (vn6.east)--(cn4.west);
\draw[line width=0.05mm] (vn3.east)--(cn4.west);
\draw[line width=0.05mm] (vn1.east)--(cn4.west);

\draw[line width=0.05mm] (vn2.east)--(cn3.west);
\draw[line width=0.05mm] (vn4.east)--(cn3.west);
\draw[line width=0.05mm] (vn5.east)--(cn3.west);

\draw[line width=0.05mm] (vn2.east)--(cn3.west);
\draw[line width=0.05mm] (vn4.east)--(cn3.west);
\draw[line width=0.05mm] (vn5.east)--(cn3.west);

\draw[line width=0.05mm] (vn6.east)--(cn2.west);
\draw[line width=0.05mm] (vn5.east)--(cn2.west);
%\draw[line width=0.05mm] (vn3.east)--(cn2.west);
\draw[line width=0.05mm] (vn1.east)--(cn2.west);

\draw[line width=0.05mm] (vn1.east)--(cn1.west);
\draw[line width=0.05mm] (vn2.east)--(cn1.west);
\draw[line width=0.05mm] (vn4.east)--(cn1.west);

\def\moveX {1.8*\nodewidth};
\def\moveXA {2*\nodewidth};

\node(cn1text)[anchor=west,right] at (cn1.east) {\fsize{$x_1\mathbf{c}_1+\mathbf{w_{1}}$}};
\node[anchor=north,below] at (cn1text.south) {\fsize{\textcolor{red}{Decode $x_1$ first}}};
\node[anchor=west,right](cn2text) at (cn2.east){\fsize{$x_1\mathbf{c}_1+x_6\mathbf{c}_6$}};
\node[below] at (cn2text.south) {\fsize{$+\mathbf{w_{2}}$}};
\node[anchor=west,right] at (cn3.east) {\fsize{0+$\mathbf{w_{3}}$}};
\node[anchor=west,right](cn4text) at (cn4.east) {\fsize{$x_1\mathbf{c}_1+x_3\mathbf{c}_3$}};
\node[anchor=north, below] at (cn4text.south) {\fsize{$+x_6\mathbf{c}_6+\mathbf{w_{4}}$}};
%-----------------------*(&^#@$^&*(^%$^&*(&^--------------------------------------------------------------------

\end{tikzpicture}}
		   \end{subfigure}%
		  \begin{subfigure}{.5	\textwidth}
		  \centering
		  \scalebox{1.1}{\begin{tikzpicture}
\def\horzgap{1.5in}; %Horizontal gap between nodes/levels
\def \gapVN{0.9in}; %vertical gap between nodes
\def \gapCN{1.2in}; %Horizontal gap between nodes


\def\nodewidth{0.3in};
\def\nodewidthA{0.3in};
\def \edgewidth{0.12in};
\def\ext{1.2in};

\def\fsize{\small};

\tikzstyle{check} = [rectangle, draw,line width=0.2mm,  inner sep=0mm, fill=black, minimum height=\nodewidthA, minimum width=\nodewidthA]
\tikzstyle{bit} = [circle, draw, line width=0.2mm, inner sep=0mm, fill=ProcessBlue, minimum size=\nodewidthA]
\tikzstyle{bituncover} = [circle, draw=none, line width=0.2mm, inner sep=0mm, fill=gray, minimum size=\nodewidthA]

\tikzstyle{edgesock} = [circle, inner sep=0mm, minimum size=\edgewidth,draw, fill=white]     

                          
\foreach \vn in {2,...,6}{
 \node[bit] (vn\vn) at (0,-\vn*\gapVN) {};
}

\node[bituncover] (vn1) at (0,-1*\gapVN) {};
 
\foreach \vn in {2,4,5}{
\path (vn\vn) ++(-\nodewidth,0) node()[inner sep=0mm] {\fsize{0}};
}

\node[anchor=east, left] at (vn1.west){\fsize{$\hat{x}_{1}$}};
\node[anchor=east, left] at (vn3.west){\fsize{$x_{3}$}};
\node[anchor=east, left] at (vn6.west){\fsize{$x_{6}$}};

\foreach \cn in {1,...,4}{
\node[check] (cn\cn) at (\horzgap,-\cn*\gapCN) {};
}

\draw[line width=0.05mm] (vn6.east)--(cn4.west);
\draw[line width=0.05mm] (vn3.east)--(cn4.west);
%\draw[line width=0.05mm] (vn1.east)--(cn4.west);

\draw[line width=0.05mm] (vn2.east)--(cn3.west);
\draw[line width=0.05mm] (vn4.east)--(cn3.west);
\draw[line width=0.05mm] (vn5.east)--(cn3.west);

\draw[line width=0.05mm] (vn2.east)--(cn3.west);
\draw[line width=0.05mm] (vn4.east)--(cn3.west);
\draw[line width=0.05mm] (vn5.east)--(cn3.west);

\draw[line width=0.05mm] (vn6.east)--(cn2.west);
\draw[line width=0.05mm] (vn5.east)--(cn2.west);
\draw[line width=0.05mm] (vn3.east)--(cn2.west);

%\draw[line width=0.05mm] (vn1.east)--(cn1.west);
\draw[line width=0.05mm] (vn2.east)--(cn1.west);
\draw[line width=0.05mm] (vn4.east)--(cn1.west);

\def\moveX {1.8*\nodewidth};
\def\moveXA {2*\nodewidth};

\node(cn1text)[anchor=west,right] at (cn1.east) {\fsize{$\mathbf{w_{1}}$}};
\node[anchor=west,right](cn2text) at (cn2.east){\fsize{$x_6\mathbf{c}_6+\mathbf{w_{2}}$}};
\node[anchor=north,below] at (cn2text.south) {\fsize{\textcolor{red}{Decode $x_6$ now}}};
\node[anchor=west,right] at (cn3.east) {\fsize{0+$\mathbf{w_{3}}$}};
%\node[anchor=west,right] at (cn4.east) {\fsize{$x_3\mathbf{c}_3+x_6\mathbf{c}_6+\mathbf{w_{4}}$}};
\node[anchor=west,right](cn4text) at (cn4.east) {\fsize{$x_3\mathbf{c}_3+$}};
\node[anchor=north, below] at (cn4text.south) {\fsize{$x_6\mathbf{c}_6+\mathbf{w_{4}}$}};

%-----------------------*(&^#@$^&*(^%$^&*(&^--------------------------------------------------------------------

\end{tikzpicture}}
			\end{subfigure}	
		\end{figure}
\vspace{\paraskip}	 
	 \begin{itemize}
	   \item Continue peeling iteratively until no new singletons are found
	   \item $\exists$ threshold $c_*$ such that for $M_1>c_*K$ bins, \textcolor{blue}{peeling decoder} recovers all nodes w.h.p asymptotically
    \end{itemize}
    \end{block}  
 \end{column}

  % Third Column
  \begin{column}{\threecolwid}
      \vspace{\negativeskip}
    \begin{block}{\Large Conquer: 1 and 2-sparse GT}
	    \begin{itemize}
 	        \item Let $\mbf{b}_k$ be the binary expansion of $k$; $\overline{\mbf{b}}_k$ it's complement
    		\item At each bin, test result vector would be:
		    	\begin{equation*}
				    \mbf{y}_i=x_{i1}\begin{bmatrix}
				    \mbf{b}_1\\
				    \overline{\mbf{b}}_1
				    \end{bmatrix}  \lor x_{i2}\begin{bmatrix}
				    \mbf{b}_2\\
				    \overline{\mbf{b}}_2
				    \end{bmatrix}\lor \ldots \lor x_{ir}\begin{bmatrix}
				    \mbf{b}_r\\
				    \overline{\mbf{b}}_r
				    \end{bmatrix}
			    \end{equation*}
    		\item If it is 1-sparse i.e. only one $x_{ij}=1$, trivial to decode $j$:
		        \begin{equation*}
				    \mbf{y}_i=\begin{bmatrix}
				    \mbf{b}_k\\
				    \overline{\mbf{b}}_k
				    \end{bmatrix}
		        \end{equation*}
			\item \textcolor{red}{Non-linear OR} operation poses problem in peeling; can't remove a decoded variable node from connected bins
			\item If a bin is 2-sparse i.e. $x_{ij}, x_{ik}=1$ and an index $j$ is known: refer to as \textcolor{red}{resolvable double-ton}
		        \begin{equation*}
				    \mbf{y}_i=\begin{bmatrix}
				    \mbf{b}_j\\
				    \overline{\mbf{b}}_j
				    \end{bmatrix} \lor \begin{bmatrix}
				    \mbf{b}_k\\
				    \overline{\mbf{b}}_k
				    \end{bmatrix}
		        \end{equation*}
		      \item bins with more than 2 non-zero variables (\textcolor{red}{multi-ton}) are unusable due to \textcolor{red}{\textit{OR}}
		      \item For $M_1>c(\epsilon)K$ bins, just using singletons and double-tons, \textcolor{blue}{peeling like iterative} decoder recovers ($1-\epsilon$) fraction of non-zero nodes w.h.p asymptotically
    	\end{itemize}
    \end{block}      
%\vspace{\blockskip}    

      \begin{block}{\Large Numerical Results}
		   \begin{figure}
     		  \centering
   			\scalebox{1.2}{% This file was created by matlab2tikz v0.4.7 running on MATLAB 7.14.
% Copyright (c) 2008--2014, Nico Schlömer <nico.schloemer@gmail.com>
% All rights reserved.
% Minimal pgfplots version: 1.3
% 
% The latest updates can be retrieved from
%   http://www.mathworks.com/matlabcentral/fileexchange/22022-matlab2tikz
% where you can also make suggestions and rate matlab2tikz.
% 
\begin{tikzpicture}
\def\fsize{\normalsize}
\pgfplotsset{every y tick label/.append style={font=\footnotesize}}
\pgfplotsset{every y tick label/.append style={font=\small}}

\begin{axis}[%
width=0.85\columnwidth,
height=0.8\columnwidth,
scale only axis,
xmin=20,
xmax=700,
xtick = {100,200,...,700},
xmajorgrids,
ymode=log,
ymin=1e-06,
ymax=1,
yminorticks=true,
ymajorgrids,
yminorgrids,
xlabel={\fsize{Number of tests per defective item ($M/K$)}},
ylabel={\fsize{Fraction of unidentified defectives}},
legend style={at={(0,0)},anchor=south west,draw=black,fill=white,legend cell align=left,font=\fsize}
]
\addplot [color=blue,dashed, mark=square,mark options={solid},line width=1pt]
  table[row sep=crcr]{96	0.8\\
144	0.4\\
192	0.16\\
240	0.055\\
288	0.025\\
336	0.013\\
384	0.006\\
432	0.0035\\
480	0.0023\\
528	0.0014\\
576	0.0009\\
624	0.00055\\
672	0.0003\\
};
\addlegendentry{$\ell=3$};

\addplot [color=blue,dashed,line width=1pt,mark=triangle,mark options={solid}]
  table[row sep=crcr]{96	0.94\\
144	0.68\\
192	0.32\\
240	0.11\\
288	0.036\\
336	0.015\\
384	0.005\\
432	0.0024\\
480	0.0011\\
528	0.00055\\
576	0.0004\\
624	0.00025\\
672	0.0001\\
};
\addlegendentry{$\ell=5$};


\addplot [color=blue,dashed,mark=o,mark options={solid},line width=1pt]
  table[row sep=crcr]{96	0.98\\
144	0.88\\
192	0.62\\
240	0.25\\
288	0.082\\
336	0.029\\
384	0.011\\
432	0.0045\\
480	0.0017\\
528	0.00065\\
576	0.00032\\
624	0.00017\\
672	6e-05\\
};
\addlegendentry{$\ell=7$};

%\addplot [color=blue,dashed,line width=3pt,mark=square,mark options={solid}]
%  table[row sep=crcr]{96	0.98\\
%144	0.94\\
%192	0.85\\
%240	0.52\\
%288	0.21\\
%336	0.065\\
%384	0.024\\
%432	0.0095\\
%480	0.0032\\
%528	0.0014\\
%576	0.00055\\
%624	0.00024\\
%672	0.0002\\
%};
%\addlegendentry{$l=9$};

\addplot [color=red,solid, line width=1pt, mark=square,mark options={solid}]
  table[row sep=crcr]{
36  0.99261\\
50.40      8.22e-1 \\
99.00      3.43e-1\\
150.00    6.01e-2 \\
204.00    1.61e-2\\
251.10    3.84e-3 \\
299.70    1.72e-3  \\
 351.00    8.28e-4 \\ 
 405.00    4.67e-4\\
 450.24    2.08e-4\\
 501.60  1.26e-4 \\  
 576.00  1.09e-4\\
672.00   5.50e-5\\
%66  0.80679\\
%79.20  0.65094\\
%99  0.4133 \\
%120     0.1596\\
%150     0.06312\\
%180 0.02797\\ 
%210     0.01396\\
%253.8  0.00344 \\
%307.8  0.00153\\
%324     0.00124 \\
%361.8  0.00078 \\
%369.6  0.00046 \\
%408     0.00031\\
%480.0  0.00021 \\
%672 0.000076\\
};
\addlegendentry{$\ell=3$, regular};

\addplot [color=red, solid, line width=1pt, mark=triangle,mark options={solid}]
  table[row sep=crcr]{
   39.000    0.970769\\
    50.700      9.16e-1   \\
100.800     7.33e-1   \\
148.500     1.93e-1   \\
201.000     2.07e-2  \\
249.000     5.22e-3  \\
300.000     1.450e-3\\
351.000     2.51e-4\\
%405.000     8.60e-5\\
450.900     4.90e-5\\
500.580     1.90e-5 \\
550.800     1.20e-5\\
%602.100 	  1.10e-5\\
675.000     6.00e-6\\
%   72.000    0.946226\\
% 132.000    0.327190\\
% 198.000    0.036304 \\
% 210.000    0.015015 \\
% 240.000    0.006573\\
% 270.000    0.003021\\
% 300.000    0.001460\\
% 360.000    0.000479\\
% 378.000    0.000157\\
% 453.600    0.000048\\
% 540.000    0.000013\\
% 604.800    0.000003\\
 };
\addlegendentry{$\ell=5$, regular};

\addplot [color=red, solid, line width=1.0pt,mark=o]
  table[row sep=crcr]{
42.00  0.99      \\
58.50	0.9995  \\
78.00    0.9937 \\
108.00   0.9142 \\
151.20  0.5487  \\
198.00 8.92e-2  \\
282.00  5.0e-3    \\
360.00  5.4e-4   \\
462.00  6.42e-5   \\
499.50  9.00e-6   \\
540.00  4.00e-6   \\
594.00  1.25e-6   \\
};
\addlegendentry{$\ell=7$,regular};

\end{axis}
\end{tikzpicture}%}   
\caption{Monte Carlo simulations for $K=100, N=2^{16}$. For a given $\ell$ the bin detection size is fixed and we vary the number of bins. The plots in blue indicate the scheme in \cite{lee2015saffron} and the plots in red indicate our scheme based on left-and-right-regular bipartite graphs for various left degrees $\ell\in\{3,5,7\}$.}
			\end{figure}    
      \end{block}  
      
    \setbeamercolor{block alerted title}{fg=black,bg=norange} % Change the alert block title colors
    \setbeamercolor{block alerted body}{fg=black,bg=white} % Change the alert block body colors

         \begin{alertblock}{\Large Conclusions} 
	    \textbf{Compressed Sensing:} We propose a scheme that has
			\begin{itemize}
			\itemsep10pt
				\item \textcolor{red}{order optimal} sample complexity of \textcolor{blue}{$O(K\log (\frac{N}{K}))$}
				\item \textcolor{red}{sub-linear} optimal decoding complexity: \textcolor{blue}{$O(K\log (\frac{N}{K}))$}
			\end{itemize} 
\vspace{\paraskip}    
  
   \textbf{Group testing:} We propose a scheme that achieves
		\begin{itemize}
		\itemsep10pt
			\item \textcolor{red}{order optimal} testing complexity: \textcolor{blue}{$O(K\log (\frac{N}{K}))$}
			\item \textcolor{red}{sub-linear} optimal decoding complexity: \textcolor{blue}{$O(K\log (\frac{N}{K}))$}
		\end{itemize} 
    \end{alertblock}
\vspace{-\paraskip}    

    % Bibliography
    \vspace{2.5cm}
    \begin{block}{References}
      \begin{thebibliography}{10}
%	\bibitem{Wainwright}
%M. Wainwright, ``Information-Theoretic Limits on Sparsity Recovery in the High-Dimensional and Noisy Setting.'' \emph{IEEE Trans. \ Inform \ Theory}, vol.~55, no.~12, pp.~5728-5741, 2009.

\bibitem{lee2015saffron}
K. Lee, R. Pedarsani, and K. Ramchandran, ``Saffron: A fast, efficient, and robust framework for group testing based on sparse-graph codes'', arXiv preprint, arXiv:1508.04485, 2015.

\bibitem{li2015subisit}
X. Li, S. Pawar, and K. Ramchandran, ``Sub-linear time compressed sensing using sparse-graph codes'', in \emph{Proc. Int. Symp. Inform. Theory},
  pp.~1645-1649, 2015.
  
\bibitem{vemsublinear}
A. Vem, N. Janakiraman, and K. Narayanan,``Sub-linear time compressed sensing for support recovery using left and right regular sparse-graph codes'', in \emph{Proc. Inform. Theory. Workshop}, pp.~429-433, 2016.
 \end{thebibliography}
	  \end{block}

  \end{column}
  
\end{columns}
\end{frame}
\end{document}

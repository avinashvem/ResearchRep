\documentclass[conference,onecolumn]{IEEEtran}
\usepackage{graphicx,psfrag,epsfig,epsf,latexsym,hhline,amsmath,amssymb,amsthm}
\usepackage{url}
\interdisplaylinepenalty=2500
\oddsidemargin =0.0in
\evensidemargin=0.0in
\topmargin=-0.1in
\headsep=0.0in
\textwidth=6.5in
\textheight=9.0in
\begin{document}
%\pgfdeclarelayer{background}
%\pgfdeclarelayer{foreground}
%\pgfsetlayers{background,main,foreground}
%\overrideIEEEmargins
%\CLASSINPUTinnersidemargin
%\CLASSINPUToutersidemargin
\IEEEoverridecommandlockouts
\newtheorem{theorem}{Theorem}
\newtheorem{lemma}[theorem]{Lemma}
\newtheorem{remark}[theorem]{Remark}
\newtheorem{definition}[theorem]{Definition}
\newtheorem{corollary}[theorem]{Corollary}

%\renewcommand{\epsilon}{\varepsilon}
\newcommand{\mbf}[1]{\mathbf{#1}}
\newcommand{\mc}[1]{\mathcal{#1}}
\newcommand{\Prob}{\text{Pr}}

\title{\LARGE{Improved Bounds for Sub-linear Time Compressed Sensing for Support Recovery using Sparse-Graph Codes}}
\author{\IEEEauthorblockN{Avinash Vem, Nagaraj Thenkarai Janakiraman, Krishna Narayanan \\}
\IEEEauthorblockA{Dept. of Elect. \& Comp. Engg., \\
Texas A\&M University, \\
College Station, TX, U.S.A\\}
%\and
%\IEEEauthorblockN{Nagaraj Thenkarai Janakiraman\\}
%\IEEEauthorblockA{Dept. of Elect. \& Comp. Engg., \\
%Texas A\&M University, \\
%College Station, TX, U.S.A\\}
%\and
%\IEEEauthorblockN{Krishna R. Narayanan\\}
%\IEEEauthorblockA{Dept. of Elect. \& Comp. Engg., \\
%Texas A\&M University, \\
%College Station, TX, U.S.A}
}
\date{Feb 20, 2016}
\maketitle

\pagestyle{empty}

\section{Introduction}
In \cite{li2015subisit} (and in the expanded version in \cite{li2015subdraft}), Li, Pawar and Ramchandran have considered the problem of recovering the support of a $K$-sparse, $N$-dimensional signal from $M$ linear measurements in the presence of noise. Based on sparse-graph codes and a peeling decoder, they have proposed an elegant design of the measurement matrix and a recovery algorithm that is nearly optimal in terms of measurement complexity and computational complexity. In particular, they have proposed two designs - the first design requires $M = O(K \log N)$ measurements and a near-linear ($O(N \log N)$) decoding complexity, whereas the second design requires $M = O(K \log^{1.3} N)$ measurements with a sub-linear decoding complexity.

Specifically, the following theorems are given by Li, Pawar and Ramchandran in \cite{li2015subdraft}
\begin{theorem}[\cite{li2015subdraft} Noiseless recovery]\label{thm:li1}
In the absence of noise, given any $K$-sparse signal $x$ with $x[k] \in \mathcal{X}$ for $k \in supp (x)$, their noiseless recovery schemes achieve a vanishing failure probability $\mathbb{P}_F(O\left(\frac{1}{K})\right) \rightarrow 0$ asymptotically in $K$ and $N$ with
\begin{center}
\begin{tabular}{|c|c|c|}
  \hline
   & Measurement cost $M$ &  Computational Complexity $T$ \\
  \hline
  Fourier noisy & $2(1+\epsilon)K$ & $O(K)$ \\
  \hline
  Binary noisy & $(1+\epsilon)K(\log_2 N+1)$ & $O(K \log N)$ \\
  \hline
\end{tabular}
\end{center}
\end{theorem}
\vspace{2ex}


\begin{theorem}[\cite{li2015subdraft} Sub-linear Time Noisy Recovery]\label{thm:li2}. In the presence of i.i.d. Gaussian noise with zero mean and
variance $\sigma^2$, given any $K$-sparse signal $x$ with $x[k] \in \mathcal{X}$ for $k \in supp (x)$, our noiseless recovery schemes achieve a vanishing failure probability $\mathbb{P}_F \rightarrow 0$ asymptotically in $K$ and $N$ with
\begin{center}
\begin{tabular}{|c|c|c|}
  \hline
   & Measurement cost $M$ &  Computational Complexity $T$ \\
  \hline
  Fourier noisy & $O(K \log^{1.3}N)$ & $O(K \log^{1.3}N)$ \\
  \hline
  Binary noisy & $O(K \log N)$ & $O(K \log N)$ \\
  \hline
\end{tabular}
\end{center}
\end{theorem}
\vspace{2ex}

\begin{theorem}[\cite{li2015subdraft} Near-linear Time Noisy Recovery]\label{thm:li3}. In the presence of i.i.d. Gaussian noise with zero mean and
variance $\sigma^2$, given any $K$-sparse signal $x$ with $x[k] \in \mathcal{X}$ for $k \in supp (x)$, the RandomNoisy scheme achieves a vanishing failure probability $\mathbb{P}_F \rightarrow 0$ asymptotically in $K$ and $N$ with a measurement complexity of $M = O(K \log N)$ and computational complexity of $T = O(N \log N)$.
\end{theorem}

\section{Main Results}
We believe that Theorem~\ref{thm:li2} and Theorem~\ref{thm:li3} can be sharpened to the following new theorems
\begin{theorem}[Sub-linear Time Noisy Recovery]\label{thm:our1}. In the presence of i.i.d. Gaussian noise with zero mean and
variance $\sigma^2$, given any $K$-sparse signal $x$ with $x[k] \in \mathcal{X}$ for $k \in supp (x)$, our noisy recovery schemes achieve a vanishing failure probability $\mathbb{P}_F \rightarrow 0$ asymptotically in $K$ and $N$ with
\begin{center}
\begin{tabular}{|c|c|c|}
  \hline
   & Measurement cost $M$ &  Computational Complexity $T$ \\
  \hline
  Fourier noisy & $O\left(K \log^{1.3} \frac{N}{K} \right)$ & $O\left(K \log^{1.3} \frac{N}{K} \right)$ \\
  \hline
  Binary noisy & $O\left(K \log \frac{N}{K} \right)$  & $O\left(K \log \frac{N}{K} \right)$ \\
  \hline
\end{tabular}
\end{center}
\end{theorem}

\begin{theorem} [Near-linear Time Noisy Recovery]\label{thm:our2}. In the presence of i.i.d. Gaussian noise with zero mean and
variance $\sigma^2$, given any $K$-sparse signal $x$ with $x[k] \in \mathcal{X}$ for $k \in supp (x)$, the RandomNoisy scheme achieves a vanishing failure probability $\mathbb{P}_F \rightarrow 0$ asymptotically in $K$ and $N$ with a measurement complexity of $M = O \left( K \log \frac{N}{K} \right)$ and computational complexity of $T = O \left( N \log \frac{N}{K} \right)$.
\end{theorem}

\subsection{Main Idea}
The main difference between \cite{li2015subdraft} and our approach is that we replace the left $d$-regular ensemble of graphs $\mathcal{G}^N_{\rm reg}(R,d)$ in \cite{li2015subdraft} by $\left(d, \frac{dN}{R} \right)$ left {\em and} right regular ensemble with fixed degree. This allows us to use a bin detection matrix with $P' = O\left(\log_2 \left( \frac{N}{K} \right)\right)$ rows
instead of using a bin detection matrix with $P = O(\log_2 N)$ rows.

\begin{definition}[Left and right regular graph ensemble]\label{def:leftandrighreg}
Let $\mathcal{G}^N_{\rm reg,reg}\left(R,d, \frac{dN}{R} \right)$ denote the ensemble of graphs with $N$ left nodes and $R$ right nodes, where each left node $k \in [N]$ is connected to $d$ right nodes and each right node $j \in [R]$ is connected to $\frac{dN}{R}$ left nodes.
\end{definition}

The number of observations $R$ will be chosen to be $R = \eta K = O(K)$. A matrix $\mathbf{H}$ is chosen at random from this ensemble and used as the coding matrix. A bin detection matrix $\mathbf{S}$ is chosen similar to that in \cite{li2015subdraft} however now the bin detection matrix has only $P' = \log_2 \left(\frac{N}{R}\right)$ rows. Since $R = O(K)$, this implies that $P' = O(N/K)$. The probability of error for the bin detection algorithm can be analyzed exactly as in \cite{li2015subdraft} with the new $P'$ and shown to have exactly the same behavior in $K$. However, there will be a penalty to pay in terms of the minimum SNR required and this is discussed next.

\subsection{Minimum SNR required}
We will take the RandomNoisy case as an example and let us consider the case when $K = O(N^\delta)$ for some $0 < \delta < 1$.  In the derivation of the probability of error in \cite{li2015subdraft} in Proposition 3 and Proposition 4 in Appendix D, it can be seen that the probability of error includes terms of the form
\[
e^{-\frac{P}{4} \frac{\gamma^2}{1+4\gamma}} + 2 e^{-c_6 P \left(1-\frac{\gamma \sigma^2}{A^2_{\rm min}} \right)}
\]

Right after Proposition~5, it is mentioned that by choosing $P = O(\log N)$, the probability of error can be made to decay as $O\left(\frac{1}{N^3} \right) <  O\left(\frac{1}{K^3} \right)$. First, it should be noted that when $P = O(\log N)$, in order for the probability of error to decay as $O\left(\frac{1}{K^3} \right)$, the following two conditions must be satisfied on $\gamma$
\[
\frac{\gamma^2}{4(1+4\gamma)} > 2 \delta, \ \ {\rm and} \ c_6 \left(1 - \frac{\gamma \sigma^2}{A_{\rm min}^2} \right) > 2 \delta
\]

Hence, for the overall probability of error to decay as $O(1/K^3)$, $A_{\rm min}^2/\sigma^2$ should be large enough such that $0 < 2 \delta < \gamma < A_{\rm min}^2/\sigma^2$. Hence, the theorems appear to be valid only for $A_{\rm min}^2/\sigma^2 > SNR_{\rm Th}$.

\subsection{SNR versus Measurements}
It should be noted that if the objective is to get the probability of error to decay only as $O(1/K^3)$, it is not required for the probability of error to decay as $O(1/N^3)$ and hence, we could directly choose $P = O(N/K) = N^{1-\delta}$ or in fact, $P = O(N^\beta)$ for any $\beta$ and that there will be a corresponding $SNR_{\rm Th}$. Indeed, this appears to reasonable since the assumption that the non-zero values of the signals are from a finite set and hence, when the $\sigma^2 \rightarrow 0$, the number of samples required should smoothly go to $O(K)$. Hence, it appears that the order of measurements required should be a function of SNR.

Based on this, indeed, it appears that when using RandomNoisy measurement matrix, it is not even required to switch to the $\mathcal{G}_{\rm reg-reg}$ ensemble and we could have simply chosen $P = O(N^\beta)$. However, if we want to choose the Random DFT ensembele (see Page 22 \cite{li2015subdraft}), then we have to switch to the $\mathcal{G}_{\rm reg-reg}$. It appears convenient to switch to this ensemble for all results

\section{Proofs}
We have to prove that we can use the $\mathcal{G}^N_{\rm reg,reg}\left(R,d, \frac{dN}{R} \right)$ for optimal peeling. This indeed appears to be straightforward.
%Still it may be useful to make the equivalence between the erasure decoding problem and syndrome source coding problem and show that this ensemble will provide a probability of failure of at most $O\left(\frac{1}{K}\right)$.
In the next section we consider a $\mathcal{G}(R,d,\frac{dN}{R})$ ensemble and show that this ensemble will provide a probability of failure of at most $O\left(\frac{1}{K}\right)$. Although it appears this can be achieved in a straight forward manner by using a capacity achieving spatially-coupled LDPC ensemble and use the existing results, there are two main obstacles to this:
\begin{itemize}
\item  In traditional LDPC codes and peeling decoder over binary erasure channel, the input to the decoder is the channel output corresponding to $N$ bit nodes and the check nodes on the right are mere parity checks whose sum modulo 2 is zero. Whereas in this problem the values corresponding to the $N$ variable nodes on the left need to be evaluated by the decoder given the values corresponding to the $R$ check nodes (the real sum of the variable nodes connected) on the right node are non-zero  and are given as input to the decoder.\\
\item  In traditional LDPC case a constant fraction $\epsilon$ of these $N$ variable nodes are erased by the channel and usually the emphasis is on analyzing the performance of peeling decoder asymptotically in $N$ or $R$ when rate=$1-\frac{R}{N}$ is fixed. But in our case the fraction of the nodes erased $=1-\frac{K}{N}$, where $K$ is usually of the form $K=N^{\delta}$, tend to one and the rate of the code$=1-\frac{R}{N}=1-\frac{\eta N^{\delta}}{N}$ tend to one asymptotically in $N$.
\end{itemize}
\vspace{1ex}

Consider a left and right regular LDPC code $\mathcal{G}_{\text{LDPC}}(N,l,r)$ where $N$ is the number of variable nodes on the left and $l,r$ are the regular left and right degrees respectively. Let $\text{P}_{\text{BEC}}^{(i)}(\mathbf{y})$ be the degree distribution of the number of check nodes after iteration $i$ of peeling decoder given $\mathbf{y}$ is the channel output. And similarly $\mathcal{G}_{\text{reg-reg}}(N,l,r)$ be the graph corresponding to the parity check matrix in compressed sensing problem and $\text{P}_{\text{CS}}^{(i)}(\mathbf{z})$ be the degree distribution of the check nodes on the right after iteration $i$ of the oracle-based peeling decoder, given $\mathbf{z}$ is the CS equivalent of syndrome corresponding to $\mathbf{x}$ i.e., $\mbf{z}=\mbf{H}\mbf{x}$ where the operations are over the real field. Note that $\mbf{y}$ is a vector of dimension $N$ whereas $\mbf{z}$ is of dimension $\frac{Nl}{r}$.\\

Note that in the peeling decoder we consider, we peel off one degree-1 check node and the variable node connected to it from the graph in each iteration. In the case of LDPC-BEC we remove all the variable nodes that are not erased by the channel and the resulting graph is input to the decoder. Similarly in the case of compressed sensing we consider the  oracle based peeling decoder in \cite{li2015subdraft} and we analyze the \textit{pruned}-graph where we remove all the zero nodes from the original graph.

\begin{lemma}[Equivalence to LDPC-BEC]
\label{Lemma:Equiv_LDPC_BEC}
Whenever $\mbf{y}$ and $\mbf{z}$ satisfy
\begin{align*}
\mbf{z=Hx} \text{ such that  } S:=|\text{supp}(\mbf{x})|=|\{i:y_i=\mathcal{E}\}| \qquad \text{where } \mathcal{E} \text{ denotes erasure},
\end{align*}
then $\text{P}_{\text{BEC}}^{(i)}(\mbf{y})=\text{P}_{\text{CS}}^{(i)}(\mbf{z}) \quad \forall i$.
\end{lemma}
\begin{IEEEproof}
Define $S^c=[1:N]\backslash S$. In the case of LDPC codes on BEC we peel off all non-erased variable nodes corresponding to $S^c$  and input the resulting graph to the peeling decoder. Similarly in the case of bipartite graph in CS problem we peel off all the zero nodes corresponding to $S^c$ and we input the resulting graph to oracle based peeling decoder. From this point onward the peeling decoders are identical and thus we have our result.
\end{IEEEproof}
\vspace{1ex}
%The above lemma gives us the equivalence of peeling decoder of LDPC codes on BEC and oracle based peeling decoder of compressed sensing problem. 
Thus by considering a BEC of erasure probability $\epsilon=\frac{K}{N}$ we can equivalently consider peeling decoder of LDPC codes on BEC channel and use various existing results.
\begin{lemma}
\label{lem:RightDegEvolution}
The evolution of the left and right degree distribution as the peeling decoder progresses can be given by
\begin{align*}
\tilde{L}_d(y)&= y^d,\\
\tilde{R}_{1}(y)&=r\epsilon y^{l-1}[y-1+ (1-\epsilon y^{l-1})^{r-1}]\\
\tilde{R}_{i}(y)&=\binom{r}{i}(\epsilon y^{l-1})^i (1-\epsilon y^{l-1})^{r-1}, \quad i\geq 2
%\tilde{R}_{0}(y)&=1-\sum_{j\geq 1}\tilde{R}_{j}(y),\\
\end{align*}
where $\epsilon=\frac{K}{N}$ and $r=\frac{Nd}{\eta K}$. Note that the curve corresponding to $\tilde{L}_i(y)(\tilde{R}_{i}(y))$ for $y\in[0,1]$ gives the expected number of degree $i$ variable nodes (check nodes) normalized with respect to $K$ ($\eta K)$.
\end{lemma}
\begin{IEEEproof}
As we showed in Lemma \ref{Lemma:Equiv_LDPC_BEC} the peeling decoder for an LDPC on BEC channel and oracle based peeling decoder for CS are identical upto the residual degree distributions at each iteration. Hence we can use the result for LDPC codes \cite{richardson2008modern} with equivalent channel erasure probability $\epsilon=\frac{K}{N}$.
\end{IEEEproof}

\begin{definition}[BP Threshold]
We define the BP threshold, $\eta^{\text{BP}}$ to be the minimum value of $\eta$ for which there is no non-zero solution for the equation:
\begin{align*}
y&=\lim_{N\rightarrow\infty}1-\left(1-\frac{y^{l-1}}{N^{1-\delta}}\right)^{\frac{dN^{1-\delta}}{\eta}}\\
  &=1-e^{\frac{-ly^{l-1}}{\eta}}
\end{align*}
in the range $y\in [0,1]$.
\end{definition}


\begin{lemma}\cite{richardson2008modern}
\label{lem:PeelSmallGraph}
If $\eta>\eta^{BP}$ then with probability at least $1-O\left(K^{1/6}e^{-\frac{\sqrt{Kl}}{(lr)^3}}\right)$ the peeling decoder of a specific instance progresses until the number of residual variable nodes in the graph has reached size $\gamma K$ where $\gamma$ is an arbitrary positive constant.
\end{lemma}

\begin{definition}[Expander Graphs]
\label{Def:ExpanderGraph}
A bipartite graph with K left nodes and regular left degree $l$ is called a $(\gamma,1/2)-$ expander if for all subsets $S$ of left nodes with $|S|\leq \gamma K$, the right neighborhood of $S$ denoted by $\mathcal{N}(S)$ satisfies $|\mc{N}(S)|>l|S|/2$.
\end{definition}

\begin{lemma}\label{Lem:ExpGraph}
Consider a left and right regular ensemble $\mc{G}_{\text{reg-reg}}(N,l,\frac{Nl}{\eta K})$, then the pruned graph resulting from any given $K$-sparse signal $\mbf{x}$ is a $(\gamma,1/2)$-expander with probability at least $1-O\left(\frac{1}{K^{l-2}}\right)$ for a sufficiently small constant $\gamma >0$.
\end{lemma}
\begin{IEEEproof}
The proof is similar to the proof used in \cite{li2015subdraft} with minor modifications. Let $E_{v}$ denote the event that a subset $S_{v}$ of variable nodes on the left with size $v$ has at most $l|S_v|/2$ neighbors whose probability can be computed as
\begin{align}
\Prob(E_v)&\leq \binom{K}{v}\binom{\eta K}{lv/2}\left(\frac{vl}{2\eta K}\right)^{lv}\label{Eqn:ExpIneq1}\\
				&\leq c^{vl/2}\left(\frac{v}{K}\right)^{v(l/2-1)}\label{Eqn:ExpIneq2_Binom}
%				&\leq \left(\frac{vc}{K}\right)^{vl/2}\notag
\end{align}
where $c=\frac{le^{2}}{2\eta}$ is a constant. In \eqref{Eqn:ExpIneq1} we upper bound the probability of $E_v$ via union bound over all possible size $v$ subsets on the left and size $lv/2$ subsets on the right. In \eqref{Eqn:ExpIneq2_Binom} we use the inequality $\binom{a}{b}\leq \left(ae/b\right)^b$ and we assume $l\geq 2$ to simplify the constant factor. Then we union bound over all subsets of size upto the remaining nodes $\gamma^{*} K$ where we choose $\gamma^{*}=\left(4c^l\right)^{\frac{-1}{l-2}}$
\begin{align*}
\sum_{v=2}^{\gamma^{*} K}\Prob(E_{v})&\leq \sum_{v=2}^{\gamma^{*} K}\left(c^{l}\left(\frac{v}{K}\right)^{l-2}\right)^{v/2}\\
														&=O\left(\frac{1}{K^{l-2}}\right)
\end{align*}
Thus we showed that asymptotically in $K$, the left and right regular graphs are good expander graphs with probability atleast $1-O(1/K^{l-2}).$
\end{IEEEproof}
\vspace{1ex}


\begin{theorem}
Consider the ensemble $\mathcal{G}_{\text{reg-reg}}(N,l,\frac{Nl}{\eta K})$, the oracle based peeling decoder peels off all the edges in the pruned graph in $\eta K$ iterations with probability at least $1-O\left(1/K^{l-2}\right)$.
\end{theorem}
\begin{IEEEproof}
%Here is the brief outline of the proof. We first show that expressions for the evolution of degree distributions from LDPC-BEC are valid here as shown in Lemma. \ref{lem:RightDegEvolution}. Then
 Lemma \ref{lem:PeelSmallGraph} shows us that the peeling decoder fails to peel off till the residual graph has $\gamma N$ variable nodes remaining with an exponentially low probability. Then in Lemma \ref{Lem:ExpGraph} tells us that the left regular graphs are not good expanders with a probability of atmost $O(1/K^{l-2})$ and hence the remaining $\gamma N$ nodes can be peeled off with high probability. Thus the overall probability of failure will be dominated by small stopping sets which can be upper bounded by $O(1/K^{l-2})$.
\end{IEEEproof}
\vspace{3ex}


\section{Extensions} % For the group testing problem addressed in \cite{lee2015saffron} also, it appears we can replace the ensemble used by Lee, Pedarsani and Ramchandran by the regular-regular ensemble and obtain a sharper result, i.e., $O(K \log (N/K))$ measurements for the case of $K = N^\delta$, where $\delta < 2/3$.
%
%When the right regular ensemble is considered, then the probability of error at each measurement node will be $O\left(\frac{K^2}{N^2}\right)$ and there are $K$ stages in the peeling process. Thus, for the overall error probability to decay with $K$, we require that
%\[
%\frac{K^3}{N^2} < 1 \Rightarrow \frac{N^{3\delta}}{N^2} < 1 \Rightarrow \delta < \frac{2}{3}.
%\]
%
%The main idea in the proof is to look at the reduced subgraph induced only by the defective items and when $N/K \rightarrow \infty$, this graph will be a graph from the left regular, right Poisson ensemble considered in \cite{lee2015saffron}.

For the group testing problem addressed in \cite{lee2015saffron} also, it appears we can replace the ensemble used by Lee, Pedarsani and Ramchandran by the regular-regular ensemble and obtain a sharper result, i.e., $O(K \log (N/K))$ measurements for the case of $K = N^\delta$.

We use a left and right regular ensemble for the LDPC code and we use a measurement matrix with $2 \beta \log_2(N/K)$ rows and $N/K$ columns. When the right regular ensemble is considered, then the probability of error at each measurement node will be $O\left(\frac{K^{\beta-1}}{N^{\beta-1}}\right)$ and there are $K$ stages in the peeling process. Thus, for the overall error probability to decay with $K$ according to $O\left(\frac{K^{\beta}}{N^{\beta-1}}\right) = O\left(N^{\delta \beta - \beta + 1}\right)$. If we want this probability of error to be $O(N^{-\delta})$, then we can set $\beta = \frac{1+\delta}{1-\delta}$ to satisfy this. Then, the total number of measurements will be $2 (1+\delta) K \log_2 N$. This is strictly an improvement over $6 K \log_2 N$ measurements used in \cite{lee2015saffron}.

The main idea in the proof is to look at the reduced subgraph induced only by the defective items and when $N/K \rightarrow \infty$, this graph will be a graph from the left regular, right Poisson ensemble considered in \cite{lee2015saffron}.
\bibliographystyle{IEEEtran}
\bibliography{IEEEabrv,sparseestimation}
\end{document}

\documentclass[10pt]{beamer}
\mode<presentation>
{
\usetheme{CambridgeUS}
\usecolortheme{dolphin}
}

\setbeamertemplate{footline}[frame number]
\definecolor{purple}{RGB}{255,0,204}

\usecolortheme[RGB={0,100,0}]{structure}

\AtBeginSection[]
{
  \begin{frame}<beamer>{Outline}
    \tableofcontents[currentsection,currentsubsection]
  \end{frame}
}



\usepackage{amsmath,amssymb,amsthm}
\usepackage{enumerate}
\usepackage{stfloats}
\usepackage{comment}

\usepackage{graphics} % for pdf, bitmapped graphics files
\usepackage{epsfig} % for postscript graphics files
\usepackage{mathptmx} % assumes new font selection scheme installed
\usepackage[mathscr]{euscript}
\usepackage{algorithm}
\usepackage[noend]{algpseudocode}

\makeatletter
\def\BState{\State\hskip-\ALG@thistlm}
\makeatother

\usepackage{tikz}
\usetikzlibrary{arrows,shapes,chains,matrix,positioning,scopes,patterns,calc}
\usepackage{pgfplots}
\usepgflibrary{shapes}

\newtheorem{remark}[theorem]{Remark}
\renewcommand{\epsilon}{\varepsilon}

\newcommand{\h}{\texttt{h}}
\newcommand{\hbp}{\h^{\mathrm{BP}}}
\newcommand{\hmap}{\h^{\mathrm{MAP}}}
\newcommand{\hstab}{\h^{\mathrm{stab}}}
\newcommand{\harea}{\h^{A}}
\newcommand\indep{\protect\mathpalette{\protect\independenT}{\perp}}
\def\independenT#1#2{\mathrel{\rlap{$#1#2$}\mkern2mu{#1#2}}}

\newcommand{\expt}{\mathbb{E}}
\newcommand{\indicator}[1]{\mathbbm{1}_{\left\{ {#1} \right\} }}
\newcommand{\abs}[1]{\left\lvert#1\right\rvert}

\newcommand{\mb}[1]{\mathbf{#1}}
\newcommand{\mbb}[1]{\mathbb{#1}}
\newcommand{\mr}[1]{\mathrm{#1}}
\newcommand{\mc}[1]{\mathcal{#1}}
\newcommand{\ms}[1]{\mathsf{#1}}
\newcommand{\msc}[1]{\mathscr{#1}}
\newcommand{\mf}[1]{\mathfrak{#1}}

\newcommand{\RNum}[1]{\uppercase\expandafter{\romannumeral #1\relax}}

\newcommand{\mse}{\mathsf{e}}
\newcommand{\msx}{\mathsf{x}}
\newcommand{\msxvn}{\tilde{\mathsf{x}}}
\newcommand{\msy}{\mathsf{y}}
\newcommand{\msz}{\mathsf{z}}
\newcommand{\msa}{\mathsf{a}}
\newcommand{\msb}{\mathsf{b}}
\newcommand{\msbx}{\underline{\mathsf{x}}}
\newcommand{\msby}{\underline{\mathsf{y}}}
\newcommand{\msbxvn}{\tilde{\underline{\mathsf{x}}}}
\newcommand{\msbz}{\underline{\mathsf{z}}}
\newcommand{\msba}{\underline{\mathsf{a}}}
\newcommand{\msbb}{\underline{\mathsf{b}}}
\newcommand{\msbc}{\underline{\mathsf{c}}}

%\newcommand{\bv}{\underline{\mathrm{b}}}
%\newcommand{\xv}{\underline{\mathrm{x}}}
%\newcommand{\yv}{\underline{\mathrm{y}}}
%\newcommand{\zv}{\underline{\mathrm{z}}}
%\newcommand{\rv}{\underline{\mathrm{r}}}
%\newcommand{\wv}{\underline{\mathrm{w}}}

%\newcommand{\Xv}{\underline{\mathrm{X}}}
%\newcommand{\Yv}{\underline{\mathrm{Y}}}
%\newcommand{\Zv}{\underline{\mathrm{Z}}}
%\newcommand{\Rv}{\underline{\mathrm{R}}}
%\newcommand{\RXYv}{\underline{\mathrm{R}_{XY}}}

\newcommand{\bv}{\vec{\mathrm{b}}}
\newcommand{\xv}{\vec{\mathrm{x}}}
\newcommand{\yv}{\vec{\mathrm{y}}}
\newcommand{\zv}{\vec{\mathrm{z}}}
\newcommand{\rv}{\vec{\mathrm{r}}}
\newcommand{\wv}{\vec{\mathrm{w}}}


\newcommand{\Xv}{\vec{X}}
\newcommand{\Yv}{\vec{Y}}
\newcommand{\Zv}{\vec{Z}}
\newcommand{\Rv}{\vec{R}}
\newcommand{\RXYv}{\vec{R}_{XY}}

\DeclareMathAlphabet{\mcl}{OMS}{cmsy}{m}{n}

\newcommand{\wh}{\widehat}
\newcommand{\bop}{\ast}
\newcommand{\vnop}{\varoast}
\newcommand{\disth}{d_{\mathrm{H}}}
\newcommand{\cnop}{\boxast}
\newcommand{\diff}[1]{d#1}
\newcommand{\deri}[1]{\mathrm{d}_{ #1 }\hspace{0.05cm}}
\newcommand{\dderi}[1]{\mathrm{d}_{ #1 }^2\hspace{0.05cm}}
\newcommand{\bvert}[1]{\,\Big{\vert}_{ #1  }}
\newcommand{\degr}{\succ}
\newcommand{\degreq}{\succeq}
\newcommand{\upgr}{\prec}
\newcommand{\upgreq}{\preceq}
\newcommand{\extR}{\overline{\mathbb{R}}}

\newcommand{\des}{\mathsf{T}_\mathrm{s}}
\newcommand{\dec}{\mathsf{T}_\mathrm{c}}
\newcommand{\pots}{U_\mathrm{s}}
\newcommand{\potc}{U_\mathrm{c}}
\newcommand{\shft}{\mathsf{S}}

\newcommand{\vnunit}{\Delta_0}
\newcommand{\cnunit}{\Delta_\infty}

\newcommand{\ent}[1]{ \mathrm{H} \left( #1 \right) }

\newcommand{\meass}{\mathcal{M}}
\newcommand{\probs}{\mathcal{X}}
\newcommand{\dpros}{\mathcal{X}_{\mathrm{d}}}
\newcommand{\chend}{N_{w}}

\newcommand{\minf}{\mathsf{a}_{0}}
\newcommand{\minfb}{\underline{\minf}}

\DeclareMathOperator*{\argmin}{\,arg\ min}
\DeclareMathOperator*{\argmax}{\,arg\ max}

\newlength\tikzwidth
\newlength\tikzheight

\textfloatsep=0.05in

\newcommand{\coleq}{\mathrel{\mathop:}=}
\newcommand{\defeq}{\triangleq}
%%\def\BayStationPosition{(0,0)}
%%
%\def\SensorOnePosition{(-1.2,   -0.3)}
%\def\SensorTwoPosition{(-1.5,   0.3)}
%\def\SensorThreePosition{(-1.3, 1.1)}
%\def\SensorFourPosition{(-0.3,  1.6)}
%\def\SensorFivePosition{(0.3,   2.2)}
%\def\SensorSixPosition{(0.7,    1.7)}
%\def\SensorSevenPosition{(1.7,  1.8)}
%\def\SensorEightPosition{(1.6,  0.8)}
%\def\SensorNinePosition{(2.1,   0.3)}
%\def\SensorTenPosition{(1.8,    -0.4)}
%\def\SensorElevenPosition{(1.0, -0.7)}
%\def\SensorTwelvePosition{(1.0, -1.4)}
%\def\SensorThirteenPosition{(0.0,   -2.2)}
%\def\SensorFourteenPosition{(-0.3,  -1.4)}
%\def\SensorFifteenPosition{(-1.3,   -1.9)}





\tikzset
{
    vnodeStyle/.style =
    {
        % -- shape properties --
        circle,                                 % shape
%       rounded corners = 3mm,                  % kind of corner (and radius of the roundness)
%       minimum height  = 0.15\textwidth,       % | minimum size of the node
%       minimum width   = 0.9\textwidth,        % |
        minimum size    = 10pt,                %
        rotate          = 0,                    % angle of rotation
        scale           = 1.0,                  % scaling factor
        thick,                                  % thickness of the border
        %
        % -- colours properties --
        % filling: [ trasparent | monocolored | shaded]; decomment what you prefer
%       %                                       % transparent (all commented)
        fill            = white,             % monocolored
%       top color       = white,                % | filling of the node
%       bottom color    = red!50!black!20,      % |
        text            = black,                % colour of the fonts
        draw            = black,                % colour of the border
        %
        % -- fonts --
        font            = \scriptsize,              % shape of the font (or dimension, like \tiny)
%       text centered,                          % text alignment [text centered | text badly centered | text justified | text ragged | text badly ragged]
        inner xsep      = 0mm,                  % minimum distance between text and borders along x dimension
        inner ysep      = 0mm,                  % minimum distance between text and borders along y dimension
        text height     = 0.2cm,
        text depth      = 0.12cm,
    }
}






\tikzset
{
    cnodeStyle/.style =
    {
        % -- shape properties --
        rectangle,                                  % shape
%        rounded corners = 1mm,                  % kind of corner (and radius of the roundness)
%       minimum height  = 0.15\textwidth,       % | minimum size of the node
%       minimum width   = 0.9\textwidth,        % |
        minimum size    = 8pt,                %
        rotate          = 0,                    % angle of rotation
        scale           = 1.0,                  % scaling factor
        thick,                                  % thickness of the border
        %
        % -- colours properties --
        % filling: [ trasparent | monocolored | shaded]; decomment what you prefer
%       %                                       % transparent (all commented)
        fill            = white,             % monocolored
%       top color       = white,                % | filling of the node
%       bottom color    = red!50!black!20,      % |
        text            = black,                % colour of the fonts
        draw            = black,                % colour of the border
        %
        % -- fonts --
        font            = \scriptsize,              % shape of the font (or dimension, like \tiny)
        text centered,                          % text alignment [text centered | text badly centered | text justified | text ragged | text badly ragged]
%       text height     = 1mm,                  % ! minimum size of the text    % NOT WORKING
%       text depth      = 1mm,                  % !
        inner xsep      = 0mm,                  % minimum distance between text and borders along x dimension
        inner ysep      = 0mm                   % minimum distance between text and borders along y dimension
    }
}




\tikzset {
    WarningTextStyle/.style =
    {
        rectangle,                      % shape
        rounded corners = 0.6cm,        %
        minimum size    = 1.2cm,          %
        rotate          = 0,            % angle of rotation
        scale           = 1.0,          % scaling factor
        thick,                          % thickness of the border
        fill            = red!10,       % monocolored
        text            = red!10!black, % colour of the fonts
        draw            = red,          % colour of the border
        font            = \large,       % shape of the font (or dimension, like \tiny)
        text centered,                  % text alignment
        text width      = 10cm,         % text alignment
        inner xsep      = 0.1cm,        % minimum distance between text and borders along x dimension
        inner ysep      = 0.1cm         % minimum distance between text and borders along y dimension
    }
}



\tikzset {
    RemarkTextStyle/.style =
    {
        rectangle,                      % shape
        rounded corners = 0.6cm,        %
        minimum size    = 1.2cm,          %
        rotate          = 0,            % angle of rotation
        scale           = 1.0,          % scaling factor
        thick,                          % thickness of the border
        fill            = blue!10,       % monocolored
        text            = blue!10!black, % colour of the fonts
        draw            = blue,          % colour of the border
        font            = \large,       % shape of the font (or dimension, like \tiny)
        text centered,                  % text alignment
        text width      = 10cm,         % text alignment
        inner xsep      = 0.3cm,        % minimum distance between text and borders along x dimension
        inner ysep      = 0.3cm         % minimum distance between text and borders along y dimension
    }
}




\tikzset
{
    FittingStyle/.style =
    {
        % -- shape properties --
        shape = ellipse,                            % shape
%       rounded corners = 1mm,                  % kind of corner (and radius of the roundness)
%       minimum height  = 0.15\textwidth,       % | minimum size of the node
%       minimum width   = 0.9\textwidth,        % |
%       minimum size    = 4.5mm,                %
%       rotate          = 0,                    % angle of rotation
        scale           = 1.0,                  % scaling factor
        thick,                                  % thickness of the border
        %
        % -- colours properties --
        % filling: [ trasparent | monocolored | shaded]; decomment what you prefer
%       %                                       % transparent (all commented)
%       fill            = black!30,             % monocolored
%       top color       = white,                % | filling of the node
%       bottom color    = red!50!black!20,      % |
%       text            = black,                % colour of the fonts
        draw            = red,              % colour of the border
        %
        % -- fonts --
%       font            = \scriptsize,              % shape of the font (or dimension, like \tiny)
%       text centered,                          % text alignment [text centered | text badly centered | text justified | text ragged | text badly ragged]
%       text height     = 1mm,                  % ! minimum size of the text    % NOT WORKING
%       text depth      = 1mm,                  % !
        inner xsep      = 0mm,                  % minimum distance between text and borders along x dimension
        inner ysep      = 0mm                   % minimum distance between text and borders along y dimension
    }
}



\tikzset
{
    GenericNodeStyle/.style =
    {
        % -- shape properties --
        shape = rectangle,                          % shape
        rounded corners = 3mm,                  % kind of corner (and radius of the roundness)
        minimum height  = 1cm,                  % | minimum size of the node
        minimum width   = 2cm,                  % |
        scale           = 1.0,                  % scaling factor
        thick,                                  % thickness of the border
        %
        % -- colours properties --
        % filling: [ trasparent | monocolored | shaded]; decomment what you prefer
%       %                                       % transparent (all commented)
        fill            = green!10!white,               % monocolored
%       top color       = white,                % | filling of the node
%       bottom color    = red!50!black!20,      % |
%       text            = black,                % colour of the fonts
        draw            = green,                % colour of the border
        %
        % -- fonts --
%       font            = \scriptsize,              % shape of the font (or dimension, like \tiny)
%       text centered,                          % text alignment [text centered | text badly centered | text justified | text ragged | text badly ragged]
%       text height     = 1mm,                  % ! minimum size of the text    % NOT WORKING
%       text depth      = 1mm,                  % !
        inner xsep      = 3mm,                  % minimum distance between text and borders along x dimension
        inner ysep      = 3mm                   % minimum distance between text and borders along y dimension
    }
}




\tikzset
{
    NormalNodeStyle/.style =
    {
        shape = circle,                         % shape
        minimum size    = 20,                   %
        rotate          = 0,                    % angle of rotation
        scale           = 1.0,                  % scaling factor
        thick,                                  % thickness of the border
        text            = black,                % colour of the fonts
        draw            = black,                % colour of the border
        font            = \small,               % shape of the font (or dimension, like \tiny)
        text centered,                          % text alignment
        inner xsep      = 0,                    % minimum distance between text and borders along x dimension
        inner ysep      = 0                     % minimum distance between text and borders along y dimension
    }
}



\tikzset
{
    BridgeNodeStyle/.style =
    {
        circle,                                 % shape
        minimum size    = 20,                   %
        rotate          = 0,                    % angle of rotation
        scale           = 1.0,                  % scaling factor
        thick,                                  % thickness of the border
        fill            = black!30,             % monocolored
        text            = black,                % colour of the fonts
        draw            = black,                % colour of the border
        font            = \small,               % shape of the font (or dimension, like \tiny)
        text centered,                          % text alignment
        inner xsep      = 0,                    % minimum distance between text and borders along x dimension
        inner ysep      = 0                     % minimum distance between text and borders along y dimension
    }
}






\tikzstyle{sNormalBlockStyle} = [
    draw,
    rectangle,
    rounded corners = 0.1cm,
    fill            = blue!20,
    minimum height  = 3em,
    minimum width   = 6em,
]





\tikzstyle{sSumBlockStyle} =
[
    shape           = circle,
    draw,
    fill            = blue!20,
]




\tikzstyle{sArrowsStyle} =
[
    thick,
    color   = black,
    -latex
]




\tikzstyle{sLinesStyle} =
[
    thick,
    color   = black,
    -
]




\tikzstyle{sTextBlockStyle} =
[
    draw,
    rectangle,
    drop shadow,
    rounded corners = 0.1cm,
    fill            = blue!10,
    thick,
    inner xsep      = 0.2cm,        % minimum distance between text and borders along x dimension
    inner ysep      = 0.2cm         % minimum distance between text and borders along y dimension
]



\tikzfading % DO NOT CHANGE THE ORDER OF THE COLORS otherwise it will not work
[
    name            = middle,
    top color       = transparent!100,
    bottom color    = transparent!100,
    middle color    = transparent!00,
]



\tikzstyle{sCoalFiredPlant} =
[
    shape           = rectangle,
    minimum height  = 0.5cm,
    minimum width   = 0.5cm,
    rounded corners = 0.1cm,
    fill            = black!20,
    draw            = black,
    line width      = 0.1cm,
    inner xsep      = 0.2cm,
    inner ysep      = 0.2cm
]

\input{../../bib/avinash_beamer.def}

\def\figpath{../Figures}
\def\mac_figpath{../Figures/MAC}
\def\cs_figpath{../Figures/CS}
\def\gt_figpath{../Figures/GT}
\graphicspath{{../Figures/}{../../SCLDPC-lattices/figures/}{../../SCLDPC-lattices/ISIT_Talk/Figures/}}


\begin{document}
\title{\bf Applications of coding theory to massive multiple access and big data problems}
\author{\textbf{Avinash Vem}\\ \vspace{3pt} \small Advisor: Dr.Krishna Narayanan} 
\vspace{10pt}
\institute{Department of Electrical and Computer Engineering \\ Texas A\&M University}
\date{} % if no date wanted, keep it blank

\titlegraphic{
\includegraphics[width=0.5in]{TAMULogoBox.pdf}
}	

\frame{\titlepage}
\begin{frame}
	\frametitle{Outline}
	\tableofcontents
\end{frame}

%%%%%%%%%%%%----------------------------------------------------------------------------------%%%%%%%%%%%%%%%
\begin{frame}
\frametitle{Motivation for Massive Multiple Access}

\begin{itemize}
\item 5G -- not just ``4G but faster'' but includes M2M transmissions, IoT
\item Current wireless -- a few devices with sustained data rate requirements
\item Future wireless -- {\color{blue}many uncoordinated} devices requiring {\color{blue}sporadic connectivity}
\end{itemize}

\begin{center}
\includegraphics[width=0.9\textwidth]{\mac_figpath/5Gchanginglandscape.pdf}
\end{center}
\end{frame}

%%%%%%%%%%%%----------------------------------------------------------------------------------%%%%%%%%%%%%%%%
\begin{frame}
\frametitle{Motivation for Compressed sensing and Group testing}

\begin{itemize}
\item Future wireless -- {\color{blue}many uncoordinated} devices requiring {\color{blue}sporadic connectivity}
\end{itemize}

\begin{center}
%\includegraphics[width=0.9\textwidth]{Figures/5Gchanginglandscape.pdf}
\end{center}
\end{frame}

%\section{Peeling decoder}
\subsection{LDPC Code}

\begin{frame}\frametitle{Tanner Graph - LDPC Code}
\begin{columns}

\column{0.45\textwidth}
\begin{defn}{Parity-Check Matrix}
\centering
\begin{align*}
H=
\begin{pmatrix}
1 & 0 & 1 & 1 & 0 & 0\\
1 & 1 & 0 & 0 & 1 & 0\\
0 & 1 & 1 & 0 & 0 & 1
\end{pmatrix}
\end{align*}
\vspace{6ex}
\begin{align*}
\text{LDPC Code } \mc{C}=\{x: H\odot x=0 \}
\end{align*}
\end{defn}

\pause

\column{0.45\textwidth}
\begin{defn}{Tanner Graph}
\centering
\vspace{0.6cm}
\resizebox{\textwidth}{!}{\begin{tikzpicture}
\def\horzgap{7ex}; %Horizontal gap between nodes/levels
\def \gapVN{4ex}; %vertical gap between variable nodes
\def \gapCN{6ex}; %Horizontal gap between check nodes
\def\nodewidth{1.5ex};

\def \textoffs{1ex}; %Offset for writing text east of a node
%\def\nodewidthA{0.05in};
%\def \edgewidth{0.02in};
%\def\ext{0.2in};
`
\def \n{8};
\def\ldeg{2};
\def \m {4};
\def\rdeg{3};

\def\langle{40};%120 degrees/3
\def\langle{20};%120 degrees/6

\tikzstyle{check} = [rectangle, draw, line width=0.75pt, inner sep=0mm, minimum height=\nodewidth, minimum width=\nodewidth]
\tikzstyle{bit} = [circle, draw,line width=0.75pt, inner sep=0mm, minimum size=\nodewidth]

                          
\foreach \vn in {1,...,6}{
 \node[bit] (vn\vn) at (0,-\vn*\gapVN) {\footnotesize $x_{\vn}$};
}


\foreach \cn in {1,...,3}{
\node[check] (cn\cn) at (\horzgap,-0.2*\gapCN-\cn*\gapCN) {};
}

\path (cn1.east)++(\textoffs,0) node ()[anchor=west] {\tiny{$x_1\oplus x_3\oplus x_4=0$}};

%\draw (vn6.east)--(cn4.west);
%\draw (vn3.east)--(cn4.west);
%\draw (vn1.east)--(cn4.west);

%\draw(vn2.east)--(cn3.west);
%\draw (vn4.east)--(cn3.west);
%\draw(vn5.east)--(cn3.west);

\draw (vn2.east)--(cn3.west);
\draw (vn3.east)--(cn3.west);
\draw  (vn6.east)--(cn3.west);

\draw (vn1.east)--(cn2.west);
\draw (vn2.east)--(cn2.west);
\draw (vn5.east)--(cn2.west);

\draw (vn1.east)--(cn1.west);
\draw (vn3.east)--(cn1.west);
\draw (vn4.east)--(cn1.west);

\node () at (0,-6.5*\gapVN){\tiny Variable Nodes};
\node () at (1.1*\horzgap,-3.5*\gapCN){\tiny Check Nodes};

\end{tikzpicture}}
\end{defn}

\end{columns}
\end{frame}

%--------------------------12314yoi34u9123092u308912------------------------------
\subsection{Decoder-BEC Channel}
\begin{frame}
\frametitle{Iterative peeling process}
\begin{columns}
\column{0.4\textwidth}
\begin{align*}
x_1\oplus x_3\oplus x_4&=0\\
x_1\oplus x_2\oplus x_5&=0\\
x_2\oplus x_3\oplus x_5&=0
\end{align*}

\vspace{0.2cm}
\centering
\only<2->{\includegraphics[scale=0.15,angle=270]{\figpath/BECchannelmodel.pdf}}

\column{0.6\textwidth}
\resizebox{0.9\textwidth}{!}{\pgfdeclarelayer{background}
\pgfdeclarelayer{foreground}
\pgfdeclarelayer{m-f}
\pgfdeclarelayer{main}

\pgfsetlayers{background,foreground}
\colorlet{LightBlue}{blue!10!white}
\colorlet{DarkBlue}{blue!80!white}

\begin{tikzpicture}[scale=1.0]
\clip (-0.15in,0.15in) rectangle (1.3in,-2.5in);

\def\n     {6}   % #-Variable nodes
\def\m     {3}  % #-Check nodes
\def\nodewidth{0.15in}
\def\nodegapVN{0.3in}
\def\nodegapCN{0.5in}

\tikzstyle{check} = [rectangle, draw, text centered, thick, fill=red,
                          minimum height=\nodewidth, minimum width=\nodewidth]
\tikzstyle{bit} = [circle, draw, text centered, thick, fill=LightBlue,
                          radius=0.5*\nodewidth]
\tikzstyle{bitpeeled} = [circle, draw, text centered, thick, fill=DarkBlue,
                          radius=0.5*\nodewidth]

\begin{pgfonlayer}{background}
%\draw[gray,step=0.5in] (-0.15in,0.15in) grid (1.5in,-2.5in);
\foreach \vn in {1,...,\n}{
  \node[bit] (vn\vn) at (0,-\vn*\nodegapVN) {};
 }

 \foreach \cn in {1,...,\m}{
  \node[check] (cn\cn) at (1in,-\cn*\nodegapCN) {};
 }
\end{pgfonlayer}



\begin{pgfonlayer}{foreground}

%Text to left of VN
\only<1>{
\foreach \vn in {1,...,\n}{
  \node[left] (nodetxt) at (vn\vn.west) {\normalsize{$x_\vn$}};
 	}  	
}

\only<2-8>{
\foreach \vn/\txt in {2/1,4/1,5/0}{
\node[left] (nodetxt) at (vn\vn.west) {\normalsize{\txt}};
 	}	
}

\only<2-3>\node[left] (nodetxt) at (vn1.west) {\normalsize{E}};
\only<2-5>\node[left] (nodetxt) at (vn3.west) {\normalsize{E}};
\only<2-7>\node[left] (nodetxt) at (vn6.west) {\normalsize{E}};


\only<4-8>\node[left] (nodetxt) at (vn1.west) {\normalsize{E=1}};
\only<6-8>\node[left] (nodetxt) at (vn3.west) {\normalsize{E=0}};
\only<8>\node[left] (nodetxt) at (vn6.west) {\normalsize{E=1}};

%Edges
\uncover<1-2>{
\foreach \vn/\cn in {2/2,2/3,4/1,5/2}{
 \draw[thick] (vn\vn.east)--(cn\cn.west);
  }
}

\only<1-3>\draw[thick] (vn1.east)--(cn2.west);
\only<1-4>\draw[thick] (vn1.east)--(cn1.west);

\only<1-5>\draw[thick] (vn3.east)--(cn1.west);
\only<1-6>\draw[thick] (vn3.east)--(cn3.west);

\only<1-5>\draw[thick] (vn3.east)--(cn1.west);
\only<1-6>\draw[thick] (vn3.east)--(cn3.west);

\only<1-7> \draw[thick] (vn6.east)--(cn3.west);

%% Peeled bits color
\uncover<3-8>{
  \foreach \vn in {2,4,5}{
    \node[bitpeeled] () at (vn\vn) {};
    }
  }
 \only<4-8>\node[bitpeeled] () at (vn1) {};
 \only<6-8>\node[bitpeeled] () at (vn3) {};
  \only<8>\node[bitpeeled] () at (vn6) {};

%Check node values
\only<2,5,6,7,8> \node[right] (nodetxt) at (cn1.east) {\normalsize{0}};
\only<3,4> \node[right] (nodetxt) at (cn1.east) {\normalsize{1}};

\only<2,4,5,6,7,8> \node[right] (nodetxt) at (cn2.east) {\normalsize{0}};
\only<3> \node[right] (nodetxt) at (cn2.east) {\normalsize{1}};

\only<2,6,7,8> \node[right] (nodetxt) at (cn3.east) {\normalsize{0}};
\only<3,4,5> \node[right] (nodetxt) at (cn3.east) {\normalsize{1}};



%% Text at the bottom
\only<1> \node[minimum width=10cm] (txt) at (0.5in,-7*\nodegapVN) {Tanner Graph};
\only<2> \node[minimum width=10cm] (txt) at (0.5in,-7*\nodegapVN) {Received block};
\only<3> \node[minimum width=10cm] (txt) at (0.5in,-7*\nodegapVN) {Peeling Step 1};
\only<4-5> \node[minimum width=10cm] (txt) at (0.5in,-7*\nodegapVN) {Peeling Step 2};
\only<6-7> \node[minimum width=10cm] (txt) at (0.5in,-7*\nodegapVN) {Peeling Step 3};
\only<8> \node[minimum width=10cm] (txt) at (0.5in,-7*\nodegapVN) {Peeling Step 4};

\end{pgfonlayer}
\end{tikzpicture} }
\end{columns}
\end{frame}

%--------------------------12314yoi34u9123092u308912------------------------------
\begin{frame}\frametitle{Threshold behavior}
\begin{columns}
\column{0.45\textwidth}
\begin{defn}{Threshold behavior}
\begin{itemize}
\item $N$-variable nodes, $m$-check nodes %$\epsilon$-erasure probability
\item If $m=\alpha N$ scales linearly, $\exists \alpha^{*}$
\begin{align*}
\lim_{N\rightarrow \infty}\Pr(\mc{E})=0 ~~ \text{ if }\alpha>\alpha^*
\end{align*}
\item $\Pr(\mc{E})>0$ if $\alpha<\alpha^*$
\end{itemize}
\end{defn}

\pause

\column{0.55\textwidth}
\includegraphics[width=\textwidth]{\figpath/BEC_MCT.png}
\begin{itemize}
\item $\Pr(\mc{E})$ for (3,6) d.d. for $N=2^i, i=6,\ldots,20$.  $m=N/2$
\end{itemize}

\end{columns}
\end{frame}

%--------------------------12314yoi34u9123092u308912------------------------------
\begin{frame}\frametitle{General framework}
\begin{columns}
\column{0.4\textwidth}
\resizebox{\textwidth}{!}{\begin{tikzpicture}
\def\horzgap{7ex}; %Horizontal gap between nodes/levels
\def \gapVN{4ex}; %vertical gap between variable nodes
\def \gapCN{5ex}; %Horizontal gap between check nodes
\def\nodewidth{1.5ex};

\def \textoffs{1ex}; %Offset for writing text east of a node
%\def\nodewidthA{0.05in};
%\def \edgewidth{0.02in};
%\def\ext{0.2in};
`
\def \n{8};
\def\ldeg{2};
\def \m {4};
\def\rdeg{3};

\def\langle{40};%120 degrees/3
\def\langle{20};%120 degrees/6

\tikzstyle{check} = [rectangle, draw, line width=0.75pt, inner sep=0mm, minimum height=\nodewidth, minimum width=\nodewidth]
\tikzstyle{bit} = [circle, draw,line width=0.75pt, inner sep=0mm, minimum size=\nodewidth]

                          
\foreach \vn in {1,...,6}{
 \node[bit] (vn\vn) at (0,-\vn*\gapVN) {};
}


\foreach \cn in {1,...,4}{
\node[check] (cn\cn) at (\horzgap,-0.1*\gapCN-\cn*\gapCN) {\tiny{$f()$}};
}

\path (cn2)++(0,-0.4*\gapCN) node () {\small{$\vdots$}};

\path (cn1.east) node[anchor=west] (ytexta) {\tiny{$y_1=f(\{x_i\})$}};
\path (ytexta) node[anchor=north] () {\tiny{$i\in\mc{N}_1$}};

%\path (cn4.east) node[anchor=west] () {\tiny{$y_m=f(\{x_i,i\in\mc{N}_m\})$}};

\path (cn4.east) node[anchor=west] (ytextb) {\tiny{$y_m=f(\{x_i\})$}};
\path (ytextb) node[anchor=north] () {\tiny{$i\in\mc{N}_m$}};


\draw (vn4.east)--(cn4.west);
\draw  (vn5.east)--(cn4.west);

\draw (vn2.east)--(cn3.west);
\draw (vn3.east)--(cn3.west);
\draw  (vn6.east)--(cn3.west);

\draw (vn1.east)--(cn2.west);
\draw (vn2.east)--(cn2.west);
\draw (vn5.east)--(cn2.west);

\draw (vn1.east)--(cn1.west);
\draw (vn3.east)--(cn1.west);
\draw (vn4.east)--(cn1.west);

%\node[text width=1.5cm] () at (0,-6.5*\gapVN){\tiny $N$ Variable Nodes $K$ unknown};
\node[align=left] () at (0,-6.5*\gapVN){\tiny $N$ Variable Nodes};
\node[align=left] () at (0,-6.8*\gapVN){\tiny $K$ unknown};
\node () at (1.2*\horzgap,-4.6*\gapCN){\tiny $m$ Function Nodes};

\end{tikzpicture}}

\column{0.6\textwidth}
\begin{itemize}
\item Need to recover $K$ unknowns$/N$ variables
\item We know values of $m$ functions $y_1,\ldots,y_m$
\item via peeling decoder, $m=cK$ sufficient {\tiny ($c>1$)}
\end{itemize}
\pause
\vspace{5ex}
\begin{itemize}
\item $c$ depends on the d.d of the graph, can be evaluated analytically via DE
\item $f$ amenable for peeling
\end{itemize}

\end{columns}
\end{frame}



\section{Unsourced multiple access}
%%%%%%%%%%%%----------------------------------------------------------------------------------%%%%%%%%%%%%%%%
\begin{frame}
\frametitle{System Model}

  \begin{itemize}
  \item $N$ \textbf{uncoordinated} users present in the system
  \item $K$ users \textbf{active} at any given time. Each user has 1 packet to transmit
  \item Access point(AP) interested only in the \textbf{set} of packets, not the \textbf{source}
  \item Time is \textbf{slotted}, transmissions occur in slots
    \begin{itemize}
	   \item In slot $j$, each user employs a policy to decide to transmit or not
  \end{itemize}

  \end{itemize}
\pause
  \begin{center}
  \input{\mac_figpath/slots}
  \end{center}
\end{frame}

%%%%%%%%%%%%----------------------------------------------------------------------------------%%%%%%%%%%%%%%%
\begin{frame}\frametitle{Problem statement}
  \begin{itemize}
  \item Each user has a $B$-bit message
  \item $n$ channel uses available per round
  \pause
  \item $\{w_1,w_2,\ldots,w_K\}$ - set of messages transmitted
  \item Channel model $\yv=\sum_{j=1,\ldots,K}\xv_{w_j}+\zv$
  \item Power constraint -$~\mathbb{E}_w[||\xv_w||^2]\leq nP$
  \item If $\mc{L}(\yv) $ is the decoder output, probability of decoding error per user
  \[
P_e=\frac{1}{K}\sum_{i=1}^{K} \Pr\left( w_i \notin \mc{L}(\yv) \right)  
  \]
  \pause
  \item Minimize $P$ such that $P_e\leq \epsilon$
  \end{itemize}
\end{frame}

%%%%%%%%%%%%----------------------------------------------------------------------------------%%%%%%%%%%%%%%%
\begin{frame}\frametitle{Tanner graph}
\centering
\resizebox{0.56\textwidth}{!}{\begin{tikzpicture}
\def\horzgap{7ex}; %Horizontal gap between nodes/levels
\def \gapVN{4ex}; %vertical gap between variable nodes
\def \gapCN{5ex}; %Horizontal gap between check nodes
\def\nodewidth{1.5ex};

\def \textoffs{1ex}; %Offset for writing text east of a node
%\def\nodewidthA{0.05in};
%\def \edgewidth{0.02in};
%\def\ext{0.2in};
`
\def \n{8};
\def\ldeg{2};
\def \m {4};
\def\rdeg{3};

\def\langle{40};%120 degrees/3
\def\langle{20};%120 degrees/6

\tikzstyle{check} = [rectangle, draw, line width=0.4pt, inner sep=0mm, minimum height=0.8*\nodewidth, minimum width=1.6*\nodewidth]%,rounded corners]
\tikzstyle{bit} = [circle, draw,line width=0.6pt, inner sep=0mm, minimum size=0.6*\nodewidth]

                          
\foreach \vn in {1,...,6}{
 \node[bit] (vn\vn) at (0,-\vn*\gapVN) {};
}


\foreach \cn in {1,...,4}{
\node[check] (cn\cn) at (\horzgap,-0.1*\gapCN-\cn*\gapCN) {};
}

\draw [decorate,decoration={brace,amplitude=1pt,mirror,raise=0.2pt},xshift=2*\nodewidth] (cn1.south east) -- (cn1.north east) node [black,midway,xshift=4ex] {\tiny $n/m$ ch. uses};

\path (cn2)++(0,-0.4*\gapCN) node () {\small{$\vdots$}};
\path (cn3.east)++(\textoffs,0) node ()[anchor=west] {\tiny{$\yv_j=\sum\limits_{i\in\mc{N}_j}\xv_i+\zv$}};

\draw (vn4.east)--(cn4.west);
\draw  (vn5.east)--(cn4.west);

\draw (vn2.east)--(cn3.west);
\draw (vn3.east)--(cn3.west);
\draw  (vn6.east)--(cn3.west);

\draw (vn1.east)--(cn2.west);
\draw (vn2.east)--(cn2.west);
\draw (vn5.east)--(cn2.west);

\draw (vn1.east)--(cn1.west);
\draw (vn3.east)--(cn1.west);
\draw (vn4.east)--(cn1.west);

\node[align=left] () at (0,-6.5*\gapVN){\tiny $K$ active users};
\node () at (\horzgap,-4.5*\gapCN){\tiny $m$ slots};
%\node[align=left] () at (1.2*\horzgap,-4.7*\gapCN){\tiny each $n/m$ channel uses};

\end{tikzpicture}}
\pause
\vspace{3ex}
\begin{itemize}
\item power constraint $\implies $ sparse Tanner graph 
\end{itemize}
\end{frame}

%%%%%%%%%%%%----------------------------------------------------------------------------------%%%%%%%%%%%%%%%
\begin{frame}\frametitle{Potential solution via peeling decoder}
\begin{itemize}
\item Joint decoding via successive interference cancellation: $\textbf{peeling}$ algorithm
\onslide<2->{
 \item Challenges in implementing peeling:
	\begin{itemize}
	\item Decoding in presence of noise
	\item Tx slots of users unknown at AP
	\item Power constraint $\implies$ need efficiency in repetition pattern
	}
	\end{itemize}
\end{itemize}

\begin{columns}
\column{0.5\textwidth}
\onslide<3->{
\begin{block}{Solution}
Proposed design scheme involves solutions to the above challenges
\end{block} 
}
\column{0.4\textwidth}
\centering
\resizebox{\textwidth}{!}{\input{\mac_figpath/MAC_tannergraph_dup}}
\end{columns}
\end{frame}

%%%%%%%%%%%%----------------------------------------------------------------------------------%%%%%%%%%%%%%%%
\begin{frame}
\frametitle{Main Result}
Main features of the proposed scheme:
\begin{itemize}
\item Encoder is independent of the user. Solely depends on the message
\pause
\item Message is split into two parts: $w=(w^{\mathrm{p}},w^{\mathrm{c}})$
\pause
	\begin{itemize}
	\item $w^{\mathrm{c}}$: encoded using an LDPC type channel code
	\item Designed for a $T$-user Gaussian multiple access(GMAC) channel
\pause
	\item $w^{\mathrm{p}}$: preamble, chooses an interleaver for the channel code
	\item  preamble encoded using a compressed sensing type scheme
	\end{itemize}
\pause
\item Encoded codeword is repeated in multiple slots
	\begin{itemize}
	\item  repetition pattern is dependent solely on the message $w$
	\end{itemize}
	\pause
\item Peeling based SIC decoding
\end{itemize}
\begin{block}{Main result}
Optimization of the individual components results in the best performing scheme to date
\end{block}
\end{frame}

%%%%%%%%%%%%----------------------------------------------------------------------------------%%%%%%%%%%%%%%%
\begin{frame} \frametitle{Multiple access - Prior Work}

\begin{itemize}
   \item Information-theoretic view of MAC (asymptotic in blocklength)
\begin{itemize}
\item  Liao-1972, Ahlswede-1973
\end{itemize}

   \item Network-theoretic view (random access strategy)
	\begin{itemize}
		\item ALOHA (Abramson 1970)
		\item Contention-resolution diversity slotted ALOHA (Cassini 2007)
	\end{itemize}

	\item Coding-theoretic view (small number of users)- CDMA, multi-user detection
	\pause
	\item Gaussian coding for unsourced MAC (Polyanskiy 2017)
	\begin{itemize}
		\item First considered the three issues together: 
		\begin{itemize}
			\item small payload; \emph{finite block length effects}
			\item finite number of users active at a given time \emph{random access}
			\item a large number of users in the system \emph{massive access}
		\end{itemize}
		\item Derived achievability bounds via random coding and joint typical decoding
	\end{itemize}
	
	\item Low complexity coding scheme for unsourced MAC (OP-2017)
	\begin{itemize}
		\item Concatenated coding scheme- Integer forcing and mod-$p$ $T$-user GMAC
		\item sub-optimal coding scheme for the $T$-user real adder GMAC
		\item Absence of SIC
	\end{itemize}
\end{itemize}

\end{frame}

%%%%%%%%%%%%----------------------------------------------------------------------------------%%%%%%%%%%%%%%%
\begin{frame}
\frametitle{Proposed design scheme}
\centering
\resizebox{0.85\textwidth}{!}{\begin{tikzpicture}

\def\nodewidth{0.5in}
\def\fsize{\Large}
\def\sfsize{\normalsize}
\def\xoffs{3.25cm}
\def\ya{3.5}
\def\xr{4}
\def\xslots{11}
\def\xdec{18}
\tikzstyle{block} = [rectangle, draw, thick, opacity=0.7,line width =2, minimum size=\nodewidth]
\tikzstyle{vertRectangle} = [rectangle, draw, opacity=0.7,line width =2, minimum width=\nodewidth, minimum height=12*\nodewidth]
\tikzstyle{opnode} = [rectangle, draw, thick,opacity=0.7,line width=1, minimum size=0.2in]

\node[block] (r11) at (-0.5,0){\fsize \textbf{Encoder} $\mc{C}$};
\node[block] (r21) at (\xr,0) {\fsize \begin{tabular}{c}
Repeat \\
$\ell(w_1)$ times
\end{tabular}};

\node[block] (r12) at (-0.5,-\ya){\fsize \textbf{Encoder} $\mc{C}$};
\node[block] (r22) at (\xr,-\ya) {\fsize \begin{tabular}{c}
Repeat \\
$\ell(w_i)$ times
\end{tabular}};

\node[block] (r13) at (-0.5,-3*\ya){\fsize \textbf{Encoder} $\mc{C}$};
\node[block] (r23) at (\xr,-3*\ya) {\fsize 
\begin{tabular}{c}
Repeat \\
$\ell(w_{K})$ times
\end{tabular}};

\node[vertRectangle] (r3) at (\xslots,-1.5*\ya){\fsize \bf Slots};
\node[vertRectangle,align=center] (r4) at (\xdec,-1.5*\ya){\fsize \bf \begin{tabular}{c}
Up to T-users\\
jointly decoded \\ 
 in each slot\\
+ \\ 
Serial Interference \\
Cancellation (SIC) \\
across slots.
\end{tabular}
};

\draw[<-, thick, line width=2] (r11.west)--node[midway, above]{\fsize $w_1=(w_1^{\mathrm{p}},w_1^{\mathrm{c}})$}+(-\xoffs,0);
\draw[->, thick, line width=2] (r11.east)--node[midway, above]{\fsize $\xv_{w_1}$}(r21.west);

\draw[<-, thick, line width=2] (r12.west)--node[midway, above]{\fsize $w_i=(w_i^{\mathrm{p}},w_i^{\mathrm{c}})$}+(-\xoffs,0);
\draw[->, thick, line width=2] (r12.east)--node[midway, above]{\fsize $\xv_{w_i}$}(r22.west);

\draw[<-, thick, line width=2] (r13.west)--node[midway, above]{\fsize $w_{K}=(w_K^{\mathrm{p}},w_K^{\mathrm{c}})$}+(-\xoffs,0);
\draw[->, thick, line width=2] (r13.east)--node[midway, above]{\fsize $\xv_{w_{K}}$}(r23.west);

\draw[transform canvas={yshift=-\nodewidth},thick] (r3.north west) -- (r3.north east);
\draw[transform canvas={yshift=-2*\nodewidth},thick] (r3.north west) -- (r3.north east);
\draw[transform canvas={yshift=-3*\nodewidth},thick] (r3.north west) -- (r3.north east);
\draw[transform canvas={yshift=\nodewidth}] (r3.south west) -- (r3.south east);
\node [below=0.3*\nodewidth of r3.north](s1) {\fsize Slot 1};
\node [below=1.3*\nodewidth of r3.north](s2) {\fsize Slot 2};
\node [below=2.3*\nodewidth of r3.north](s3) {\fsize Slot 3};
\node [above=0.3*\nodewidth of r3.south](s4) {\fsize Slot V};
\node [below=3.5*\nodewidth of r3.north]() {\Huge $\vdots$};
\node [above=3.5*\nodewidth of r3.south]() {\Huge $\vdots$};


\begin{scope}[very thick,decoration={
    markings,
    mark=at position 0.5 with {\arrow{>}}}
    ] 
\draw[transform canvas={yshift=-0.5*\nodewidth},postaction={decorate}] (r3.north east) -- (r4.north west);
\draw[transform canvas={yshift=-1.5*\nodewidth},postaction={decorate}] (r3.north east) -- (r4.north west);
\draw[transform canvas={yshift=-2.5*\nodewidth},postaction={decorate}] (r3.north east) -- (r4.north west);
\draw[transform canvas={yshift=0.5*\nodewidth},postaction={decorate}] (r3.south east) -- (r4.south west);
\end{scope}

\draw[<-, thick] (r3.north west)++(0,-2.5*\nodewidth)--node[midway,above] {\fsize $\xv_{w_1}$} (r21.east);
\draw[<-, thick] (r3.north west)++(0,-9.5*\nodewidth)--node[pos=0.85,above] {\fsize $\xv_{w_1}$} (r21.east);
\path (r21.east) -- ++(5:0.5cm) coordinate(r21a) 
		  (r21.east) -- ++(320:0.5cm) coordinate(r21b) ;
%\draw[<->,thick] (r21a) to [bend left=30] (r21b) node [right] {\fsize $L_{w_1}$};

\draw[<-, thick] (r3.north west)++(0,-2.5*\nodewidth)--(r22.east);
\draw[<-, thick] (r3.north west)++(0,-1.5*\nodewidth)--(r22.east);
\draw[<-, thick] (r3.north west)++(0,-8.5*\nodewidth)--(r22.east);
\path (r22.east) -- ++(25:0.5cm) coordinate(r22a) 
		  (r22.east) -- ++(335:0.5cm) coordinate(r22b) ;
%\draw[<->,thick] (r22a) to [bend left=30] (r22b) node [right] {\fsize $L_{w_i}$};


\draw[<-, thick] (r3.north west)++(0,-0.5*\nodewidth)--(r23.east);
\draw[<-, thick] (r3.north west)++(0,-6.5*\nodewidth)--(r23.east);
\path (r23.east) -- ++(22:0.5cm) coordinate(r23a) 
		  (r23.east) -- ++(55:0.5cm) coordinate(r23b) ;
%\draw[<->] (r23a) to [bend right=30] (r23b) node [right] {\fsize $L_{w_{K_a}}$};


\draw[->, thick, line width=2] (r4.east)--node[midway, above]{\fsize $\widehat{w}_1,\ldots, \widehat{w}_{K_a}$}+(\xoffs,0);
\end{tikzpicture}}
%\begin{itemize}
%\end{itemize}

\end{frame}



%%%%%%%%%%%%----------------------------------------------------------------------------------%%%%%%%%%%%%%%%
\begin{frame}
\frametitle{Encoder for $T$-GMAC}

\begin{itemize}
	\item B-bit message $w=(\wpdash,\wc).$ $\wpdash\in[1:2^{B_{\mathrm{p}}}],\wc\in[1:2^{B_{\mathrm{c}}}]$
	\begin{itemize}
		\item $B_{\mathrm{p}} \ll B_{\mathrm{c}}$ and $ B=B_{\mathrm{c}}+B_{\mathrm{p}}$
		\item $\wpdash,\wc$ encoded by compressed sensing and channel coding parts resp.
%		\item $\wpdash\in[1:2^{B_{\mathrm{p}}}]$ is encoded by the compressed sensing part
%		\item $\wc\in[1:2^{B_{\mathrm{c}}}]$ is encoded by the channel coding part		
	\end{itemize}
	\pause
	\item Compressed sensing encoder ($\wpdash$)
	\begin{itemize}
		\item sensing matrix $\mathbf{A}=[\av_1,\av_2,\ldots,\av_{2^{B_{\mathrm{p}}}}]$
		\end{itemize}
	\item Channel coding component ($\wc$)
	\begin{itemize}
		\item  $\mc{C}_{\mathrm{c}}=[\cv_1,\cv_2,\ldots,\cv_{2^{B_{\mathrm{c}}}}]$ be a LDPC type channel code
		\item interleaver $\pi_{f(\wpdash)}$ such that $f(\wpdash)\in [1:N_{\mathrm{c}}!]$
		\end{itemize}
\end{itemize}
\vspace{3ex}
\centering
\resizebox{0.6\textwidth}{!}{\begin{tikzpicture}

\def\fsize{\normalsize}
\def\fsizes{\scriptsize}
\def\ext{1}
%Message node and final codeword Rectangle
\node[font=\fsize,draw,rectangle] (msg) at (3,1) {\fsizes $w=(w_c,w_p)$};
\node[rectangle, draw, minimum width=2.7in,thick] (codeword) at (3+0.5\ext,1+2*\ext) {\fsizes $\pi_{\tau_{w_p}}(\vec{c}_{w_c})$};
%$c_{w}(\pi_{\tau_{w_p}^1}),c_{w}(\pi_{\tau_{w_p}^2}),\ldots,c_{w}(\pi_{\tau_{w_p}})$};


% From messages to Ch. Encoder and Compressive Sensing Encoder
\draw [->] (msg.east) -- +(0:\ext) node[midway, above] {\fsizes $w_c$} node[draw, inner sep=5pt,at end, anchor= west] (encoder) {\fsizes Ch. Encoder} ;
\draw [->] (msg.west) -- +(0:-\ext) node[midway, above] {\fsizes  $w_p$} node[draw, inner sep=5pt,at end, anchor= east] (CSencoder) {\fsizes $\mathbf{A}$}; %{Sensing Matrix $\mathbf{A}$};

%From Encoder to north
\path (encoder.north)-- +(90:0.6*\ext) node[draw, rectangle,fill=white](CWperm){\fsizes Permute};
\draw[->] (encoder.north)-- (CWperm.south) node[midway,right]{\fsizes $\vec{c}_{w_c}$};
\draw[->](CWperm.north)-- (CWperm.north |- codeword.south) node[midway, right]{\fsizes $\pi_{\tau_{w_p}}(\vec{c}_{w_c})$};

%From CSEncoder to north
\draw[->](CSencoder.north)-- (CSencoder.north |- codeword.south) node[midway, left]{\fsizes  $\vec{a}^T_{w_p}$};

%Partitioning the Tx codeword block
\path let \p{A}=(codeword) in (3-1.7*\ext,\y{A})node (partition){} -- (partition |- codeword.north);
\draw (partition |- codeword.north) -- (partition |- codeword.south) ;
\node () at (partition -| CSencoder) {\fsizes $\vec{a}_{w_p}$};
\node [above] at (codeword.north) {$\vec{\tilde{c}}_{w}$};

\draw [->] (msg.north) -- (msg.north |- CWperm.west)node[midway,left]{\fsizes $w_p$} -- (CWperm.west) node[midway,above] {\fsizes $\pi_{\tau_{w_p}}=f(w_p)$};
\end{tikzpicture} }

\end{frame}


%%%%%%%%%%%%----------------------------------------------------------------------------------%%%%%%%%%%%%%%%
\begin{frame}\frametitle{Decoder across slots: Peeling}
\begin{itemize}
\item Given $\yv_j=\xv_{w_i}+\zv_j$, assume we can decode $\xv_{w_i}$
\end{itemize}
\centering
%\resizebox{0.4\textwidth}{!}{\begin{tikzpicture}
\def\horzgap{0.125in}; %Horizontal gap between nodes/levels
\def \gapVN{0.075in}; %vertical gap between nodes
\def \gapCN{0.1in}; %Horizontal gap between nodes


\def\nodewidth{0.5ex}; 
\def\nodewidthA{0.5ex};
\def \edgewidth{15ex}; 
\def\ext{0.1in};


\tikzstyle{check} = [rectangle, draw,line width=0.07mm, inner sep=0mm, minimum height=\nodewidthA, minimum width=\nodewidthA]
\tikzstyle{bit} = [circle, draw, line width=0.07mm, inner sep=0mm,  minimum size=\nodewidthA]
\tikzstyle{bituncover} = [circle, draw=none, line width=0.05mm, inner sep=0mm, fill=gray, minimum size=\nodewidthA]

 \def\moveX {1.8*\nodewidth};
\def\moveXA {2*\nodewidth};

            
\onslide<1>{             
\foreach \vn in {1,...,6}{
 \node[bit] (vn\vn) at (0,-\vn*\gapVN) {};
}

\foreach \vn in {1,...,6}{
\path (vn\vn) ++(-\nodewidth,0) node()[scale=0.25, inner sep=0mm] {\tiny{$\vec{x}_{\vn}$}};
}

\foreach \cn in {1,...,4}{
\node[check] (cn\cn) at (\horzgap,-\cn*\gapCN) {};
}

\draw[line width=0.05mm] (vn4.east)--(cn4.west);
\draw[line width=0.05mm] (vn3.east)--(cn4.west);

\draw[line width=0.05mm] (vn2.east)--(cn3.west);
\draw[line width=0.05mm] (vn1.east)--(cn3.west);

\draw[line width=0.05mm] (vn2.east)--(cn2.west);

\draw[line width=0.05mm] (vn1.east)--(cn1.west);
\draw[line width=0.05mm] (vn3.east)--(cn1.west);
\draw[line width=0.05mm] (vn5.east)--(cn1.west);
\draw[line width=0.05mm] (vn6.east)--(cn1.west);


\node [scale=0.2,anchor=west] at (cn1.east) {\tiny{$\vec{x}_1+\vec{x}_3+\vec{x}_5+\vec{x}_6+\vec{z}_1$}};
\node [scale=0.2,anchor=west] at (cn2.east) {\tiny{$\vec{x}_2+\vec{z}_2$}};
\node [scale=0.2,anchor=west] at (cn3.east) {\tiny{$\vec{x}_1+\vec{x}_2+\vec{z}_3$}};
\node [scale=0.2,anchor=west] at (cn4.east) {\tiny{$\vec{x}_3+\vec{x}_4+\vec{z}_4$}};

}

%-----------------------*(&^#@$^&*(^%$^&*(&^--------------------------------------------------------------------
%Dotted x_2
\onslide<2>{
\foreach \vn in {1,3,4,5,6}{
 \node[bit] (vn\vn) at (0,-\vn*\gapVN) {};
}

\foreach \vn in {1,3,4,5,6}{
\path (vn\vn) ++(-\nodewidth,0) node()[scale=0.25, inner sep=0mm] {\tiny{$\vec{x}_{\vn}$}};
}


\foreach \vn in {2}{
 \node[bituncover] (vn\vn) at (0,-\vn*\gapVN) {};
}

\foreach \vn in {2}{
\path (vn\vn) ++(-\nodewidth,0) node()[scale=0.25, inner sep=0mm] {\tiny{$\hat{\vec{x}}_2$}};
}


\foreach \cn in {1,...,4}{
\node[check] (cn\cn) at (\horzgap,-\cn*\gapCN) {};
}


\draw[line width=0.05mm] (vn4.east)--(cn4.west);
\draw[line width=0.05mm] (vn3.east)--(cn4.west);

\draw[line width=0.05mm, densely dotted] (vn2.east)--(cn3.west);
\draw[line width=0.05mm] (vn1.east)--(cn3.west);

\draw[line width=0.05mm, densely dotted] (vn2.east)--(cn2.west);

\draw[line width=0.05mm] (vn1.east)--(cn1.west);
\draw[line width=0.05mm] (vn3.east)--(cn1.west);
\draw[line width=0.05mm] (vn5.east)--(cn1.west);
\draw[line width=0.05mm] (vn6.east)--(cn1.west);

\node [scale=0.2,anchor=west] at (cn1.east) {\tiny{$\vec{x}_1+\vec{x}_3+\vec{x}_5+\vec{x}_6+\vec{z}_1$}};
\node [scale=0.2,anchor=west] at (cn2.east) {\tiny{$\vec{x}_2+\vec{z}_2$}};
\node [scale=0.2,anchor=west] at (cn3.east) {\tiny{$\vec{x}_1+\vec{x}_2+\vec{z}_3$}};
\node [scale=0.2,anchor=west] at (cn4.east) {\tiny{$\vec{x}_3+\vec{x}_4+\vec{z}_4$}};
}
%----------------------------------^%$#@%^&*()_*&^%$#^&*------------------------------------
%--Peeled x_2
\onslide<3>{
\foreach \vn in {1,3,4,5,6}{
 \node[bit] (vn\vn) at (0,-\vn*\gapVN) {};
}

\foreach \vn in {1,3,4,5,6}{
\path (vn\vn) ++(-\nodewidth,0) node()[scale=0.25, inner sep=0mm] {\tiny{$\vec{x}_{\vn}$}};
}


\foreach \vn in {2}{
 \node[bituncover] (vn\vn) at (0,-\vn*\gapVN) {};
}

\foreach \vn in {2}{
\path (vn\vn) ++(-\nodewidth,0) node()[scale=0.25, inner sep=0mm] {\tiny{$\hat{\vec{x}}_2$}};
}


\foreach \cn in {1,...,4}{
\node[check] (cn\cn) at (\horzgap,-\cn*\gapCN) {};
}


\draw[line width=0.05mm] (vn4.east)--(cn4.west);
\draw[line width=0.05mm] (vn3.east)--(cn4.west);

\draw[line width=0.05mm] (vn1.east)--(cn3.west);

\draw[line width=0.05mm] (vn1.east)--(cn1.west);
\draw[line width=0.05mm] (vn3.east)--(cn1.west);
\draw[line width=0.05mm] (vn5.east)--(cn1.west);
\draw[line width=0.05mm] (vn6.east)--(cn1.west);

\node [scale=0.2,anchor=west] at (cn1.east) {\tiny{$\vec{x}_1+\vec{x}_3+\vec{x}_5+\vec{x}_6+\vec{z}_1$}};
\node [scale=0.2,anchor=west] at (cn2.east) {\tiny{$\vec{z}_2$}};
\node [scale=0.2,anchor=west] at (cn3.east) {\tiny{$\vec{x}_1+\vec{z}_3$}};
\node [scale=0.2,anchor=west] at (cn4.east) {\tiny{$\vec{x}_3+\vec{x}_4+\vec{z}_4$}};
}

%---------%^&*^%$#^&----------------------------
%Dotted x_1
\onslide<4>{
\foreach \vn in {3,4,5,6}{
 \node[bit] (vn\vn) at (0,-\vn*\gapVN) {};
}

\foreach \vn in {1,2}{
 \node[bituncover] (vn\vn) at (0,-\vn*\gapVN) {};
}


\foreach \vn in {3,4,5,6}{
\path (vn\vn) ++(-\nodewidth,0) node()[scale=0.25, inner sep=0mm] {\tiny{$\vec{x}_{\vn}$}};
}

\foreach \vn in {1,2}{
\path (vn\vn) ++(-\nodewidth,0) node()[scale=0.25, inner sep=0mm] {\tiny{$\hat{\vec{x}}_{\vn}$}};
}

\foreach \cn in {1,...,4}{
\node[check] (cn\cn) at (\horzgap,-\cn*\gapCN) {};
}


\draw[line width=0.05mm] (vn4.east)--(cn4.west);
\draw[line width=0.05mm] (vn3.east)--(cn4.west);

\draw[line width=0.05mm, densely dotted] (vn1.east)--(cn3.west);
\draw[line width=0.05mm, densely dotted] (vn1.east)--(cn1.west);

\draw[line width=0.05mm] (vn3.east)--(cn1.west);
\draw[line width=0.05mm] (vn5.east)--(cn1.west);
\draw[line width=0.05mm] (vn6.east)--(cn1.west);


\node [scale=0.2,anchor=west] at (cn1.east) {\tiny{$\vec{x}_1+\vec{x}_3+\vec{x}_5+\vec{x}_6+\vec{z}_1$}};
\node [scale=0.2,anchor=west] at (cn2.east) {\tiny{$\vec{z}_2$}};
\node [scale=0.2,anchor=west] at (cn3.east) {\tiny{$\vec{x}_1+\vec{z}_3$}};
\node [scale=0.2,anchor=west] at (cn4.east) {\tiny{$\vec{x}_3+\vec{x}_4+\vec{z}_4$}};
}

%----------------------------------^%$#@%^&*()_*&^%$#^&*------------------------------------
%-peeled off x_1
\onslide<5>{
\foreach \vn in {3,4,5,6}{
 \node[bit] (vn\vn) at (0,-\vn*\gapVN) {};
}

\foreach \vn in {1,2}{
 \node[bituncover] (vn\vn) at (0,-\vn*\gapVN) {};
}


\foreach \vn in {3,4,5,6}{
\path (vn\vn) ++(-\nodewidth,0) node()[scale=0.25, inner sep=0mm] {\tiny{$\vec{x}_{\vn}$}};
}

\foreach \vn in {1,2}{
\path (vn\vn) ++(-\nodewidth,0) node()[scale=0.25, inner sep=0mm] {\tiny{$\hat{\vec{x}}_{\vn}$}};
}

\foreach \cn in {1,...,4}{
\node[check] (cn\cn) at (\horzgap,-\cn*\gapCN) {};
}


\draw[line width=0.05mm] (vn4.east)--(cn4.west);
\draw[line width=0.05mm] (vn3.east)--(cn4.west);

\draw[line width=0.05mm] (vn3.east)--(cn1.west);
\draw[line width=0.05mm] (vn5.east)--(cn1.west);
\draw[line width=0.05mm] (vn6.east)--(cn1.west);


\node [scale=0.2,anchor=west] at (cn1.east) {\tiny{$\vec{x}_3+\vec{x}_5+\vec{x}_6+\vec{z}_1$}};
\node [scale=0.2,anchor=west] at (cn2.east) {\tiny{$\vec{z}_2$}};
\node [scale=0.2,anchor=west] at (cn3.east) {\tiny{$\vec{z}_3$}};
\node [scale=0.2,anchor=west] at (cn4.east) {\tiny{$\vec{x}_3+\vec{x}_4+\vec{z}_4$}};
}

%----------------------------------^%$#@%^&*()_*&^%$#^&*------------------------------------
\onslide<6>{
\foreach \vn in {3,4,5,6}{
 \node[bit] (vn\vn) at (0,-\vn*\gapVN) {};
}

\foreach \vn in {1,2}{
 \node[bituncover] (vn\vn) at (0,-\vn*\gapVN) {};
}


\foreach \vn in {3,4,5,6}{
\path (vn\vn) ++(-\nodewidth,0) node()[scale=0.25, inner sep=0mm] {\tiny{$\vec{x}_{\vn}$}};
}

\foreach \vn in {1,2}{
\path (vn\vn) ++(-\nodewidth,0) node()[scale=0.25, inner sep=0mm] {\tiny{$\hat{\vec{x}}_{\vn}$}};
}

\foreach \cn in {1,...,4}{
\node[check] (cn\cn) at (\horzgap,-\cn*\gapCN) {};
}


\draw[line width=0.05mm] (vn4.east)--(cn4.west);
\draw[line width=0.05mm] (vn3.east)--(cn4.west);

\draw[line width=0.05mm] (vn3.east)--(cn1.west);
\draw[line width=0.05mm] (vn5.east)--(cn1.west);
\draw[line width=0.05mm] (vn6.east)--(cn1.west);


\node [scale=0.2,anchor=west] at (cn1.east) {\tiny{$\vec{x}_3+\vec{x}_5+\vec{x}_6+\vec{z}_1$}};
\node [scale=0.2,anchor=west] at (cn2.east) {\tiny{$\vec{z}_2$}};
\node [scale=0.2,anchor=west] at (cn3.east) {\tiny{$\vec{z}_3$}};
\node [scale=0.2,anchor=west] at (cn4.east) {\tiny{$\vec{x}_3+\vec{x}_4+\vec{z}_4$}};

\node[scale=0.35] () at (0.5*\horzgap,-5*\gapCN) {\tiny{Stuck}};
}
\end{tikzpicture}}
\end{frame}

%%%%%%%%%%%%----------------------------------------------------------------------------------%%%%%%%%%%%%%%%
\begin{frame}
\frametitle{Decoder across slots: $T$-Peeling}
\begin{itemize}
\item In slot $j$, assume we can decode $\{\xv_{w_i},i\in\mc{N}_j\}$ if $|\mc{N}_j|\leq T$
\[
\yv_j=\sum\limits_{i\in\mc{N}_j}\xv_{w_i}+\zv_j
\]
\end{itemize}
\centering
%\resizebox{0.4\textwidth}{!}{\begin{tikzpicture}
\def\horzgap{0.125in}; %Horizontal gap between nodes/levels
\def \gapVN{0.075in}; %vertical gap between nodes
\def \gapCN{0.1in}; %Horizontal gap between nodes


\def\nodewidth{0.5ex}; 
\def\nodewidthA{0.5ex};
\def \edgewidth{15ex}; 
\def\ext{0.1in};


\tikzstyle{check} = [rectangle, draw,line width=0.07mm, inner sep=0mm, minimum height=\nodewidthA, minimum width=\nodewidthA]
\tikzstyle{bit} = [circle, draw, line width=0.07mm, inner sep=0mm,  minimum size=\nodewidthA]
\tikzstyle{bituncover} = [circle, draw=none, line width=0.05mm, inner sep=0mm, fill=gray, minimum size=\nodewidthA]

 \def\moveX {1.8*\nodewidth};
\def\moveXA {2*\nodewidth};

            
\onslide<1>{             
\foreach \vn in {1,...,6}{
 \node[bit] (vn\vn) at (0,-\vn*\gapVN) {};
}

\foreach \vn in {1,...,6}{
\path (vn\vn) ++(-\nodewidth,0) node()[scale=0.25, inner sep=0mm] {\tiny{$\vec{x}_{\vn}$}};
}

\foreach \cn in {1,...,4}{
\node[check] (cn\cn) at (\horzgap,-\cn*\gapCN) {};
}

\draw[line width=0.05mm] (vn4.east)--(cn4.west);
\draw[line width=0.05mm] (vn3.east)--(cn4.west);

\draw[line width=0.05mm] (vn2.east)--(cn3.west);
\draw[line width=0.05mm] (vn1.east)--(cn3.west);

\draw[line width=0.05mm] (vn2.east)--(cn2.west);

\draw[line width=0.05mm] (vn1.east)--(cn1.west);
\draw[line width=0.05mm] (vn3.east)--(cn1.west);
\draw[line width=0.05mm] (vn5.east)--(cn1.west);
\draw[line width=0.05mm] (vn6.east)--(cn1.west);


\node [scale=0.2,anchor=west] at (cn1.east) {\tiny{$\vec{x}_1+\vec{x}_3+\vec{x}_5+\vec{x}_6+\vec{z}_1$}};
\node [scale=0.2,anchor=west] at (cn2.east) {\tiny{$\vec{x}_2+\vec{z}_2$}};
\node [scale=0.2,anchor=west] at (cn3.east) {\tiny{$\vec{x}_1+\vec{x}_2+\vec{z}_3$}};
\node [scale=0.2,anchor=west] at (cn4.east) {\tiny{$\vec{x}_3+\vec{x}_4+\vec{z}_4$}};

}

%-----------------------*(&^#@$^&*(^%$^&*(&^--------------------------------------------------------------------
%Dotted x_2
\onslide<2>{
\foreach \vn in {1,3,4,5,6}{
 \node[bit] (vn\vn) at (0,-\vn*\gapVN) {};
}

\foreach \vn in {1,3,4,5,6}{
\path (vn\vn) ++(-\nodewidth,0) node()[scale=0.25, inner sep=0mm] {\tiny{$\vec{x}_{\vn}$}};
}


\foreach \vn in {2}{
 \node[bituncover] (vn\vn) at (0,-\vn*\gapVN) {};
}

\foreach \vn in {2}{
\path (vn\vn) ++(-\nodewidth,0) node()[scale=0.25, inner sep=0mm] {\tiny{$\hat{\vec{x}}_2$}};
}


\foreach \cn in {1,...,4}{
\node[check] (cn\cn) at (\horzgap,-\cn*\gapCN) {};
}


\draw[line width=0.05mm] (vn4.east)--(cn4.west);
\draw[line width=0.05mm] (vn3.east)--(cn4.west);

\draw[line width=0.05mm, densely dotted] (vn2.east)--(cn3.west);
\draw[line width=0.05mm] (vn1.east)--(cn3.west);

\draw[line width=0.05mm, densely dotted] (vn2.east)--(cn2.west);

\draw[line width=0.05mm] (vn1.east)--(cn1.west);
\draw[line width=0.05mm] (vn3.east)--(cn1.west);
\draw[line width=0.05mm] (vn5.east)--(cn1.west);
\draw[line width=0.05mm] (vn6.east)--(cn1.west);

\node [scale=0.2,anchor=west] at (cn1.east) {\tiny{$\vec{x}_1+\vec{x}_3+\vec{x}_5+\vec{x}_6+\vec{z}_1$}};
\node [scale=0.2,anchor=west] at (cn2.east) {\tiny{$\vec{x}_2+\vec{z}_2$}};
\node [scale=0.2,anchor=west] at (cn3.east) {\tiny{$\vec{x}_1+\vec{x}_2+\vec{z}_3$}};
\node [scale=0.2,anchor=west] at (cn4.east) {\tiny{$\vec{x}_3+\vec{x}_4+\vec{z}_4$}};
}
%----------------------------------^%$#@%^&*()_*&^%$#^&*------------------------------------
%--Peeled x_2
\onslide<3>{
\foreach \vn in {1,3,4,5,6}{
 \node[bit] (vn\vn) at (0,-\vn*\gapVN) {};
}

\foreach \vn in {1,3,4,5,6}{
\path (vn\vn) ++(-\nodewidth,0) node()[scale=0.25, inner sep=0mm] {\tiny{$\vec{x}_{\vn}$}};
}


\foreach \vn in {2}{
 \node[bituncover] (vn\vn) at (0,-\vn*\gapVN) {};
}

\foreach \vn in {2}{
\path (vn\vn) ++(-\nodewidth,0) node()[scale=0.25, inner sep=0mm] {\tiny{$\hat{\vec{x}}_2$}};
}


\foreach \cn in {1,...,4}{
\node[check] (cn\cn) at (\horzgap,-\cn*\gapCN) {};
}


\draw[line width=0.05mm] (vn4.east)--(cn4.west);
\draw[line width=0.05mm] (vn3.east)--(cn4.west);

\draw[line width=0.05mm] (vn1.east)--(cn3.west);

\draw[line width=0.05mm] (vn1.east)--(cn1.west);
\draw[line width=0.05mm] (vn3.east)--(cn1.west);
\draw[line width=0.05mm] (vn5.east)--(cn1.west);
\draw[line width=0.05mm] (vn6.east)--(cn1.west);

\node [scale=0.2,anchor=west] at (cn1.east) {\tiny{$\vec{x}_1+\vec{x}_3+\vec{x}_5+\vec{x}_6+\vec{z}_1$}};
\node [scale=0.2,anchor=west] at (cn2.east) {\tiny{$\vec{z}_2$}};
\node [scale=0.2,anchor=west] at (cn3.east) {\tiny{$\vec{x}_1+\vec{z}_3$}};
\node [scale=0.2,anchor=west] at (cn4.east) {\tiny{$\vec{x}_3+\vec{x}_4+\vec{z}_4$}};
}

%---------%^&*^%$#^&----------------------------
%Dotted x_1
\onslide<4>{
\foreach \vn in {3,4,5,6}{
 \node[bit] (vn\vn) at (0,-\vn*\gapVN) {};
}

\foreach \vn in {1,2}{
 \node[bituncover] (vn\vn) at (0,-\vn*\gapVN) {};
}


\foreach \vn in {3,4,5,6}{
\path (vn\vn) ++(-\nodewidth,0) node()[scale=0.25, inner sep=0mm] {\tiny{$\vec{x}_{\vn}$}};
}

\foreach \vn in {1,2}{
\path (vn\vn) ++(-\nodewidth,0) node()[scale=0.25, inner sep=0mm] {\tiny{$\hat{\vec{x}}_{\vn}$}};
}

\foreach \cn in {1,...,4}{
\node[check] (cn\cn) at (\horzgap,-\cn*\gapCN) {};
}


\draw[line width=0.05mm] (vn4.east)--(cn4.west);
\draw[line width=0.05mm] (vn3.east)--(cn4.west);

\draw[line width=0.05mm, densely dotted] (vn1.east)--(cn3.west);
\draw[line width=0.05mm, densely dotted] (vn1.east)--(cn1.west);

\draw[line width=0.05mm] (vn3.east)--(cn1.west);
\draw[line width=0.05mm] (vn5.east)--(cn1.west);
\draw[line width=0.05mm] (vn6.east)--(cn1.west);


\node [scale=0.2,anchor=west] at (cn1.east) {\tiny{$\vec{x}_1+\vec{x}_3+\vec{x}_5+\vec{x}_6+\vec{z}_1$}};
\node [scale=0.2,anchor=west] at (cn2.east) {\tiny{$\vec{z}_2$}};
\node [scale=0.2,anchor=west] at (cn3.east) {\tiny{$\vec{x}_1+\vec{z}_3$}};
\node [scale=0.2,anchor=west] at (cn4.east) {\tiny{$\vec{x}_3+\vec{x}_4+\vec{z}_4$}};
}

%----------------------------------^%$#@%^&*()_*&^%$#^&*------------------------------------
%-peeled off x_1
\onslide<5>{
\foreach \vn in {3,4,5,6}{
 \node[bit] (vn\vn) at (0,-\vn*\gapVN) {};
}

\foreach \vn in {1,2}{
 \node[bituncover] (vn\vn) at (0,-\vn*\gapVN) {};
}


\foreach \vn in {3,4,5,6}{
\path (vn\vn) ++(-\nodewidth,0) node()[scale=0.25, inner sep=0mm] {\tiny{$\vec{x}_{\vn}$}};
}

\foreach \vn in {1,2}{
\path (vn\vn) ++(-\nodewidth,0) node()[scale=0.25, inner sep=0mm] {\tiny{$\hat{\vec{x}}_{\vn}$}};
}

\foreach \cn in {1,...,4}{
\node[check] (cn\cn) at (\horzgap,-\cn*\gapCN) {};
}


\draw[line width=0.05mm] (vn4.east)--(cn4.west);
\draw[line width=0.05mm] (vn3.east)--(cn4.west);

\draw[line width=0.05mm] (vn3.east)--(cn1.west);
\draw[line width=0.05mm] (vn5.east)--(cn1.west);
\draw[line width=0.05mm] (vn6.east)--(cn1.west);


\node [scale=0.2,anchor=west] at (cn1.east) {\tiny{$\vec{x}_3+\vec{x}_5+\vec{x}_6+\vec{z}_1$}};
\node [scale=0.2,anchor=west] at (cn2.east) {\tiny{$\vec{z}_2$}};
\node [scale=0.2,anchor=west] at (cn3.east) {\tiny{$\vec{z}_3$}};
\node [scale=0.2,anchor=west] at (cn4.east) {\tiny{$\vec{x}_3+\vec{x}_4+\vec{z}_4$}};
}

%----------------------------------^%$#@%^&*()_*&^%$#^&*------------------------------------
\onslide<6>{
\foreach \vn in {3,4,5,6}{
 \node[bit] (vn\vn) at (0,-\vn*\gapVN) {};
}

\foreach \vn in {1,2}{
 \node[bituncover] (vn\vn) at (0,-\vn*\gapVN) {};
}


\foreach \vn in {3,4,5,6}{
\path (vn\vn) ++(-\nodewidth,0) node()[scale=0.25, inner sep=0mm] {\tiny{$\vec{x}_{\vn}$}};
}

\foreach \vn in {1,2}{
\path (vn\vn) ++(-\nodewidth,0) node()[scale=0.25, inner sep=0mm] {\tiny{$\hat{\vec{x}}_{\vn}$}};
}

\foreach \cn in {1,...,4}{
\node[check] (cn\cn) at (\horzgap,-\cn*\gapCN) {};
}


\draw[line width=0.05mm] (vn4.east)--(cn4.west);
\draw[line width=0.05mm] (vn3.east)--(cn4.west);

\draw[line width=0.05mm] (vn3.east)--(cn1.west);
\draw[line width=0.05mm] (vn5.east)--(cn1.west);
\draw[line width=0.05mm] (vn6.east)--(cn1.west);


\node [scale=0.2,anchor=west] at (cn1.east) {\tiny{$\vec{x}_3+\vec{x}_5+\vec{x}_6+\vec{z}_1$}};
\node [scale=0.2,anchor=west] at (cn2.east) {\tiny{$\vec{z}_2$}};
\node [scale=0.2,anchor=west] at (cn3.east) {\tiny{$\vec{z}_3$}};
\node [scale=0.2,anchor=west] at (cn4.east) {\tiny{$\vec{x}_3+\vec{x}_4+\vec{z}_4$}};

\node[scale=0.35] () at (0.5*\horzgap,-5*\gapCN) {\tiny{Stuck}};
}
\end{tikzpicture}}

\end{frame}

%%%%%%%%%%%%----------------------------------------------------------------------------------%%%%%%%%%%%%%%%
\begin{frame}\frametitle{Decoding in a slot: Compressed sensing}
\begin{itemize}
	\item Received signal: $\yv_j=\sum_{i\in\mc{N}_j}\xv_{w_i}+\zv_i$, $~~|\mc{N}_j|\leq T$
	\item Input to the Compressed sensing decoder: $\yv^{\mathrm{p}}_j=\sum_{i\in\mc{N}_j}\av_{\wpdash_i}+\zv^{\mathrm{p}}_i	$
\end{itemize}

\onslide<2->{
\begin{align*}
	\yv^{\mathrm{p}}_j	=\mathbf{A}\vec{b}_j+\zv^{\mathrm{p}}_i \qquad \text{where }\vec{b}_j\in\{0,1\}^{2^{\Bp}},~ |\vec{b}_j|_1=T.
	\end{align*}
		}


\onslide<3->{
\begin{itemize}
\item \emph{Step1}: Get a coarse estimate of $\vec{b}_j$ via sub-optimal CS algorithms
\item Form a list of positive support set
\item \emph{Step2}: Refine the estimate by doing ML-estimation on the list
\end{itemize}
}
\onslide<1-2>{
\centering
\vspace{2ex}
\resizebox{0.4\textwidth}{!}{\begin{tikzpicture}

\def\fsize{\normalsize}
\def\fsizes{\scriptsize}
\def\ext{1}
\def\yoffs{-1.5cm}

%Message node and final codeword Rectangle
\node[rectangle, draw,minimum width=2.7in,thick] (codeword) at (3+0.5\ext,1+2*\ext) {\fsizes $\pi_{f(\wpdash_1)}(\vec{c}_{\wc})$};
\path let \p{A}=(codeword) in (3-1.7*\ext,\y{A})node (partition){} -- (partition |- codeword.north);
\draw (partition |- codeword.north) --  (partition |- codeword.south) ;
\path (partition) -- node[midway](csvec){\fsizes $\av_{\wpdash_1}$}  (codeword.west);

\path (codeword)-- +(0,0.5*\yoffs) node () {\large $+$};


\path (codeword)-- +(0,\yoffs) node[rectangle, draw,minimum width=2.7in,thick] (codeworda) {\fsizes $\pi_{f(\wpdash_2)}(\vec{c}_{\wc_2})$};
\draw (partition)++(0,\yoffs) -- (partition |- codeworda.north);
\draw (partition)++(0,\yoffs) -- (partition |- codeworda.south);
\node () at (codeworda.west -| csvec) {\fsizes $\av_{\wpdash_2}$}  ;

\end{tikzpicture} }
}
%	\begin{itemize}
%	\item LASSO: 
%	\begin{align*}
%	\hat{b}_j&=\min_{\vec{\beta}} |\yv_j-\mathbf{A}\vec{\beta}|^2+\lambda |\vec{\beta}|_1\\
%	\text{subj. to }& \beta_i\geq 0 ~\forall i.
%\end{align*}
%
%\item Least-squares: 
%\begin{align*}
%\hat{b}_j &=\min_{\vec{\beta}} |\yv_j-\mathbf{A}\vec{\beta}|^2\\
%\text{subj. to }& \beta_i\geq 0 ~\forall i.
%\end{align*}
\end{frame}


\begin{frame}\frametitle{Decoding in a slot: Compressed sensing}
	\begin{align*}
	\yv^{\mathrm{p}}_j	=\mathbf{A}\vec{b}_j+\zv^{\mathrm{p}}_i \qquad \text{where }\vec{b}_j\in\{0,1\}^{2^{\Bp}},~ |\vec{b}_j|_1=T.
	\end{align*}

\begin{itemize}
\item \emph{Step1}: Get a coarse estimate of $\vec{b}_j$ via a sub-optimal CS algorithm
	\begin{itemize}
	\item $\ell_1$-regularized LASSO: 
	\begin{align*}
		\hat{\vec{b}}_j&=\min_{\vec{\beta}} |\yv_j-\mathbf{A}\vec{\beta}|^2+\lambda |\vec{\beta}|_1~\text{ where }\vec{\beta} \succeq 0
%		\text{subj. to }& \beta_i\geq 0 ~\forall i.
	\end{align*}

	\item or constrained least-squares: 
	\begin{align*}
		\hat{\vec{b}}_j &=\min_{\vec{\beta}} |\yv_j-\mathbf{A}\vec{\beta}|^2 ~\text{ where }\vec{\beta} \succeq 0
%		\text{subj. to }& \beta_i\geq 0 ~\forall i.
	\end{align*}

	\end{itemize}
\pause
\item Form a list of positive support set: $\mc{W}_{\text{list}}=\{i:\hat{\vec{b}}_j(i)>\eta_{Th}\}$
\item Output 
\[
\widehat{\mc{W}}_j^{\mathrm{p}}=\argmin_{S\subseteq\mc{W}_{\text{list}},|S|=T}||\yv_j^{\mathrm{p}}-\sum_{i\in S}\av_i||^2_2.
\]
%\item \emph{Step2}: Refine the estimate by doing ML-estimation on the list
\end{itemize}




%	where the sparse vector $\vec{b}_j\in\{0,1\}^{2^{\Bp}}$ and $|\vec{b}_j|_1=T$.

\end{frame}


%%%%%%%%%%%%----------------------------------------------------------------------------------%%%%%%%%%%%%%%%
\begin{frame}\frametitle{Decoding in a slot: Belief propagation for MAC }
\begin{itemize}
	\item Input:
	\begin{itemize}
 	\item Set of interleavers from CS decoder: $\{\pi_{f(\wpdash_i)}: i\in\mc{N}_j\}$
	\item  $\yv^{\mathrm{c}}_j=\sum_{i\in\mc{N}_j} \pi_{f(\wpdash_2)}(\vec{c}_{\wc_2}) +\zv^{\mathrm{c}}_j$
	\end{itemize}
	\item<2-> Joint Tanner graph of LDPC code for $|\mc{N}_j|$:
\end{itemize}
\onslide<2->{
	\centering
	\vspace{2ex}
	\resizebox{0.55\textwidth}{!}{\input{\mac_figpath/decodergraph_permutation}}
}
\end{frame}

%%%%%%%%%%%%----------------------------------------------------------------------------------%%%%%%%%%%%%%%%
\begin{frame}\frametitle{Decoding in a slot: Belief propagation for MAC }
\begin{itemize}
\item Message passing rules at bit/check nodes identical to single user
\onslide<2->{
\item Message passing rule at GMAC node:
\begin{align*}
v^{1}_{\text{MAC},i}&=h(u^{2}_{i,\text{MAC}},y_{i,\text{ch}})\\
v^{2}_{i,\text{MAC}}&=h(u^{1}_{i,\text{MAC}},y_{i,\text{ch}}) ~~~\text{ where}\notag\\
h(l,y)&=\log \frac{1+e^{l}e^{2(y-1)/\sigma^2}}{e^{l}+e^{-2(y+1)/\sigma^2}}
\end{align*}
}
\end{itemize}

	\centering
	\vspace{2ex}
	\resizebox{0.85\textwidth}{!}{\begin{tikzpicture}
\def \depth{1.5}; %vertical gap between nodes/levels
\def \xgap{2.8}
\def \gap{0.6}; %Horizontal gap between nodes
\def \gapA{0.3}; %Encoder Width
\def \textoffs{0.24}; %Offset for writing text above a node
\def\nodewidth{0.5};
\tikzstyle{bitwhite} = [circle, draw, thick, fill=white, radius=0.5*\nodewidth]
\tikzstyle{bitshaded} = [circle, draw, thick, fill=gray, radius=0.5*\nodewidth]
\tikzstyle{checkwhite} = [rectangle, draw, thick, fill=white,minimum height=\nodewidth, minimum width=\nodewidth]
\tikzstyle{checkshaded} = [rectangle, draw, thick, fill=gray,minimum height=\nodewidth, minimum width=\nodewidth]
                          
\def \fsize{\footnotesize}; %Defining a generic font size to be adjusted depending on the scaling
\def \fsizea{\tiny}; %Defining a generic font size to be adjusted depending on the scaling
\def \dotsize{\Huge}; %Defining a generic font size to be adjusted depending on the scaling

%%---------------------- Graph 1---------------------------------------------------------------------------
\node [bitshaded](b1) at (0,0) {\fsizea $i$} ;
\node [bitwhite,radius=0.8*\nodewidth](bMAC) at (0,1) {\fsize $+$} ;
\draw[thick,decoration={markings,mark=at position 0.5 with {\arrow{>}}},postaction={decorate}] (b1.north) -- node[midway,left]{{\fsize $u_{i,\text{MAC}}$}} node[midway,right] {\fsize $=\sum\limits_{j=1}^{3}v_{j,i}$} (bMAC.south);

\foreach \i in {1,2,3}
{
\path (b1) -- +(180+\i*45:\depth) node[checkwhite] (c1\i) {\fsize $\i$};
\begin{scope}[decoration={ markings,    mark=at position 0.5 with {\arrow{>}}} ] 
\draw[postaction=decorate] (c1\i.north) -- node[pos=0.3,left]{{\fsize $v_{\i,i}$}} (b1);
\end{scope}
}

%%---------------------- Graph 2---------------------------------------------------------------------------
\node [bitshaded,radius=0.5*\nodewidth](bMAC2) at (\xgap,0) {\fsize $+$} ;
\node [bitwhite](b2) at (\xgap,1) {\fsizea $i$} ;
%\node () at (b2.east) {\fsizea User 2} ;
\draw[decoration={markings,mark=at position 0.5 with {\arrow{>}}},postaction={decorate}] (b2.south) -- node[midway,left]{{\fsize $u^{2}_{i,\text{MAC}}$}} (bMAC2.north);
\draw[decoration={markings,mark=at position 0.5 with {\arrow{<}}},postaction={decorate}] (bMAC2.west) -- node[midway,below]{{\fsize $y_{i,\text{ch}}$}} +(-0.25*\xgap,0);


\path (bMAC2) -- +(180+90:\depth) node[bitwhite] (b21) {\fsizea $i$};
\draw[thick,decoration={ markings, mark=at position 0.5 with {\arrow{<}}}, postaction=decorate] (b21.north) -- node[midway,left]{{\fsize $v^{1}_{\text{MAC},i}$}} (bMAC2);


%%---------------------- Graph 3---------------------------------------------------------------------------
\node [bitshaded](b3) at (2*\xgap,0) {\fsizea $i$} ;
\node [bitwhite,radius=0.8*\nodewidth](bMAC3) at (2*\xgap,1) {\fsize $+$} ;
\draw[decoration={markings,mark=at position 0.5 with {\arrow{<}}},postaction={decorate}] (b3.north) -- node[midway,left]{{\fsize $v_{\text{MAC},i}$}} (bMAC3.south);

\foreach \i in {1,2,3}
{
\path (b3) -- +(180+\i*45:\depth) node[checkwhite] (c3\i) {\fsize $\i$};
}
\foreach \i in {1,2}
{
\draw[decoration={ markings, mark=at position 0.5 with {\arrow{>}}}, postaction=decorate] (c3\i.north) -- node[pos=0.3,left]{{\fsize $v_{\i,i}$}} (b3.south);
}
\draw[thick,decoration={ markings, mark=at position 0.6 with {\arrow{<}}}, postaction=decorate] (c33.north) -- node[midway,right]{{\fsize $u_{i,3}=v_{\text{MAC},i}+\sum\limits_{j=1}^{2}v_{j,i}$}} (b3.south);

\end{tikzpicture}}
\end{frame}


%\section{Compressed Sensing: Support recovery}
\subsection{Main result: Comparison with known limits}
\begin{frame}{Compressed Sensing}
\begin{equation*}
\mathbf{y=Ax +w}
\end{equation*}


\begin{itemize}
\item $\mbf{x}$ -$N \times 1$ sparse signal
\item $\mbf{A}$ -$M \times N$ measurement matrix
\item $\mbf{w}$ -additive noise
\item $\mbf{y}$ -$M \times 1$ measurement vector
\onslide<2->
\item $\text{supp}(\mbf{x})\coleq \{i: x_i\neq 0, i\in [N]\}$
\item $K=\card{\text{supp}(\mbf{x})}$
\item Sparsity- $K\ll N$ 
\end{itemize}
\end{frame}

%-------------------------------------------25@#$@#$^%$##$%^%$^%^_----------------------------------------------------------------
\begin{frame}{Problem Statement}
\begin{itemize}
\item Decoder: Given $\mbf{y}$ reconstruct the vector $\mbf{x}$ denoted by $\widehat{\mbf{x}}$
\item Prob. of failure of support recovery $\mbb{P}_{F}\coleq \text{Pr}(\text{supp}(\widehat{\mbf{x}})\neq \text{supp}(\mbf{x}))$
\item Metrics of interest:
\begin{itemize}
\item Sample complexity ($M$)
\item Decoding complexity
\item $\mbb{P}_{F}$
\end{itemize} 
\end{itemize}
\vspace{5ex}

\onslide<2->
\begin{block}{Objective}
Devise a scheme with minimizing num. of measurements $M$ and decoding complexity such that $\mbb{P}_{F}\rightarrow 0$ as $N (\text{and } K) \rightarrow \infty$
\end{block}
\end{frame}
%-------------------------------------------25@#$@#$^%$##$%^%$^%^_----------------------------------------------------------------
%\subsection{Known Limits}
\begin{frame}
\begin{block}{Optimal order for Support Recovery [1]}
\begin{itemize}
\item In the sub-linear sparsity regime, $K=o(N)$, necessary and sufficient conditions are shown to be:
\begin{equation*}
C_1 K\log\left(\frac{N}{K}\right)<M<C_2 K\log\left(\frac{N}{K}\right)
\end{equation*}
\item In the linear sparsity regime, $K=\alpha N,$ it was shown that $M=\Theta(N)$ measurements are sufficient for asymptotically reliable recovery. 
 \end{itemize}
\end{block}
\vspace{7ex}

\onslide<2->{
\begin{itemize}
\item In [1], the minimum value of the signal space affects the bounds on $M$
\end{itemize}
\begin{align*}
x_i\in\mc{X}&\defeq\{A e^{i\theta}: A\in \mc{A},\theta\in \Omega\}\cup \{0\} ,\\
\mc{A}&=\{A_{\min}+\rho l\}_{l=0}^{L_1}, \Omega =\left\lbrace 2\pi l/L_2\right\rbrace_{l=0}^{L_2}
\end{align*}
}

[1] Information Theoretic Limits of Support Recovery- Wainwright-2007
\end{frame}
%-------------------------------------------25@#$@#$^%$##$%^%$^%^_----------------------------------------------------------------

%\subsection{Main Result}
\begin{frame}{Main result}

\begin{block}{Sub-linear Sparsity-Optimal Sample and Decoding Complexities}
In the sub-linear sparsity regime, for a given SNR of $\frac{A^{2}_{\text{min}}}{\sigma^{2}}$, our scheme has 
\begin{itemize}
\item Sample complexity of $M=c_1 K\log (\frac{c_2 N}{K})$
\item Decoding complexity of $O\left(K\log(\frac{N}{K})\right)$ 
\item $\mbb{P}_{\text{F}}\rightarrow 0$ asymptotically in $K$
\end{itemize} 
where the constants $c_{1}$ and $c_{2}$ are dependent on SNR, desired rate of decay of $\mbb{P}_{\text{F}}$ and left degree $\ell$.
\end{block}
\onslide<2->
\begin{block}{Linear Sparsity Regime}
In the linear sparsity regime our scheme has 
\begin{itemize}
\item Sample complexity of $M=c_3 K\log K$
\item Decoding complexity of $O\left(K\log(K)\right)$ 
\item $\mbb{P}_{\text{F}}\rightarrow 0$ asymptotically in $K$
\end{itemize} 
where the constant $c_{3}>1$ is a parameter dependent on left degree $\ell$.
\end{block}
\end{frame}
%-------------------------------------------25@#$@#$^%$##$%^%$^%^_----------------------------------------------------------------

%\subsection{Prior Work}
\begin{frame}\frametitle{Prior Work}
\begin{itemize}
\item Zhang and Pfister, ``Verification Decoding of High-Rate LDPC Codes With Applications in Compressed Sensing", 2008
\item  Jafarpour, Xu, Hassibi and Calderbank, ``Efficient and robust compressed sensing using optimized expander graphs", 2009
\begin{itemize}
\item Sample complexity of $O(K\log N)$ and decoding complexity of $O(K)$ for noiseless setting
\end{itemize}
\item Dimakis, Smarandache and Vontobel, ``LDPC codes for Compressed Sensing", 2011

\vspace{3ex}
\onslide<2->
\item Li, Pedarsani and Ramchandran, ``Sub-linear compressed sensing for support recovery using sparse-graph codes", 2014[LPR14]
\begin{itemize}
\item Introduced sparse-graph codes peeling decoder framework to CS
\item Sample and measurement complexities of $O(K\log N)$ for noisy setting
\item Sample and measurement complexities of $2K$ and $O(K)$ for noiseless setting
\end{itemize}
\end{itemize}
\end{frame}
%-------------------------------------------25@#$@#$^%$##$%^%$^%^_----------------------------------------------------------------

\subsection{Design scheme}
%\subsection{Sensing Matrix}
\begin{frame}{Graphical Representation}
$(N,\ell,r,\mbf{S})$ ensemble. $\ell N=rM_1$. 
\onslide<2->{$\text{dim}(\mbf{S})=P\times r$.}
\begin{figure}
\scalebox{1.3}{\begin{tikzpicture}
%\clip(-0.2in,0) rectangle (1.1in,0.28in) ;
\def\horzgap{0.75in}; %Horizontal gap between nodes/levels
\def \gapVN{0.3in}; %vertical gap between nodes
\def \gapCN{0.4in}; %Horizontal gap between nodes

\def \textoffs{0.12in}; %Offset for writing text above a node
\def\nodewidth{0.05in};
\def\nodewidthA{0.1in};
\def\edgewidth{2pt};
\def\ext{0.2in};

\def \n {8};
\def\ldeg{3};
\def \m {4};
\def\rdeg{6};
\def\langle{40};%120 degrees/3
\def\langle{20};%120 degrees/6

\tikzstyle{check} = [rectangle, draw,  inner sep=0mm, fill=black, minimum height=\nodewidthA, minimum width=\nodewidthA]
\tikzstyle{checksm} = [rectangle, draw, inner sep=0mm, fill=blue,minimum height=\edgewidth,minimum width=\edgewidth]

\tikzstyle{bit} = [circle, draw, inner sep=0mm, fill=red, minimum size=\nodewidthA]
\tikzstyle{bitsm} = [circle, draw, inner sep=0mm,fill=red, minimum size=\edgewidth]
\tikzstyle{edgesock} = [circle, inner sep=0mm, minimum size=\edgewidth,draw, fill=white]     

                          
\foreach \vn in {2,3,6,7}{
 \node[bit] (vn\vn) at (0,\vn*\gapVN) {};
\path (vn\vn) ++(30:\ext) node (evA\vn) [edgesock] {};
\path (vn\vn) ++(0:\ext) node (evB\vn) [edgesock] {};
\path (vn\vn) ++(-30:\ext) node (evC\vn) [edgesock] {};

  \draw (vn\vn) -- (evA\vn.west); 
  \draw (vn\vn) -- (evB\vn.west); 
  \draw (vn\vn) -- (evC\vn.west); 
}
\path (vn3)--node(vndots) {\Large{$\vdots$}} (vn6);
\node[left =\nodewidth of vndots](){\tiny{$N$ var nodes}};

\foreach \cn in {2,4}{
\node[check] (cn\cn) at (0.8*\horzgap,0.2in+\cn*\gapCN) {};

\path (cn\cn) ++(150:\ext) node (ecA\cn) [edgesock] {};
\path (cn\cn) ++(170:\ext) node (ecB\cn) [edgesock] {};
\path (cn\cn) ++(190:\ext) node (ecC\cn) [edgesock] {};
\path (cn\cn) ++(210:\ext) node (ecD\cn) [edgesock] {};

  \draw (cn\cn) -- (ecA\cn.east); 
  \draw (cn\cn) -- (ecB\cn);  
  \draw (cn\cn) -- (ecC\cn); 
    \draw (cn\cn) -- (ecD\cn); 
}
\path (cn2)--node(cndots) {\Large{$\vdots$}} (cn4);
\node [right=0.01*\nodewidth of cndots]{\tiny{$m_{1}$ bin nodes}};

\node[draw,minimum width=\horzgap-2*\ext,minimum height=6.5*\gapVN,thick](perm) at (0.4*\horzgap,4.5*\gapVN){\Large{$\pi$}};

\def\moveX {2.5*\nodewidth};
\onslide<2->
\foreach \cn in {2,4}{
\path (cn\cn) ++(\moveX,1.7*\nodewidth) node (bitnA\cn) [bitsm] {};
\path (cn\cn) ++(\moveX,0.6*\nodewidth) node (bitnB\cn) [bitsm] {};
\path (cn\cn) ++(\moveX,-0.6*\nodewidth) node (bitnC\cn) [bitsm] {};
\path (cn\cn) ++(\moveX,-1.7*\nodewidth) node (bitnD\cn) [bitsm] {};

\path (bitnB\cn) ++(\moveX,0) node (checknB\cn) [checksm] {};
\path (bitnC\cn) ++(\moveX,0) node (checknC\cn) [checksm] {};

\draw (bitnA\cn.east)--(checknB\cn.west);
\draw (bitnC\cn.east)--(checknB\cn.west);
\draw (bitnB\cn.east)--(checknC\cn.west);
\draw (bitnD\cn.east)--(checknC\cn.west);
\path(bitnD\cn)++(0.6*\moveX,-0.3*\moveX) node(){\tiny{$\mbf{S}$}};
}

\node [right=0.3*\nodewidth of checknC2]{\tiny{$c\log r$ test nodes}};
\onslide<3->
\node [below=5*\nodewidth of bitnD2,anchor=west]{\footnotesize{$m=m_1 \times c\log r$ } \tiny{tests}};

\end{tikzpicture}}
\end{figure}
\end{frame}
%-------------------------------------------25@#$@#$^%$##$%^%$^%^_----------------------------------------------------------------

\begin{frame}{Matrix Representation}
$(N,\ell,r,\mbf{S})$ ensemble. 
\begin{itemize}
\item $\mbf{H}$ be the adjacency matrix (binning operation)- $M_1 \times N$
\item $\mbf{S}$ be the bin-detection matrix at each bin - $P \times r$
\end{itemize}
\begin{align*}
\mbf{\tilde{y}=H(x)}&= 
\begin{bmatrix}
   \tilde{\mbf{ y}}_{1} \\
   \tilde{ \mbf{y}}_{2} \\
    \vdots \\
   \tilde{\mbf{y}}_{M_1}
\end{bmatrix}
,\text{dim}(\tilde{\mbf{y}}_{i} )=r \times 1,
\\
\vspace{10pt}
\mbf{y}&= 
\begin{bmatrix}
   \mbf{ y}_{1}\\
    \mbf{y}_{2}  \\
    \vdots \\
    \mbf{y}_{M_1}
\end{bmatrix}
, \text{where } \mbf{y}_i= \mbf{S \tilde{y}}_{i}, \text{dim} (\mbf{y}_i)=P \times 1   
\end{align*}
\begin{itemize}
\item We define a tensor operation such that 
\begin{equation*}
\mbf{y=(S\boxplus H)  x}
\end{equation*}
\end{itemize}
\end{frame}
%-------------------------------------------25@#$@#$^%$##$%^%$^%^_----------------------------------------------------------------

\begin{frame}{Tensor Operation}
\begin{itemize}
\item Sensing matrix $\mbf{A}_{M_{1}P\times N}= S_{P\times r}\boxplus H_{M_{1}\times N}$ where
\vspace{2ex}
\onslide<2->
\item $\forall i\in [1:M_1]$, define a $P\times N$ matrix
\begin{equation*}
\mbf{S}_i=\mbf{h}_i \boxtimes \mbf{S}\defeq [\mbf{0},\ldots,\mbf{0},\mbf{s}_1,\mbf{0},\ldots,\mbf{s}_2,\ldots,\mbf{0},\mbf{s}_r,\mbf{0}]
\end{equation*}
where the $r$ columns are placed in the $r$ non-zero indices of $\mbf{h}_i$.
\vspace{2ex}
\item $S\boxplus H=\begin{bmatrix}
\mbf{S}_1\\
\mbf{S}_2\\
\vdots\\
\mbf{S}_{M_1}
\end{bmatrix}
$
\end{itemize}
\end{frame}
%-------------------------------------------25@#$@#$^%$##$%^%$^%^_----------------------------------------------------------------

\begin{frame}{Example}
\begin{columns}
\begin{column}{0.5\textwidth}
\[\mbf{H} = \begin{bmatrix}
1 & 0 & 0 & 1 & 0 & 1    \\
0 & 1 & 1 & 0 & 1 & 0 \\
1 & 1 & 0 & 1 & 0 & 0 \\
0 & 0 & 1 & 0 & 1 & 1
\end{bmatrix} \] 
\onslide<2->{
and 
\[ \mbf{S} = \begin{bmatrix}
+1 & -1 & -1\\
-1 & +1 & -1
\end{bmatrix}. \]}
\onslide<3->{
Sensing matrix $\bf A$ with $M = 8$: 
\[ \mbf{A = H \boxplus S} \ = \begin{bmatrix}
+1 & \ \ 0 & \ \ 0 & -1 & \ \ 0 & -1\\
-1 & \ \ 0 & \ \ 0 & +1 & \ \ 0 & -1\\
\ \ 0 & +1 & -1 & \ \ 0 & -1 & \ \ 0\\
\ \ 0 & -1 & +1 & \ \ 0 & -1 & \ \ 0\\
+1 & -1 & \ \ 0 & -1 & \ \ 0 &\ \ 0\\
-1 & +1 & \ \ 0 & -1 & \ \ 0 &\ \ 0\\
\ \ 0 & \ \ 0 & +1 & \ \ 0 & -1 & -1\\
\ \ 0 & \ \ 0 & -1 & \ \ 0 & +1 & -1
\end{bmatrix}
 \]
 }
\end{column}
\begin{column}{0.48\textwidth}
\only<1-2>{
\begin{figure}
\scalebox{3}{\begin{tikzpicture}
\def\horzgap{0.25in}; %Horizontal gap between nodes/levels
\def \gapVN{0.15in}; %vertical gap between nodes
\def \gapCN{0.2in}; %Horizontal gap between nodes



\def \textoffs{0.12in}; %Offset for writing text above a node
\def\nodewidth{0.05in};
\def\nodewidthA{0.05in};
\def \edgewidth{0.02in};
\def\ext{0.2in};

\def \n {8};
\def\ldeg{3};
\def \m {4};
\def\rdeg{6};
\def\langle{40};%120 degrees/3
\def\langle{20};%120 degrees/6

\tikzstyle{check} = [rectangle, draw,line width=0.05mm,  inner sep=0mm, fill=black, minimum height=\nodewidthA, minimum width=\nodewidthA]
\tikzstyle{checksm} = [rectangle, draw, line width=0.05mm, inner sep=0mm, fill=blue,minimum height=\edgewidth,minimum width=\edgewidth]

\tikzstyle{bit} = [circle, draw,line width=0.05mm, inner sep=0mm, fill=red, minimum size=\nodewidthA]
\tikzstyle{bitsm} = [circle, draw, very thin, inner sep=0mm,fill=red, minimum size=\edgewidth]
\tikzstyle{edgesock} = [circle, inner sep=0mm, minimum size=\edgewidth,draw, fill=white]     

                          
\foreach \vn in {1,...,6}{
 \node[bit] (vn\vn) at (0,\vn*\gapVN) {};
}

\foreach \cn in {1,...,4}{
\node[check] (cn\cn) at (\horzgap,\cn*\gapCN) {};
}

\draw[line width=0.05mm] (vn6.east)--(cn4.west);
\draw[line width=0.05mm] (vn3.east)--(cn4.west);
\draw[line width=0.05mm] (vn1.east)--(cn4.west);

\draw[line width=0.05mm] (vn2.east)--(cn3.west);
\draw[line width=0.05mm] (vn4.east)--(cn3.west);
\draw[line width=0.05mm] (vn5.east)--(cn3.west);

\draw[line width=0.05mm] (vn2.east)--(cn3.west);
\draw[line width=0.05mm] (vn4.east)--(cn3.west);
\draw[line width=0.05mm] (vn5.east)--(cn3.west);

\draw[line width=0.05mm] (vn6.east)--(cn2.west);
\draw[line width=0.05mm] (vn5.east)--(cn2.west);
\draw[line width=0.05mm] (vn3.east)--(cn2.west);

\draw[line width=0.05mm] (vn1.east)--(cn1.west);
\draw[line width=0.05mm] (vn2.east)--(cn1.west);
\draw[line width=0.05mm] (vn4.east)--(cn1.west);

\onslide<2->
{
\def\moveX {1.5*\nodewidth};
\foreach \cn in {1,2,3,4}{
\path (cn\cn) ++(\moveX,\nodewidth) node (bitnA\cn) [bitsm] {};
\path (cn\cn) ++(\moveX,0*\nodewidth) node (bitnB\cn) [bitsm] {};
\path (cn\cn) ++(\moveX,-1*\nodewidth) node (bitnC\cn) [bitsm] {};

\def\moveXA {2*\nodewidth};
\path (bitnB\cn) ++(\moveXA,0.25*\moveXA) node (checknB\cn) [checksm] {};
\path (bitnC\cn) ++(\moveXA,0.25*\moveXA) node (checknC\cn) [checksm] {};

\draw[very thin] (bitnA\cn.east)--(checknB\cn.west);
\draw[very thin] (bitnA\cn.east)--(checknC\cn.west)node[pos=0.5,above,scale=0.3]{\tiny{-1}} ;
\draw[very thin] (bitnB\cn.east)--(checknB\cn.west) node[pos=0.1,above,scale=0.3]{\tiny{-1}};
\draw[very thin] (bitnB\cn.east)--(checknC\cn.west);
\draw[very thin] (bitnC\cn.east)--(checknB\cn.west) node[pos=0.5,below,scale=0.3]{\tiny{-1}};
\draw[very thin] (bitnC\cn.east)--(checknC\cn.west) node[pos=0.5,below,scale=0.3]{\tiny{-1}};
}

}%onslide<2->
\end{tikzpicture}

%
}
\end{figure}
}
\only<3>{
\begin{figure}
\scalebox{3}{\begin{tikzpicture}
\def\horzgap{0.25in}; %Horizontal gap between nodes/levels
\def \gapVN{0.15in}; %vertical gap between nodes
\def \gapCN{0.1in}; %Horizontal gap between nodes

\def \textoffs{0.12in}; %Offset for writing text above a node
\def\nodewidth{0.05in};
\def\nodewidthA{0.05in};
\def \edgewidth{0.05in};
\def\ext{0.2in};

\def \n {8};
\def\ldeg{3};
\def\rdeg{6};
\def\langle{40};%120 degrees/3
\def\langle{20};%120 degrees/6

\tikzstyle{check} = [rectangle, draw,line width=0.05mm,  inner sep=0mm, fill=black, minimum height=\nodewidthA, minimum width=\nodewidthA]
\tikzstyle{checksm} = [rectangle, draw=none, inner sep=0mm, fill=blue,minimum height=\edgewidth,minimum width=\edgewidth]

\tikzstyle{bit} = [circle, draw,line width=0.05mm, inner sep=0mm, fill=red, minimum size=\nodewidthA]
\tikzstyle{bitsm} = [circle, draw, very thin, inner sep=0mm,fill=red, minimum size=\edgewidth]
\tikzstyle{edgesock} = [circle, inner sep=0mm, minimum size=\edgewidth,draw, fill=white]     

                          
\foreach \vn in {1,...,6}{
 \node[bit] (vn\vn) at (0,\vn*\gapVN) {};
}

\foreach \cn in {1,...,8}{
\node[checksm] (cn\cn) at (\horzgap,\cn*\gapCN) {};
}

\draw[line width=0.05mm] (vn6.east)--(cn4.west);
\draw[line width=0.05mm] (vn3.east)--(cn4.west);
\draw[line width=0.05mm] (vn1.east)--(cn4.west);

\draw[line width=0.05mm] (vn2.east)--(cn3.west);
\draw[line width=0.05mm] (vn4.east)--(cn3.west);
\draw[line width=0.05mm] (vn5.east)--(cn3.west);

\draw[line width=0.05mm] (vn2.east)--(cn3.west);
\draw[line width=0.05mm] (vn4.east)--(cn3.west);
\draw[line width=0.05mm] (vn5.east)--(cn3.west);

\draw[line width=0.05mm] (vn6.east)--(cn2.west);
\draw[line width=0.05mm] (vn5.east)--(cn2.west);
\draw[line width=0.05mm] (vn3.east)--(cn2.west);

\draw[line width=0.05mm] (vn1.east)--(cn1.west);
\draw[line width=0.05mm] (vn2.east)--(cn1.west);
\draw[line width=0.05mm] (vn4.east)--(cn1.west);

\end{tikzpicture}}
\end{figure}
}
\end{column}
\end{columns}
\end{frame}
%-------------------------------------------25@#$@#$^%$##$%^%$^%^_----------------------------------------------------------------

%\subsection{Decoding}
\begin{frame}{Bin Decoding}
At each bin, input to the decoder is $\mbf{y}_i=\sum_{j=1}^{r}x_{\mbf{h}_{i}^{j}}\mbf{s}_j+\mbf{w}_i$
\begin{itemize}
\onslide<2->
\item Zero-ton: Is it just noise?
\begin{equation*}
\widehat{\mc{H}}_i=\mc{H}_{Z}, ~~\text{if } \frac{1}{P}\norm{\mbf{y}_i}^2\leq (1+\gamma)\sigma^2
\end{equation*}
\onslide<3->
\item Singleton: If a single variable is non-zero? $\mbf{y}_i=x_j\mbf{s}_j+\mbf{w}_i$
\begin{align*}
\alpha_{k}&=\frac{\mbf{s}_k^{\dagger}\mbf{y}_i}{\norm{\mbf{s}_k}^2}\\
\hat{k}&=\arg \min_{k}~~ \norm{\mbf{y}_i-\alpha_{k}s_k}\\
\hat{x}[\hat{k}]&=\arg \min_{x\in\mc{X}} \norm{x-\alpha_{\hat{k}}}
\end{align*}
\onslide<4->
\item Multi-ton: More than one non-zero variable?
\begin{equation*}
\widehat{\mc{H}}_i=\mc{H}_{S}(\hat{k},\hat{x}[\hat{k}]), ~~\text{if } \frac{1}{P}\norm{\mbf{y}_{i}-\hat{x}[\hat{k}]\mbf{s}_{\hat{k}}}^2\leq (1+\gamma)\sigma^2
\end{equation*}
\end{itemize}
\end{frame}
%-------------------------------------------25@#$@#$^%$##$%^%$^%^_----------------------------------------------------------------

\begin{frame}{Peeling Decoding}
\begin{columns}
\column{0.55\textwidth}
%\begin{algorithmic}
%\While {$\exists i\in[M_1]: \mc{H}_i=\mc{H}_Z ~\text{or } \mc{H}_S $, }
%\If {$\mc{H}_i=\mc{H}_Z$}
%    \State Remove the bin $i$\\
%   \hspace{2ex} Assign $0$ to all the variables connected
%\ElsIf {$\mc{H}_i=\mc{H}_S(k,x[k])$}   
%   \hspace{4ex} Assign $x[k]$ to $k^{\text{th}}$ variable in bin $i$\\
%  \hspace{2ex} Subtract $x[k]\mbf{s}_k$ from connected $\mbf{y}_i$  \\
%  \hspace{2ex} Remove the bin and all variables connected
%\EndIf
%\EndWhile
%\end{algorithmic}


\column{0.45\textwidth}
\begin{figure}
\scalebox{5}{\input{\cs_figpath/PeelingAnimation}}
\end{figure}

\end{columns}
\end{frame}
%-------------------------------------------25@#$@#$^%$##$%^%$^%^_----------------------------------------------------------------

\subsection{Analysis}
%\subsection{Peeling Decoder}
\begin{frame}{Oracle based Peeling Decoder}
\begin{itemize}
\item Assume the hypothesis detection in each bin decoder is correct
\item Equivalence to peeling decoder on pruned graph- all zero variables are removed
\end{itemize}
\begin{block}{Equivalence to $(N,l,r)$ LDPC on BEC($\epsilon=\frac{K}{N}$)}
If $\text{supp}(\mbf{x})=\{i:y_i=\mc{E}\}$, then $P^{(i)}_{\text{BEC}}(\mbf{y})=P^{(i)}_{\text{SR}}(\mbf{z})$  for $\mbf{z=Hx}$.
\end{block}
\onslide<2->
\begin{itemize}
\item Choose $M_1=\eta K$ thus $r=\frac{\ell N}{\eta K}$
\end{itemize}
\vspace{1ex}
\begin{block}{DE for Peeling decoder on LDPC -BEC channel}
Fractional number of degree one checks remaining
\begin{equation*}
\tilde{R}_1(y)=r\epsilon y^{l-1}[y-1+(1-\epsilon y^{l-1})^{r-1}]
\end{equation*}
where $\epsilon=\frac{K}{N}$ and $r=\frac{\ell N}{\eta K}$
\end{block}
\end{frame}
%-------------------------------------------25@#$@#$^%$##$%^%$^%^_----------------------------------------------------------------

\begin{frame}{}
\begin{block}{Peeling threshold}
$\eta^{\text{Th}}$ is defined to be the minimum value of $\eta$ for which there is no non-zero solution for the equation:
\begin{align*}
y&=\lim_{\frac{N}{K}\rightarrow\infty}1-\left(1-\frac{Ky^{\ell-1}}{N}\right)^{\frac{\ell N}{\eta K}}\\
  &=1-e^{\frac{-\ell y^{\ell-1}}{\eta}}
\end{align*}
in the range $y\in [0,1]$.
\end{block}
\vspace{3ex}
\onslide<2->
\begin{block}{Threshold behavior}
For $M_1>\eta^{\text{Th}}K$ bin nodes, the peeling decoder will be successful with probability $1-O\left(\frac{1}{K^{\ell-2}}\right)$
\end{block}
Note that $\eta^{\text{Th}}$ is a function of just the left degree $\ell$.
\end{frame}
%-------------------------------------------25@#$@#$^%$##$%^%$^%^_------------------------------------------------------------------

%\subsection{Bin Decoder}
\begin{frame}{Bin detection matrix}
\begin{itemize}
\item Singleton detection is the crucial part of bin decoding:
\begin{equation*}
\mbf{y}_i=x_{k}\mbf{s}_k +\mbf{w}_i
\end{equation*} 
\item Error correction coding: $\mbf{S}$ be the codebook, where each $\mbf{s}_i$ is a codeword.
\item Block length =$P$.  $\#$ codewords $\geq \frac{N\ell}{\eta K}$
\item Choose a code with rate $R(\beta)$ s.t. fractional minimum distance $\beta$ s.t. $\frac{\beta}{2} >\mbb{P}_{e}\coleq e^{-\frac{A_{\text{min}}^{2}}{2\sigma^2}}$
\item Thus $P=\frac{\lceil {\log_2(\frac{N\ell}{\eta K})}\rceil}{R(\beta)}$.
\end{itemize} 

\begin{block}{Sample Complexity}
 \begin{align*}
  M&=M_1\times P \\
   &\geq \left[\frac{\eta^{\text{Th}}}{R(2\mbb{P}_{e})}\right] K\log\left(\frac{\ell N}{\eta^{\text{Th}} K}\right)
\end{align*} 
\end{block}
\end{frame}
%-------------------------------------------25@#$@#$^%$##$%^%$^%^_----------------------------------------------------------------

\begin{frame}{Analysis of Bin Decoding}
\begin{itemize}
\item Let $\text{E}_{\text{bin}}$ be the event an error was made in overall bin decoding
\item Union bounding: $\text{E}_{\text{bin}}\leq (\eta K+\ell K)\text{Pr(E)}$
\end{itemize}
\onslide<2->
\begin{block}{Error Probability of a bin [LPR14]}
$\text{Pr(E)}\leq 3e^{-\frac{P}{4}\frac{\gamma^2}{1+4 \gamma}}+2e^{-\frac{P}{4}(\sqrt{1+2\gamma}-1)^2}+4e^{-c_6 P\left(1-\frac{2\gamma\sigma^2}{A^{2}_{\text{min}}}\right)}+2e^{-P\frac{\left(\beta-\mbb{P}_{e}\right)^2}{2\mbb{P}_{e}(1-\mbb{P}_{e})}}$
\end{block}
\vspace{2ex}
\onslide<3->
\begin{description}
    \item[\textbf{Sub-Linear sparsity}]
\end{description}
\begin{itemize}
\item Order optimal sample complexity with precise constants given
\item $\mbb{P}_{\text{F}}\rightarrow 0$ as $N (\text{and } K)\rightarrow \infty$
\item Trade-off between the constants in $M$, rate of decay of  $\mbb{P}_{\text{F}}$ and SNR
\item Optimal decoding complexity of $O\left(K\log\left(\frac{N}{K}\right)\right)$
\end{itemize}
\end{frame}
%-------------------------------------------25@#$@#$^%$##$%^%$^%^_----------------------------------------------------------------

\begin{frame}{Implications}
\begin{block}{Error Probability of a bin - Ramchandran \textit{et al}, 2014}
$\text{Pr(E)}\leq 3e^{-\frac{P}{4}\frac{\gamma^2}{1+4 \gamma}}+2e^{-\frac{P}{4}(\sqrt{1+2\gamma}-1)^2}+4e^{-c_6 P\left(1-\frac{2\gamma\sigma^2}{A^{2}_{\text{min}}}\right)}+2e^{-P\frac{\left(\beta-\mbb{P}_{e}\right)^2}{2\mbb{P}_{e}(1-\mbb{P}_{e})}}$
\end{block}

\vspace{2ex}
\begin{description}
    \item [\textbf{Linear sparsity: $K=\alpha N$}]
\end{description}
\begin{itemize}
\item Choice of $P=c_{1}\log\left(c_2 \frac{N}{K}\right)$ doesn't work
\item We choose $P=\log K$ and rate $R(\beta)$ as earlier
\item A sub-code of size $\frac{\ell}{\alpha\eta}$ of the codebook is chosen as $\mbf{S}$
\item Sample complexity of $\eta^{\text{Th}}K\log K$
\item Can we do $\Theta(K)$ with simple decoding?
\end{itemize}
\end{frame}
%-------------------------------------------25@#$@#$^%$##$%^%$^%^_----------------------------------------------------------------

%\section{Simulation Results}
\begin{frame}{Simulation Results}
\begin{itemize}
\item $K=50, N=10^5$. $\mc{X}=\{+1,-1\}$
\item $\ell=4, \eta =2 (M_1 =2K)$. Note that $\eta^{\text{Th}}(\ell=4)=1.295$
\item $r=\frac{N\ell}{\eta K}=4000$. $\log_2(r)=12$
\item $(12,24)$ Golay code and $(12,n)$ convolutional codes for $n=24,48,96$ with QAM form $\mbf{S}$.
\end{itemize}

\begin{figure}
\resizebox{0.5\textwidth}{!}{
\begin{centering}
\input{\cs_figpath/succ_prb_Slides}
\end{centering}
}
\end{figure}
\end{frame}
%\section{GroupTesting}
\begin{frame}\frametitle{Group Testing}
	\begin{figure}[t]
		\centering
		\includegraphics[width=1.8in]{\gt_figpath/grouptest_testubes.jpg}
	\end{figure}

\begin{itemize}
	\item II World War - detect all soldiers with syphilis
	\item Tests performed on efficiently pooled groups of items
	\item Least no. of tests ($m$) to identify $K$ defective items from $N$ items
\end{itemize}	

\end{frame}

%%%-------------------------------------------------------------------------------------------
\begin{frame}\frametitle{Group Testing}
\alert{Example}
\vspace{-0.3in}
	\begin{figure}[t]
		\centering
		\includegraphics[width=4.2in]{\gt_figpath/grouptesting_example.pdf}
	\end{figure}
\end{frame}


%%----------------------------------------------------------------------------------
\begin{frame} \frametitle{Group Testing}
\begin{figure}[t]
\centering
\includegraphics[width=3.4in]{\gt_figpath/A_times_X_group_testing.pdf}
\end{figure}

\begin{block}
	{
		\[ \small \underset{\color{blue}(Observation \ vector)}{Y_{m \times 1}} = A \odot X = \begin{bmatrix}
		{<a_1,X>} \\
		{<a_2,X>}  \\
		\vdots  \\
		{<a_m,X>}
		\end{bmatrix} \ \   <a_i , X> = \overset{N}{\underset{j=1 }{\vee}} a_{ij}X_j  \] }
		
\end{block}
\end{frame}

%%%------------------------------------------------------------------------------
\begin{frame}{Group Testing}
\begin{block}{Differences}
\begin{itemize}
\item Differs from CS: OR operation, non-linear
\item So the previous bin detection matrix would not work
\end{itemize}
\end{block}	

\onslide<2->			
\begin{block}{Singleton detection}			
{
 \tiny
$$
			\begin{bmatrix}
		H_1 \\
		\overline{H_1}
\end{bmatrix}
=
\begin{bmatrix}
					\bf{b_1} & \bf{b_2} & \bf{b_3} & \cdots & \bf{b_{n-1}}\\
				   	\bf{\overline{b_1}} & \bf{\overline{b_2}} & \bf{\overline{b_3}} & \cdots & \bf{\overline{b_{n-1}}} \end{bmatrix}
=
 \begin{bmatrix}
		0      & 0   & 0 & \cdots & 1 &  1 \\
		0      & 0   & 0 & \cdots & 1 &  1  \\
		\vdots & \vdots & \vdots & \ddots & \vdots & \vdots \\
		0      & 0   & 1 & \cdots & 1 &  1  \\
		0      & 1   & 0 & \cdots & 0 &  1  \\
		-- & -- & -- & -- & -- & --  \\
        1      & 1   & 1 & \cdots & 0 &  0 \\
		1      & 1   & 1 & \cdots & 0 &  0  \\
		\vdots & \vdots & \vdots & \ddots & \vdots & \vdots \\
		1      & 1   & 0 & \cdots & 0 &  0  \\
		1      & 0   & 1 & \cdots & 1 &  0  \\
\end{bmatrix}
$$
}
\alert{Note:} If a checknode is a singleton, with $i$th bit-node participating, then the observation vector is the $i$th column of $A$.

\begin{itemize}
\item Singleton - if the {\color{blue}weight of first two observation} vectors together is{ \ \color{blue} $L$}.
\item {\color{blue}Position} of the defective item is - {\color{blue} decimal value of the 1st observation vector}.
\end{itemize}    				
\end{block}					
\end{frame}

%%%---------------------------------------------------------------------------------
\begin{frame}\frametitle{Group Testing}
\vspace*{-0.1in}
  \begin{block}{Measurement matrix ($A_{m\times n}$)}
  {\centering
  $A_{m \times n}  = \underset{\color{blue} (d-left \ regular \ Graph)}{\bf G_{\frac{m}{6} \times n}} {\Large \bf \color{red} \otimes} \underset{\color{blue} (Singleton \ identifier)}{\bf H_{6 \times n}}$ \\
  \vspace{6pt}
  Let, $\bf{b_i}$ denote the {\color{blue}$L$-bits binary representation of the integer $i-1$}, $L=\lceil \log_2{n} \rceil$.
   {\small \[ H = \begin{bmatrix}
					\bf{b_1} & \bf{b_2} & \bf{b_3} & \cdots & \bf{b_{n-1}}\\
				   	\bf{\overline{b_1}} & \bf{\overline{b_2}} & \bf{\overline{b_3}} & \cdots & \bf{\overline{b_{n-1}}}\\
					\bf{b_{i_1}} & \bf{b_{i_2}} & \bf{b_{i_3}} & \cdots & \bf{b_{i_{n-1}}}\\
				   	\bf{\overline{b_{i_1}}} & \bf{\overline{b_{i_2}}} & \bf{\overline{b_{i_3}}} & \cdots & \bf{\overline{b_{i_{n-1}}}}\\
				   	
					\bf{b_{j_1}} & \bf{b_{j_2}} & \bf{b_{j_3}} & \cdots & \bf{b_{j_{n-1}}}\\
				   	\bf{\overline{b_{j_1}}} & \bf{\overline{b_{j_2}}} & \bf{\overline{b_{j_3}}} & \cdots & \bf{\overline{b_{j_{n-1}}}} \end{bmatrix} \]
$s_1=(i_1, i_2, \cdots, i_{n-1})$ and $s_2=(j_1, j_2, \cdots, j_{n-1})$ are permutations }}
 \end{block}

\begin{block}{Decoding procedure}
\begin{itemize}
\item  Identify and decodes singletons using weights of the observation vector
\item Identify and resolve doubletons by guessing to satisfy the first pair of observation vectors and checking if the guess satisfies the other two pairs of observations
\item The iteration continues until no doubletons can be resolved
\end{itemize}

\end{block}
\end{frame}

%----------------------------------------------------------------------------------------------------------------------------------
\begin{frame} \frametitle{Main results for group testing}
\begin{block}{Non-adaptive Group Testing (Noiseless and Noisy)}
\begin{itemize}
\item Recovers $(1-\epsilon)K$ items w.h.p.
\item Samples: $m = \Theta(K \log_2 \frac{N}{K})$ is order optimal
\item Computational complexity: $O(K \log \frac{N}{K})$  (order optimal)
\end{itemize}
\end{block}

\end{frame}

%----------------------------------------------------------------------------------------------------------------------------------
\begin{frame}{Conclusion}
\begin{itemize}
  \item Review of a simple message passing decoder called the peeling decoder
  \item Density evolution as a tool to analyze its asymptotic performance
  \item Applications 
    \begin{itemize}
      \item Sparse Fourier transform computation
      \item Compressed sensing type sparse recovery problems
    \end{itemize}
\end{itemize}
\end{frame}
	
%--------------------------------------------------------------------
\begin{frame}\frametitle{Questions?}
	\begin{figure}[t]
		\centering
		\includegraphics[width=2.8in]{\gt_figpath/questions}
	\end{figure}
	\centering
	\color{blue}
	\Huge{Thank you!}
\end{frame}

\end{document}
\chapter*{ABSTRACT}
\addcontentsline{toc}{chapter}{ABSTRACT} % Needs to be set to part, so the TOC doesnt add 'CHAPTER ' prefix in the TOC.

\pagestyle{plain} % No headers, just page numbers
\pagenumbering{roman} % Roman numerals
\setcounter{page}{2}

\indent The broad theme of this thesis is design of schemes that admit iterative algorithms with low computational complexity to a set of problems that admit sparsity. In particular this thesis focuses on the following two areas: (i) massive multiple access and interference channels within the scope of communication systems and (ii) compressed sensing and group testing problems within the scope of signal processing algorithms. Although bipartite Tanner graphs and low-complexity iterative algorithms such as peeling and message passing decoders are very popular in the channel coding literature \cite{richardson2008modern} they are not as widely used in the respective areas of study and this thesis serves as an important step in that direction to bridge that gap. 
%The focus of this thesis is on a class of problems that can all be characterized by the enormous scale of the total number of parameters but also being sparse in the parameters of interest. 
%, that closely resemble peeling decoder in the overall philosophy,
%to a class of problems that are enormous in scale but whose actual parameters of interest are sparse in nature. 

%In the problems of massive multiple access, compressed sensing and group testing the total number of parameters is very large, the parameters being the number of users present in the system, the length of the signal to be compressed and the number of items to be tested for defects for the respective problems. However the number of actual parameters of interest is fairly small in comparison, the parameters of interest being the number of users that are active at any given time, the number of dimensions with non-negligible signal power and the number of defective items for the respective problems. The design schemes proposed in this work attempt to explore this sparse nature of the parameters of interest. The main ingredients common to all the proposed schemes are: (i) a variant of a sparse bipartite Tanner graph is used as the basis for designing the interaction strategy between the parameters of interest (ii) an iterative algorithm designed on the basis of peeling decoder with computational complexity sub-linear in the total number of parameters is used for recovering the parameters of interest. The tools bipartite Tanner graphs and peeling decoder are very popular in the channel coding literature \cite{richardson2008modern} but are not as widely used in the respective areas of study and this thesis is an important step in that direction to bridge that gap.
The contributions of this thesis can be categorized into following three parts:

	\indent In~\cite{polyanskiy2017perspective}, Polyanskiy introduced an interesting and timely multiple access problem for a massive number of uncoordinated devices wherein the base station is interested only in recovering the list of messages without regard to the identity of the respective sources. Shortly after, Ordentlich and Polyanskiy\cite{ordentlich2017low} proposed a coding scheme with polynomial encoding and decoding complexities that involves a concatenated coding scheme for the $T$-user Gaussian multiple access channel (GMAC), typically $T$ being in the range $1-5$. For the parameters of interest in the Low-Powered Wide Area Netowrk (LPWAN) space, the authors \cite{ordentlich2017low} demonstrated that their proposed coding scheme outperforms all the previously known multiple access strategies such as treat interference as noise (TIN) and slotted-ALOHA. However there is still a significant gap of $\approx 20$ dB when compared to the the information rate achievable by a random Gaussian coding scheme with the optimal minimum mean square error decoder of exponential complexity. In this thesis a coding scheme with polynomial encoding and decoding complexities is proposed, the two main features of which are (i) design of a superior coding scheme for the $T$-user GMAC and (ii) successive interference cancellation decoder. The proposed coding scheme not only improves on the performance of the best known coding scheme in \cite{ordentlich2017low} by $\approx 13$ dB but is only $\approx 6$ dB away from the random Gaussian coding information rate.

\indent In the second part construction-D lattices are constructed where the underlying linear codes are nested binary spatially-coupled low-density parity-check codes (SC-LDPC) codes with uniform left and right degrees. By leveraging results on the optimality of spatially-coupled codes for binary input memoryless channels and Forney {\em et al.}'s earlier results on the optimality of construction-D, it is shown in this thesis that the proposed lattices achieve the Poltyrev limit under multistage belief propagation decoding. Leveraging this result lattice codes constructed from these lattices are applied to the three user symmetric interference channel. For channel gains within 0.39 dB from the very strong interference regime, the proposed lattice coding scheme with the iterative belief propagation decoder, for target error rates of $\approx 10^{-5}$, is only $2.6$ dB away the Shannon limit.
%is desired user can be decoded at signal-to-noise ratios within 1.53 dB from the Shannon limit. For practical
%check degrees, simulation results with BP decoding show a
%gap of about 2.6 dB 

\indent The third part focuses on support recovery in compressed sensing and the non-adaptive group testing (GT) problems. In \cite{li2015subdraft,lee2015saffron} sensing schemes based on left-regular sparse bipartite graphs and iterative recovery algorithms based on peeling decoder were proposed for the above problems. These schemes require $O(K \log N)$ and $ \Omega(K\log K\log N)$ measurements respectively to recover the sparse signal with high probability (\emph{w.h.p}), where $N,K$ denote the dimension and sparsity of the signal respectively ($K\ll N$). Also the number of measurements required to recover atleast $(1-\epsilon)$ fraction of defective items w.h.p (approximate GT) is shown to be $c_\epsilon K\log N$. More importantly the computational complexity of the recovery algorithm in the proposed schemes is $O(K \log N)$ which is known to be optimal.

In this thesis, instead of the left-regular bipartite graphs, left-and-right regular bipartite graph based measurement schemes are analyzed. It is shown that this design strategy achieves superior and sharper measurement complexities. For the support recovery problem, the number of measurements is reduced to the optimal lower bound of $\Omega\left(K \log \frac{N}{K} \right)$. Similarly for the approximate GT, proposed scheme only requires $\ceps K\log \frac{N}{K}$ measurements. Although the measurement complexity has the same asymptotic order $\mc{O}(K\log N)$ as that of \cite{lee2015saffron}, which is the best known order result for the approximate GT, this provides a sharper explicit upper bound of $\Theta(K\log \frac{N}{K})$. For the probabilistic GT, proposed scheme requires $\Omega(K\log K \log \frac{N}{K})$ measurements which is only $\log K$ factor away from the best known lower bound of $\Omega(K\log \frac{N}{K})$ \cite{chan2014non}. Apart from the asymptotic regime, the proposed schemes also demonstrate significant improvement in the required number of measurements for finite values of $K,N$. 

%also a significant improvement in the required number of tests 
%that instead A design scheme the sensing matrix based on left and right regular sparse bipartite graphs is proposed. An iterative algorithm similar to the peeling decoder is employed for the recovery of support of the sparse signal. The proposed design scheme enables to achieve the number of measurements and decoding complexity that match the best possible lower bounds upto a constant. , a testing scheme based on left and right regular bipartite graph is proposed. An iterative algorithm similar to the peeling decoder is employed for the detection of defective items. Under the proposed testing scheme the number of tests required is order optimal and more importantly the recovery algorithm has only a sub-linear time complexity.

%measurements and the first scheme require $O(N \log N)$ computations whereas the second scheme requires  computations (sub-linear time complexity when $K$ is sub-linear in $N)$.  The precise value of constant $c_\epsilon$ as a function of the required error floor $\epsilon$ is also given. More importantly the computational complexity of the proposed peeling based decoder is only $\mc{O}(K\log N)$.
%In \cite{li2015subisit,li2015subdraft}, two schemes have been proposed to recover the support of a $K$-sparse $N$-dimensional signal from noisy linear measurements. Both schemes use left-regular sparse-graph code based sensing matrices and a simple peeling-based decoding algorithm. Both the schemes require $O(K \log N)$ measurements and the first scheme require $O(N \log N)$ computations whereas the second scheme requires $O(K \log N)$ computations (sub-linear time complexity when $K$ is sub-linear in $N)$. We show that by replacing the left-regular ensemble with left and right regular ensemble, we can reduce the number of measurements required of these schemes to the optimal order of $O\left(K \log \frac{N}{K} \right)$ with decoding complexities of $O(K \log \frac{N}{K})$ and $O(N \log \frac{N}{K})$, respectively.
\pagebreak{}

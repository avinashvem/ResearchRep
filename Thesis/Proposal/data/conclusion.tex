In this thesis we provided solutions to some problems in massive multiple access and sparse signal recovery using tools from coding theory. However we believe that there are a wide variety of applications with huge potential in applying these coding theory tools. Below, we list some of the questions that emanate from this thesis that need to be pursued and also a few potential applications of the solution designs discussed.

\begin{itemize}
\item Consider the compressed sensing problem studied in Ch.~\ref{chap:MAC}:
\begin{align*}
\yv=\mathbf{A}\vec{\mathrm{b}}+\zv,
\end{align*}
where the non-zero elements of the $T$-sparse vector $\vec{\mathrm{b}}$ are all equal to one and the sparsity is very small, $T\in[1:10]$. This specific compressed sensing problem is not extensively studied in the literature for the non-asymptotic regime. Although we derived some new bounds on $T$-disjunctive codes as an application for the sensing matrix, a full characterization of the sensing matrix suitable for this problem warrants further study.
\item In Ch.~\ref{chap:uncoord_mac}, given a probability distribution for the repetition pattern of each user in the random multiple access problem, analytic expressions to compute the error probability of peeling decoder are derived. Based on these analytic expressions, an iterative linear programming optimization technique based on first order approximations to error probability, similar to \cite{amraoui2007find}, need to be studied. Through this approach distributions can be found which, for number of users $n=1000$, can potentially achieve values of throughput larger than the current best known value $\approx 80\%$ .
\item In Ch.~\ref{chap:SCLDPClattices} lattice construction based on nested linear spatially coupled LDPC code ensembles is proposed. It	 was shown that the proposed lattices are optimal for the unconstrained AWGN channel i.e., \emph{Poltyrev-good}. Given such 
\emph{Poltyrev-good} lattices, it was shown recently \cite{ling2014achieving,yan2014construction} that applying appropriate discrete Gaussian shaping over the lattice so that the power constraint is satisfied, the capacity of the power constrained AWGN channel can be achieved. The optimality of the low complexity multi-level decoding considered for the proposed SC-LDPC lattices in Ch.~\ref{chap:SCLDPClattices}, under the discrete Gaussian shaping needs to be studied. If this issue can be resolved affirmatively, the capacity of the three user symmetric interference channel can be achieved by the proposed lattices, overcoming the demonstrated $1.53$dB gap in Sec.~\ref{subsec:SCLDPC_IC_results}, due to hyper cube shaping.
\item In Chapters~\ref{chap:cs} \& \ref{chap:gt} we modified the earlier sensing schemes due to Ramchandran \textit{et al.,} by replacing the left-regular with left-and-right-regular bipartite graphs. This not only enabled us to derive sharper results in the asymptotic regime matching the lower bounds but also demonstrated improved performance in the non-asymptotic regime. A thorough comparison, in terms of the measurement and computational complexities, with the popular schemes in the literature for support recovery and group testing, particularly in the non-asymptotic regime needs to be undertaken.
\end{itemize}

%%%%%%%%%%%%%%%%%%%%%%%%%%%%%%%%%%%%%%%%%%%%%%%%%%%
%
%  New template code for TAMU Theses and Dissertations starting Fall 2016.  
%
%
%  Author: Sean Zachary Roberson
%  Version 3.17.06
%  Last Updated: 6/15/2017
%
%%%%%%%%%%%%%%%%%%%%%%%%%%%%%%%%%%%%%%%%%%%%%%%%%%%
%%%%%%%%%%%%%%%%%%%%%%%%%%%%%%%%%%%%%%%%%%%%%%%%%%%%%%%%%%%%%%%%%%%%%
%%                           ABSTRACT 
%%%%%%%%%%%%%%%%%%%%%%%%%%%%%%%%%%%%%%%%%%%%%%%%%%%%%%%%%%%%%%%%%%%%%

\chapter*{ABSTRACT}
\addcontentsline{toc}{chapter}{ABSTRACT} % Needs to be set to part, so the TOC doesnt add 'CHAPTER ' prefix in the TOC.

\pagestyle{plain} % No headers, just page numbers
\pagenumbering{roman} % Roman numerals
\setcounter{page}{2}

\indent The low-density parity-check (LDPC) codes, introduced by Robert G. Gallager in 1960, have been around for more than 50 years. Only in the last two decades these codes, through carefully optimized design, have been shown to achieve performance very close to the Shannon capacity, the fundamental limit of transmission over noisy channels. But not until as recently as 2011 a sub-class of LDPC codes known as spatially-coupled LDPC codes (first introduced to the literature as convolutional-LDPC codes) have been shown to achieve rates arbitrarily close to the Shannon capacity. It turns out that this spatial-coupling mechanism is a much more general phenomenon and it's application is not limited to the problem of error control coding for communications. This thesis explores this specific issue,  specifically the application of the spatial coupling mechanism for problems in the field of communications and information theory.

The contributions of this work can be broadly classified into three categories: 
\begin{enumerate}
\item Application of spatially-coupled LDPC codes to the design of lattices. This thesis shows that it is possible to construct lattices that are Poltyrev-optimal under practical decoding schemes. In fact along with polar lattices introduced in 2014, whose construction is based on polar codes, the proposed lattices are the first to demonstrate Poltyrev-optimality under practical computational complexity

\item Application of spatial-coupling mechanism to compound LDGM-LDPC codes. This thesis shows that the spatially-coupled compound codes when applied to the channel coding and source coding with side-information problems, Gelfand-Pinsker and Wyner-Zyv respectively, can achieve performance very close to the optimal limits. These results  are shown under practical decoding and encoding schemes respectively. It is also shown that the SC compound codes provide capacity-achieving performance for the write-once memory (WOM) systems.

\item Application of spatial-coupling to the big data problems with sparsity. Specifically the thesis focuses on the problems of compressed sensing, group testing and sub-string matching. For the problems of compressed sensing and group testing using spatial-coupling mechanism, optimal order for the number of measurements can be achieved. 
%As for the problem of sub-string matching with the aid of spatial-coupling 
\end{enumerate}  
\pagebreak{}

\section{Distributed processing framework}
In this section, we describe a distributed processing framework which enables parallel processing of the algorithm proposed in section ~\ref{sec:Algo_desc} and also point out its advantages.

Given a database(or string) of length $N$, we divide the database into $G=N^\gamma$ blocks each of length $\tilde{N} = N/G$. Now each block can be processed independently (in parallel) using the RSIDFT framework with the new database length reduced from $N$ to $\tilde{N}$. This distributed framework has the following advantages
\begin{itemize}
	\item Firstly, this enables parallel computing and hence can be distributed across different workstations.
	\item Improves the sample and computational complexity by a constant factor. {(\color{blue} cite the exact equations from the sample and computational complexity)}
	\item Sketch of the database needs to be computed only for a smaller block length and hence requires computation of only a shorter $\tilde{N}$ point FFT.
	\item Enables localized searches on only a few blocks if we have prior knowledge about where the query could be. Specifically, if we exactly know the number of matching locations, and having encountered the required number of positions from the already run processes (processed blocks), we can terminate rest of the processes.   
	\item Helps overcome implementation issues with memory and precision as the scale of the problem is reduced. 
\end{itemize}
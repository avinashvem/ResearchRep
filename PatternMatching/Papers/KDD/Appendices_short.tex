\appendix
%\section{Chernoff Bounds}


%\begin{lemma}[Hoeffding tail bound]% for bounded random variables]
%\label{Lem:Chernoff}
%Let $X_1, X_2,\ldots, X_n$ be a sequence of independent random variables such that $X_i$ has mean $\mu_i$ and sub-Gaussian parameter $\sigma_i$. Then for any $\delta>0$:
%\begin{align*}
%\text{\textbf{Upper Tail}}: ~&\mbb{P}\left[\sum \left(X_i-\mu_i\right)\geq \delta\right]\leq \exp\left\lbrace-\frac{\delta^2}{2\sum \sigma_i^2}\right\rbrace\\
%\textbf{Lower Tail}: ~&\mbb{P}\left[\sum \left(X_i-\mu_i\right)\leq -\delta\right]\leq \exp\left\lbrace-\frac{\delta^2}{2\sum \sigma_i^2}\right\rbrace
%\end{align*}
%Note that for bounded random variables $X_i\in [a,b]$ the sub-Gaussian parameter is $\sigma_i=\frac{b-a}{2}$ whereupon the upper tail Hoeffding bound can be simplified to
%\begin{align}
%%\text{\textbf{Upper Tail}}: ~&
%\mbb{P}\left[\sum_{i=1}^{n} \left(X_i-\mu_i\right)\geq \delta\right]\leq \exp\left\lbrace-\frac{2\delta^2}{n(b-a)^2}\right\rbrace.
%%\textbf{Lower Tail}: ~&\mbb{P}\left[\sum_{i=1}^{n} \left(X_i-\mu_i\right)\leq -\delta\right]\leq \exp\left\lbrace-\frac{2\delta^2}{n\sum(b-a)^2}\right\rbrace
%\label{Eqn:HoeffdingBoundedRV}
%\end{align}
%Similarly the lower tail bound can be simplified.
%\end{lemma}


%\begin{lemma}[Tail bounds for noise terms]
%	\label{Lem:tailbounds}
%Let us consider $r[\theta_0],r[\theta_1],\ldots ,r[\theta_{g_i-1}]$ where $\theta_j=\theta_0+jf_i$ and $\theta_j \notin \{\tau_1,\ldots, \tau_L\}$ is not one of the matching positions for any $j$. Then for any $\delta>0$:\\
%{\bf Upper Tail:}
%\begin{align*}
%\mbb{P}\left[ \left(\frac{1}{M}\sum\limits_{j\in[g_i]}\sum\limits_{k\in[M]}x[\theta_j+k]y[k]\right)\geq \delta\right]\leq \exp\left\lbrace-\frac{M\delta^2}{2g_i}\right\rbrace
%\end{align*}
%{\bf Lower Tail:}
%\begin{align*}
%\mbb{P}\left[ \left(\frac{1}{M}\sum\limits_{j\in[g_i]}\sum\limits_{k\in[M]}x[\theta_j+k]y[k]\right)\geq \delta\right]\leq \exp\left\lbrace-\frac{M\delta^2}{2g_i}\right\rbrace\\
%\end{align*}
%Recall that $[g_i]$ is used to denote the set $\{0,1,\ldots,g_i-1\}$.
%\end{lemma}
%\begin{proof}
%The detailed proof is provided in our longer version \cite{nagaraj2017pattern} (Lemma 7)
%\end{proof}
%-------------------------------------------------
%\begin{proof}
%Since $\theta_j$ is not one of the matching positions for any $j$, $x[\theta_j+k]\neq y[k]$ and more importantly $x[\theta_j+k]\indep y[k] ~\forall j,k$. This implies that $x[\theta_j+k]y[k]=\pm 1$ with equal probability and $\mbb{E}[x[\theta_i+k]y[k]]=0$. Let the set of random variables corresponding to a position $\theta_j$ be $S_{j}\coleq \{x[\theta_j+k]y[k],k\in[M]\}$. It is clear that the random variables in the set $S_j$ are all independent with respect to each other due to our i.i.d assumption on the database $\xv$ and $\theta_j$ being a non-matching position. For the case of $\mu<\alpha$ we have $M<f_i$ for large enough $N$ thus resulting in non-overlapping parts of $\xv$  participating in the correlation coefficients $r[\theta_i]$ and $r[\theta_j]$. Hence it can be shown that $S_i\indep S_j~~\forall i,j$ and we can apply the bounds from Eqn. \eqref{Eqn:HoeffdingBoundedRV} achieve the required result.
%
%For the case of $\mu\geq\alpha$, $M>f_i$ for large enough $N$ which results in a coefficient $x[j]$ participating in multiple correlation coefficients $r[\theta_j]$. Therefore we pursue an alternate method of proof by defining 
%$$
%p_{j,l}\coleq x[\theta_j+l]\sum_{k\in[\frac{M}{f_i}]} y[l+kf_i] ~\text{ for } l\in[f_i],
%$$
%where $\theta_j=\theta_0+jf_i$. W.L.O.G we assume that $f_i$ divides $M$ evenly although the proof can be extended on similar lines for the case where $f_i$ does not divide $M$ evenly. Now we can show that the required sum
%\begin{align*}
%\sum\limits_{j\in[g_i]}\sum\limits_{l\in[M]}x[\theta_j+l]y[l]=\sum_{j\in[g_i]}\sum_{l\in[f_i]}p_{j,l}.
%\end{align*}
%From the above equivalent representation of the required sum, we need the following to achieve the required result:
%\begin{itemize}
%\item $p_{j,l}$ is sub-Gaussian with parameter $\sigma_i=\sqrt[•]{\frac{M}{f_i}}$ since, from Eqn. \eqref{Eqn:HoeffdingBoundedRV}, 
%\begin{align*}
%\mbb{P}\left[ p_{j,l}\leq \delta\right] \leq \exp\left\lbrace \frac{-\delta^2}{2\frac{M}{f_i}}\right\rbrace
%\end{align*} 
%\item The random variables $\{p_{j,l},~\forall j,l\}$ are independent of each other due to the i.i.d assumption on the database
%\end{itemize}
% Now we can apply the tail bounds for sum of $g_i f_i$ sub-Gaussian random variables each with sub-Gaussian parameter $\sqrt{\frac{M}{f_i}}$ from Lem. \ref{Lem:Chernoff} and arrive at the required result.
%\end{proof}
%---------------------------------------------------------------------


\section{Bin Classification Errors}
\label{Append:BinClassif}
We employ classification rules based only on the first element of the measurement vector at bin $(i,j)$ which can be given by
\begin{align}
Z[1]=\begin{cases}
\sum\limits_{\ell=0}^{g_{i}-1}\sum\limits_{k=0}^{M-1} n_{l,k}  & ~~\text{ if } ~~ \msc{H}=\msc{H}_z\label{Eqn:BinCombination}\\
\vspace{\vgap}
M_1+\sum\limits_{\ell=0}^{g_{i}-2}\sum\limits_{k=0}^{M-1} n_{l,k}  & ~~\text{ if } ~~ \msc{H}=\msc{H}_s\\
\vspace{\vgap}
M_1+M_2+\sum\limits_{\ell=0}^{g_{i}-3}\sum\limits_{k=0}^{M-1} n_{l,k}  & ~~\text{ if } ~~ \msc{H}=\msc{H}_d\\
\end{cases}
\end{align}
where $n_{l,k}=x[\theta_{\ell}+k]y[k]$ and $\theta_{\ell}\notin\{\tau_1,\tau_2,\ldots,\tau_L\}$. Also for the case of exact matching $M_1=M_2=M$ whereas in the case of approximate matching the values of $M_1,M_2\in[M(1-2\eta):M]$.

\begin{lemma}[zero-ton]
\label{Lem:ZerotonClassif}
Given that the bin $(i,j)$ is a zero-ton, the classification error can be bounded by
\begin{align*}
\mbb{P}[\mc{E}_1|\msc{H}_z]\leq e^{-\frac{N^{\mu+\alpha-1}(1-2\eta)^2}{8}}
\end{align*}
\end{lemma}
%\begin{proof}
%The above expression can be derived by observing that a bin is not classified as zero-ton if $\frac{Z[1]}{M}\geq\frac{1-2\eta}{2}$. Let us denote the probability of this event as $p_{z1}$ which can be bounded as:
%\begin{align*}
%p_{z1}=&\mbb{P}\left[\frac{Z[1]}{M}\geq\frac{1-2\eta}{2}\right]\\
%&\leq e^{-\frac{Mg_i(1-2\eta)^2}{8g_i^{2}}} \approx e^{-\frac{N^{\mu+\alpha-1}(1-2\eta)^2}{8}}
%\end{align*} 
%where the second bound is due to Eqn. \eqref{Eqn:BinCombination} and Lemma.~\ref{Lem:tailbounds}. The approximation in the third line is from our design that all the $g_i$ are chosen such that $g_i\approx N^{1-\alpha}$ and $M=N^{\mu}.$
%\end{proof}

\begin{lemma}[singleton]
\label{Lem:SingletonClassif}
Given that the bin $(i,j)$ is a singleton, the classification error can be bounded by
\begin{align*}
\mbb{P}[\mc{E}_1|\msc{H}_s]\leq 2e^{-\frac{N^{\mu+\alpha-1}(1-4\eta)^2}{16}}
\end{align*}
\end{lemma}
%----------------------------------------------------------------------------------
%\begin{proof}
%We observe that a bin is not classified as singleton if $\frac{Z[1]}{M}\leq\frac{1-2\eta}{2}$ or $\frac{Z[1]}{M}\geq\frac{3-4\eta}{2}$. Let us denote the probability of the two events as $p_{s1}$ and $p_{s2}$ respectively which can be bounded as:
%\begin{align*}
%p_{s1}&=\mbb{P}\left[\frac{1}{M}\sum\limits_{\ell=0}^{g_{i}-2}\sum\limits_{k=0}^{M-1} n_{l,k}\leq\frac{1-2\eta}{2}-\frac{M_1}{M}\right]\\
%&\leq \mbb{P}\left[\frac{1}{M}\sum\limits_{\ell=0}^{g_{i}-2}\sum\limits_{k=0}^{M-1} n_{l,k}\leq-\frac{1-2\eta}{2}\right]\\
%&\leq e^{-\frac{Mg_i(1-2\eta)^2}{16g_i^{2}}}\\
%&\approx e^{-\frac{N^{\mu+\alpha-1}(1-2\eta)^2}{16}}
%\end{align*} 
%where we used  {\it lower tail} of Lemma \ref{Lem:Chernoff} and $g_i\approx N^{1-\alpha}$ and the lower bound on $\frac{M_1•}{M}\geq (1-2\eta)$. Similarly $p_{s2}$ can be upper bounded by:
%% $p_{s2|\leq e^{-\frac{N^{\mu-\alpha}(1-2\eta)^2}{16}}$
%\begin{align*}
%p_{s2}&=\mbb{P}\left[\frac{1}{M}\sum\limits_{\ell=0}^{g_{i}-2}\sum\limits_{k=0}^{M-1} n_{l,k}\geq\frac{3-4\eta}{2}-\frac{M_1}{M}\right]\\
%&\leq \mbb{P}\left[\frac{1}{Mg_i}\sum\limits_{\ell=0}^{g_{i}-2}\sum\limits_{k=0}^{M-1} n_{l,k}\geq-\frac{1-4\eta}{2g_i}\right]\\
%&\approx e^{-\frac{N^{\mu+\alpha-1}(1-4\eta)^2}{8}}
%\end{align*} 
%Thus the overall probability of error for classifying a singleton can be obtained by combining $p_{s1}$ and $p_{s2}$.
%\end{proof}
%---------------------------------------------------------------------------------
\begin{lemma}[double-ton]
\label{Lem:DoubletonClassif}
Given that the bin $(i,j)$ is a double-ton, the classification error can be bounded by
\begin{align*}
\mbb{P}[\mc{E}_1|\msc{H}_d]\leq 2e^{-\frac{N^{\mu+\alpha-1}(1-6\eta)^2}{16}}
\end{align*}
\end{lemma}
%---------------------------------------------------------------------------------
%\begin{proof}
%We observe that a bin is not classified as double-ton if $\frac{Z[1]}{M}\leq\frac{3-4\eta}{2}$ or $\frac{Z[1]}{M}\geq\frac{5-6\eta}{2}$. Let us denote the probability of these two events as $p_{d1}$ and $p_{d2}$ respectively which can be bounded similar to Lemma ~\ref{Lem:SingletonClassif}.
%\begin{align*}
%p_{d1}&=\mbb{P}\left[\frac{1}{M}\sum\limits_{\ell=0}^{g_{i}-3}\sum\limits_{k=0}^{M-1} n_{l,k}\leq\frac{3-4\eta}{2}-\frac{M_1+M_2}{M}\right]\\
%&\leq \mbb{P}\left[\frac{1}{M}\sum\limits_{\ell=0}^{g_{i}-3}\sum\limits_{k=0}^{M-1} n_{l,k}\leq-\frac{1-4\eta}{2}\right]\\
%&\leq e^{-\frac{M(g_i-2)(1-4\eta)^2}{16(g_i-2)^{2}}}\\
%&\approx e^{-\frac{N^{\mu+\alpha-1}(1-4\eta)^2}{16}}.
%\end{align*} 
%Similarly $p_{d2}$ can be bounded as 
%\begin{align*}
%p_{d2}&=\mbb{P}\left[\frac{1}{M}\sum\limits_{\ell=0}^{g_{i}-3}\sum\limits_{k=0}^{M-1} n_{l,k}\geq\frac{5-6\eta}{2}-\frac{M_1+M_2}{M}\right]\\
%&\leq \mbb{P}\left[\frac{1}{M}\sum\limits_{\ell=0}^{g_{i}-3}\sum\limits_{k=0}^{M-1} n_{l,k}\geq\frac{1-6\eta}{2}\right]\\
%&\leq e^{-\frac{M(g_i-2)(1-6\eta)^2}{8(g_i-2)^{2}}}\\
%&\approx e^{-\frac{N^{\mu+\alpha-1}(1-6\eta)^2}{8}}
%\end{align*} 
%where we use the lower bounds $M_1,M_2\leq M$.
%\end{proof}
%--------------------------------------------------------------------------------

\begin{lemma}[multi-ton]
\label{Lem:MultitonClassif}
Given that the bin $(i,j)$ is a multi-ton, the classification error can be bounded by
\begin{align*}
\mbb{P}[\mc{E}_1|\msc{H}_m]\leq e^{-\frac{N^{\mu+\alpha-1}(1-6\eta)^2}{16}}
\end{align*}
\end{lemma}

\begin{proof}
	The proof for Lemmas  ~\ref{Lem:ZerotonClassif} ~\ref{Lem:SingletonClassif},~\ref{Lem:DoubletonClassif}, ~\ref{Lem:MultitonClassif} follows similar lines as in Lemma~\ref{Lem:ZerotonClassif}. For a detailed proof refer to our longer version \cite{janakiraman2017sub} (Lemmas 9,10,11).
\end{proof}
%-------------------------------------------------------------------------------
%\begin{proof}
%We observe that a bin is not classified as multi-ton if $\frac{Z[1]}{M}\leq\frac{5-6\eta}{2}$. Let us denote the probability of this event as $p_{m1}$ which can be bounded as:
%\begin{align*}
%p_{m1}&=\mbb{P}\left[\frac{1}{M}\sum\limits_{\ell=0}^{g_{i}-3}\sum\limits_{k=0}^{M-1} n_{l,k}\leq\frac{5-6\eta}{2}-\frac{M_m}{M}\right]\\
%&\leq \mbb{P}\left[\frac{1}{M}\sum\limits_{\ell=0}^{g_{i}-m}\sum\limits_{k=0}^{M-1} n_{l,k}\leq-\frac{1-6\eta}{2}\right]\\
%&\leq e^{-\frac{M(g_i-m)(1-4\eta)^2}{16(g_i-m)^{2}}}\\
%&\leq e^{-\frac{M(1-6\eta)^2}{16 n_i}}\\
% &\approx e^{-\frac{N^{\mu+\alpha-1}(1-6\eta)^2}{16}}.
%\end{align*} 
%\end{proof}
%-------------------------------------------------------------------------------
\section{Position Identification}
\label{Append:PositionIdentif}
We will analyze the singleton identification in two separate cases:
\begin{itemize}
\item $\mc{E}_{21}$: Event where the position is identified incorrectly when the bin is classified  correctly a singleton
\item $\mc{E}_{22}$: In the case of approximate matching, event where the position is identified incorrectly when the bin is originally a double-ton and one of the non-zero variable nodes has already been peeled off
\end{itemize}

\begin{definition}[Mutual Incoherence]
	The mutual incoherence $\mu_{\text{max}}( \mb{W})$ of a matrix $\mb{W} = [\wv_1 ~ \wv_2 ~ \cdots \wv_i \cdots \wv_N ]$ is defined as 
	
	\[\mu_{\text{max}}(\mb{W}) \defeq \max \limits_{\forall i \neq j} \frac{|\wv_i^{\dagger} \wv_j |}{||\wv_i || . ||\wv_j ||} \]
\end{definition}

\begin{lemma}[Mutual Incoherence Bound for sub-sampled IDFT matrix  [\cite{pawar2014robust},Proposition~A.1]
\label{lemma:MutualCoherence}
	The mutual incoherence $\mu_{\text{max}}$ $(\mb{W_{i,k}})$ of the sensing matrix $\mb{W}_{i,k}$ (defined in Eq.~\ref{Eqn:Sensing Matrix}), with $B$ shifts, is upper bounded by
	\[ \mu_{\text{max}} < 2\sqrt{\frac{\log(5N)}{B}} \] 
	
\end{lemma}
\begin{proof}
	The proof follows similar lines as the proof for Lemma V.3. in \cite{pawar2014robust}.
\end{proof}
 
\begin{lemma} \label{Lem:Pos1}
For some constant $c_1 \in \mathbb{R}$ and the choice of $B=4c_1^2\log 5N$,  the probability of error in identifying the position of a singleton at any bin $(i,j)$ can be upper bounded by
\begin{align*}
\mbb{P}[\mc{E}_{21}]\leq \exp\left\lbrace-\frac{N^{\mu+\alpha-1}(1-2\eta)^2(c_1^2-1)}{8(c_1^2+1)}\right\rbrace
\end{align*}
\end{lemma}
%-------------------------------------------------------------------------------
%\begin{proof}
%	
%	Let $j_p$ be the variable node participating in the singleton $(i,j)$. Then the observation vector $\zv_{i,j}$ is given by
%	\begin{align*}
%	\underline{z}_{i,j} &= \begin{bmatrix}
%	\wv_{j_{1}},\wv_{j_2}, & \cdots   & \wv_{j_p}, &\cdots \ &\wv_{j_{g_i}}
%	\end{bmatrix} \times
%	\begin{bmatrix}
%	n_{1} \\
%	\vdots \\
%	r[j_p]\\
%	\vdots\\
%	n_{j} \\
%	\vdots\\
%	n_{g_i}\\
%	\end{bmatrix}\\
%	&= r[j_p] ~ \wv_{j_p}+ \sum_{k \neq p}n_k \wv_{j_k} \\
%	\end{align*}
%	where for convenience we use a simpler notation $j_k=j+(k-1)\frac{N}{f_i}, \wv_{j_k}=\wv^{j_k}$ as defined in Eq. and $n_{l}=\sum\limits_{k=0}^{M-1}x[\theta_{\ell}+k]y[k]$ as defined in Eq. \eqref{Eqn:BinCombination}.
%	
%	The estimated position $\hat{p}$ is given by
%	\begin{align}
%	\label{Eqn:SingletonBinCombination}
%	\hat{p}= \underset{l}{\argmax}~~ \frac{\wv_{j_l}^{\dagger}\underline{z}_{i,j}}{B}
%	\end{align}
%	where $\dagger$ denotes the conjugate transpose of the vector. Also note that $|| \wv_{j_k}||=B$ for any $j$ and $k$.  From Eq. \eqref{Eqn:SingletonBinCombination} we observe that the position is wrongly identified when $\exists p'$ such that
%	\begin{align*}
%	&r[j_p] + \frac{1}{B}\sum_{k \neq p} n_k 	\wv_{j_p}^{\dagger}\wv_{j_k} \leq \frac{r[j_p]}{B} ~ \wv_{j_{p'}}^{\dagger}\wv_{j_p}+ n_{p'}+\frac{1}{B}\sum_{k \neq p,p'}n_k\wv_{j_{p'}}^{\dagger} \wv_{j_k} \\
%	&\leftrightarrow \sum_{k \neq p,p'}\alpha_k n_k+\beta n_{p'}\geq  r[j_p]\left(1-\frac{\wv_{j_{p'}}^{\dagger}\wv_{j_p}}{B}\right)\geq M(1-2\eta)(1-\mu_{\text{max}})
%	\end{align*}
%	where $\alpha_k$ and $\beta$ are constants and can be shown to be in the range $\alpha_k\in[-2\mu_\text{max},2\mu_\text{max}]$ and $\beta\in[1-\mu_\text{max},1+\mu_\text{max}]$. Now using the bound given Chernoff Lemma in Lem.~\ref{Lem:tailbounds} we obtain
%	\begin{align*}
%	\mbb{P}[\mc{E}_{21}]&\leq \exp\left\lbrace-\frac{2M(1-2\eta)^2(1-\mu_{\text{max}})^2}{16g_i\mu^2_{\max}+4(1+\mu_{\max})^2}\right\rbrace\\
%	&\leq\exp\left\lbrace-\frac{2M(1-2\eta)^2(1-\mu_{\text{max}})^2}{16(g_i\mu^2_{\max}+1)}\right\rbrace\\
%	&\leq\exp\left\lbrace-\frac{2M(1-2\eta)^2(c_1-1)^2}{16(g_i+c_1^2)}\right\rbrace\\
%	&\approx\exp\left\lbrace-\frac{N^{\mu+\alpha-1}(1-2\eta)^2(c_1^2-1)}{8(c_1^2+1)}\right\rbrace\\
%	\end{align*}
%	The second inequality follows by the definition of  $\mu_{\text{max}} \leq 1$.  We choose $B=4c_1^2\log 5N$, and substituting $\mu_{\max}\leq 2\sqrt{\frac{\log 5N}{B}} = 1/c_1$ (Lemma \ref{lemma:MutualCoherence}) we get the third inequality.
%\end{proof}

\begin{lemma}\label{Lem:Pos2}
For some constant $c_1 \in \mathbb{R}$ and the choice of $B=4c_1^2\log 5N$, the probability of error in identifying the position of second non-zero variable node at a double-ton at any bin $(i,j)$, given that the first position identification is correct, can be upper bounded by
	\begin{align*}
		\mbb{P}[\mc{E}_{22}]\leq \exp\left\lbrace-\frac{N^{\mu+\alpha-1} ~ (c_1(1 - 2\eta) - 1)^2}{8(1+ c_1^2)}\right\rbrace
	\end{align*}
\end{lemma}

\begin{proof}
	The proof for Lemmas ~\ref{Lem:Pos1} and ~\ref{Lem:Pos2} can be found in our longer version\cite{janakiraman2017sub} (Lemma 14,15).
\end{proof}
%---------------------------------------------------------------------------------
%\begin{proof}
%	
%	{\bf $\mc{E}_{22}$:}
%	
%	Let $j_p$ and $j_{\tilde{p}}$ be the two variable nodes participating in the doubleton $(i,j)$. Then the observation vector $\zv_{i,j}$ is given by 
%	
%	\begin{align*}
%		\underline{z}_{i,j} &= \begin{bmatrix}
%			\wv_{j_{1}},\wv_{j_2}, & \cdots   & \wv_{j_p}, &\cdots \ &\wv_{j_{g_i}}
%		\end{bmatrix} \times
%		\begin{bmatrix}
%			n_{1} \\
%			\vdots \\
%			r[j_p]\\
%			\vdots\\
%			n_{j} \\
%			\vdots\\
%			r[j_{\tilde{p}}]\\
%			\vdots\\
%			n_{g_i}\\
%		\end{bmatrix}\\
%		&= r[j_p] ~ \wv_{j_p} + r[j_{\tilde{p}}] ~ \wv_{j_{\tilde{p}}} + \sum_{k \neq p}n_k \wv_{j_k} \\
%	\end{align*}
%	
%	Let the contribution from $j_{\tilde{p}}$ be peeled off from the doubleton at some iteration, then we get
%	\[ \zv_{i,j} = r[j_p] ~ \wv_{j_p} + \frac{e_1}{B} ~ \wv_{j_{\tilde{p}}} + \sum_{k \neq p}n_k \wv_{j_k}\]
%	
%	where $e_1 \in[-\eta M, \eta M]$ is an extra error term induced due to peeling off.
%	
%	Now the estimated second position $\hat{p}$ is calculated using Eq. \eqref{Eqn:SingletonBinCombination}. We can observe that the position is wrongly identified when $\exists p'$ such that
%	\[ \frac{\wv_{j_p}^{\dagger}\underline{z}_{i,j}}{B} \leq \frac{\wv_{j_{p'}}^{\dagger}\underline{z}_{i,j}}{B}\]
%	\begin{align*}
%		&\implies r[j_p] + \frac{1}{B}\sum_{k \neq p, \tilde{p}} n_k 	\wv_{j_p}^{\dagger}\wv_{j_k} + \frac{e_1}{B} \wv_{j_p}^{\dagger}\wv_{j_{\tilde{p}}} \\ & \qquad \leq \frac{r[j_p]}{B} ~ \wv_{j_{p'}}^{\dagger}\wv_{j_p}+ n_{p'}+\frac{1}{B}\sum_{k \neq p,p',\tilde{p}}n_k\wv_{j_{p'}}^{\dagger} \wv_{j_k} + \frac{e_1}{B} \wv_{j_{p'}}^{\dagger}\wv_{j_{\tilde{p}}}
%	\end{align*}
%	\begin{align*}
%		&\leftrightarrow \sum_{k \neq p,p',\tilde{p}}\alpha_k n_k+ \beta n_{p'}  \geq  r[j_p]\left(1-\frac{\wv_{j_{p'}}^{\dagger}\wv_{j_p}}{B}\right) - \frac{2 \eta M}{B} \wv_{j_{p'}}^{\dagger}\wv_{j_{\tilde{p}}}
%	\end{align*}
%	\[~~\geq M(1-2\eta)(1-\mu_{\text{max}}) - 2 \eta M \mu_{\text{max}} = M(1 - 2\eta - \mu_{\text{max}})                         
%	\]
%	where $\alpha_k$ and $\beta$ are constants and can be shown to be in the range $\alpha_k\in[-2\mu_\text{max},2\mu_\text{max}]$ and $\beta\in[1-\mu_\text{max},1+\mu_\text{max}]$. Now using the bound given by Chernoff Lemma in Lem.~\ref{Lem:tailbounds} we obtain
%	\begin{align*}
%		\mbb{P}[\mc{E}_{22}]&\leq \exp\left\lbrace-\frac{2M(1 - 2\eta - \mu_{\text{max}})^2}{16g_i\mu^2_{\max}+4(1+\mu_{\max})^2}\right\rbrace\\
%		&\leq\exp\left\lbrace-\frac{2M(1 - 2\eta - \mu_{\text{max}})^2}{16(g_i\mu^2_{\max}+1)}\right\rbrace\\
%		&\leq\exp\left\lbrace-\frac{M(c_1(1 - 2\eta) - 1)^2}{8(g_i+c_1^2)}\right\rbrace\\
%		&\leq\exp\left\lbrace-\frac{N^{\mu+\alpha-1} ~ (c_1(1 - 2\eta) - 1)^2}{8(1+ c_1^2)}\right\rbrace\\
%	\end{align*}
%	where for the choice of $B=4c_1^2\log 5N$, $\mu_{\max}\leq 2\sqrt{\frac{\log 5N}{B}} = 1/c_1$.
%	
%\end{proof}
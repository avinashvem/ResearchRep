\documentclass[sigconf]{acmart}
%\usepackage{booktabs} % For formal tables

\usepackage{amsmath,amssymb,amsthm}
\usepackage{enumerate}
\usepackage{stfloats}
\usepackage{comment}

\usepackage{graphics} % for pdf, bitmapped graphics files
\usepackage{epsfig} % for postscript graphics files
\usepackage{mathptmx} % assumes new font selection scheme installed
\usepackage[mathscr]{euscript}
\usepackage{algorithm}
\usepackage[noend]{algpseudocode}

\makeatletter
\def\BState{\State\hskip-\ALG@thistlm}
\makeatother

\usepackage{tikz}
\usetikzlibrary{arrows,shapes,chains,matrix,positioning,scopes,patterns,calc}
\usepackage{pgfplots}
\usepgflibrary{shapes}

\newtheorem{remark}[theorem]{Remark}
\renewcommand{\epsilon}{\varepsilon}

\newcommand{\h}{\texttt{h}}
\newcommand{\hbp}{\h^{\mathrm{BP}}}
\newcommand{\hmap}{\h^{\mathrm{MAP}}}
\newcommand{\hstab}{\h^{\mathrm{stab}}}
\newcommand{\harea}{\h^{A}}
\newcommand\indep{\protect\mathpalette{\protect\independenT}{\perp}}
\def\independenT#1#2{\mathrel{\rlap{$#1#2$}\mkern2mu{#1#2}}}

\newcommand{\expt}{\mathbb{E}}
\newcommand{\indicator}[1]{\mathbbm{1}_{\left\{ {#1} \right\} }}
\newcommand{\abs}[1]{\left\lvert#1\right\rvert}

\newcommand{\mb}[1]{\mathbf{#1}}
\newcommand{\mbb}[1]{\mathbb{#1}}
\newcommand{\mr}[1]{\mathrm{#1}}
\newcommand{\mc}[1]{\mathcal{#1}}
\newcommand{\ms}[1]{\mathsf{#1}}
\newcommand{\msc}[1]{\mathscr{#1}}
\newcommand{\mf}[1]{\mathfrak{#1}}

\newcommand{\RNum}[1]{\uppercase\expandafter{\romannumeral #1\relax}}

\newcommand{\mse}{\mathsf{e}}
\newcommand{\msx}{\mathsf{x}}
\newcommand{\msxvn}{\tilde{\mathsf{x}}}
\newcommand{\msy}{\mathsf{y}}
\newcommand{\msz}{\mathsf{z}}
\newcommand{\msa}{\mathsf{a}}
\newcommand{\msb}{\mathsf{b}}
\newcommand{\msbx}{\underline{\mathsf{x}}}
\newcommand{\msby}{\underline{\mathsf{y}}}
\newcommand{\msbxvn}{\tilde{\underline{\mathsf{x}}}}
\newcommand{\msbz}{\underline{\mathsf{z}}}
\newcommand{\msba}{\underline{\mathsf{a}}}
\newcommand{\msbb}{\underline{\mathsf{b}}}
\newcommand{\msbc}{\underline{\mathsf{c}}}

%\newcommand{\bv}{\underline{\mathrm{b}}}
%\newcommand{\xv}{\underline{\mathrm{x}}}
%\newcommand{\yv}{\underline{\mathrm{y}}}
%\newcommand{\zv}{\underline{\mathrm{z}}}
%\newcommand{\rv}{\underline{\mathrm{r}}}
%\newcommand{\wv}{\underline{\mathrm{w}}}

%\newcommand{\Xv}{\underline{\mathrm{X}}}
%\newcommand{\Yv}{\underline{\mathrm{Y}}}
%\newcommand{\Zv}{\underline{\mathrm{Z}}}
%\newcommand{\Rv}{\underline{\mathrm{R}}}
%\newcommand{\RXYv}{\underline{\mathrm{R}_{XY}}}

\newcommand{\bv}{\vec{\mathrm{b}}}
\newcommand{\xv}{\vec{\mathrm{x}}}
\newcommand{\yv}{\vec{\mathrm{y}}}
\newcommand{\zv}{\vec{\mathrm{z}}}
\newcommand{\rv}{\vec{\mathrm{r}}}
\newcommand{\wv}{\vec{\mathrm{w}}}


\newcommand{\Xv}{\vec{X}}
\newcommand{\Yv}{\vec{Y}}
\newcommand{\Zv}{\vec{Z}}
\newcommand{\Rv}{\vec{R}}
\newcommand{\RXYv}{\vec{R}_{XY}}

\DeclareMathAlphabet{\mcl}{OMS}{cmsy}{m}{n}

\newcommand{\wh}{\widehat}
\newcommand{\bop}{\ast}
\newcommand{\vnop}{\varoast}
\newcommand{\disth}{d_{\mathrm{H}}}
\newcommand{\cnop}{\boxast}
\newcommand{\diff}[1]{d#1}
\newcommand{\deri}[1]{\mathrm{d}_{ #1 }\hspace{0.05cm}}
\newcommand{\dderi}[1]{\mathrm{d}_{ #1 }^2\hspace{0.05cm}}
\newcommand{\bvert}[1]{\,\Big{\vert}_{ #1  }}
\newcommand{\degr}{\succ}
\newcommand{\degreq}{\succeq}
\newcommand{\upgr}{\prec}
\newcommand{\upgreq}{\preceq}
\newcommand{\extR}{\overline{\mathbb{R}}}

\newcommand{\des}{\mathsf{T}_\mathrm{s}}
\newcommand{\dec}{\mathsf{T}_\mathrm{c}}
\newcommand{\pots}{U_\mathrm{s}}
\newcommand{\potc}{U_\mathrm{c}}
\newcommand{\shft}{\mathsf{S}}

\newcommand{\vnunit}{\Delta_0}
\newcommand{\cnunit}{\Delta_\infty}

\newcommand{\ent}[1]{ \mathrm{H} \left( #1 \right) }

\newcommand{\meass}{\mathcal{M}}
\newcommand{\probs}{\mathcal{X}}
\newcommand{\dpros}{\mathcal{X}_{\mathrm{d}}}
\newcommand{\chend}{N_{w}}

\newcommand{\minf}{\mathsf{a}_{0}}
\newcommand{\minfb}{\underline{\minf}}

\DeclareMathOperator*{\argmin}{\,arg\ min}
\DeclareMathOperator*{\argmax}{\,arg\ max}

\newlength\tikzwidth
\newlength\tikzheight

\textfloatsep=0.05in

\newcommand{\coleq}{\mathrel{\mathop:}=}
\newcommand{\defeq}{\triangleq}
% Copyright
%\setcopyright{none}
%\setcopyright{acmcopyright}
%\setcopyright{acmlicensed}
%\setcopyright{rightsretained}
%\setcopyright{usgov}
%\setcopyright{usgovmixed}
%\setcopyright{cagov}
%\setcopyright{cagovmixed}

% DOI
\acmDOI{0/0}

% ISBN
\acmISBN{000-0000-00-000/00/00}

%Conference
\acmConference[KDD Workshop on MiLeTS]{KDD Workshop on Mining and Learning from Time Series (MiLeTS)}{August 2017}{Halifax, Nova Scotia, Canada}
\acmYear{2017}
%\copyrightyear{2016}
\acmPrice{0.00}


\begin{document}
\title[Sub-string Matching in Sub-linear Time]{Sub-string/Pattern Matching in Sub-linear Time Using a Sparse Fourier Transform Approach}
%\titlenote{Produces the permission block, and copyright information}
%\subtitle{Extended Abstract}
%\subtitlenote{The full version of the author's guide is available as \texttt{acmart.pdf} document}

\author{Nagaraj T. Janakiraman, Avinash Vem, Krishna R. Narayanan, Jean-Francois Chamberland\\
Department of Electrical and Computer Engineering \\
Texas A\&M University\\
{\tt\small {\{tjnagaraj,vemavinash,krn,chmbrlnd\}@tamu.edu} }}

\renewcommand{\shortauthors}{Janakiraman et al.}
% The default list of authors is too long for headers}

\begin{abstract}
	We consider the problem of querying a string (or, a database) of length $N$ bits to determine all the locations where a substring (query) of length $M$ appears either exactly or is within a Hamming distance of $K$ from the query. We assume that sketches of the original signal can be computed off line and stored. Using a sparse Fourier transform computation based approach, we show that all such matches can be determined with high probability in sub-linear time. Specifically, if the query length $M = O(N^\mu)$ and the number of matches $L=O(N^\lambda)$, we show that for $\lambda < 1-\mu$ all the matching positions can be determined with a probability that approaches 1 as $N \rightarrow \infty$ for $K \leq \frac{1}{6}M$. More importantly our scheme has a worst-case computational complexity that is only $O\left(\max\{N^{1-\mu}\log^2 N, N^{\mu+\lambda}\log N \}\right)$, which means we can recover all the matching positions in {\it sub-linear} time for $\lambda<1-\mu$. This is a substantial improvement over the best known non-sketching based algorithm with computational complexity of $O\left(N^{1-0.359 \mu} \right)$ for recovering one matching position by Andoni {\em et al.} \cite{andoni2013shift}. Further, the number of Fourier transform coefficients that need to be computed, stored and accessed, i.e., the sketching complexity of this algorithm is only $O\left(N^{1-\mu}\log N\right)$. Several extensions of the main theme are also discussed.
\end{abstract}

% The code below should be generated by the tool at
% http://dl.acm.org/ccs.cfm
\begin{CCSXML}
	<ccs2012>
	<concept>
	<concept_id>10002950.10003712</concept_id>
	<concept_desc>Mathematics of computing~Information theory</concept_desc>
	<concept_significance>500</concept_significance>
	</concept>
	<concept>
	<concept_id>10003752.10003809.10010031.10010032</concept_id>
	<concept_desc>Theory of computation~Pattern matching</concept_desc>
	<concept_significance>500</concept_significance>
	</concept>
	<concept>
	<concept_id>10003752.10003809.10010055</concept_id>
	<concept_desc>Theory of computation~Streaming, sublinear and near linear time algorithms</concept_desc>
	<concept_significance>500</concept_significance>
	</concept>
	<concept>
	<concept_id>10003752.10003809.10010055.10010057</concept_id>
	<concept_desc>Theory of computation~Sketching and sampling</concept_desc>
	<concept_significance>500</concept_significance>
	</concept>
	<concept>
	<concept_id>10003752.10010061.10010067</concept_id>
	<concept_desc>Theory of computation~Error-correcting codes</concept_desc>
	<concept_significance>500</concept_significance>
	</concept>
	<concept>
	<concept_id>10003752.10003777</concept_id>
	<concept_desc>Theory of computation~Computational complexity and cryptography</concept_desc>
	<concept_significance>300</concept_significance>
	</concept>
	</ccs2012>
\end{CCSXML}

\ccsdesc[500]{Mathematics of computing~Information theory}
\ccsdesc[500]{Theory of computation~Pattern matching}
\ccsdesc[500]{Theory of computation~Streaming, sublinear and near linear time algorithms}
\ccsdesc[500]{Theory of computation~Sketching and sampling}
\ccsdesc[500]{Theory of computation~Error-correcting codes}
\ccsdesc[300]{Theory of computation~Computational complexity and cryptography}

\keywords{Pattern Matching, String Matching, Sub-linear time Algorithms, Sparse Signal Processing,	Hashing, Sparse Fourier Transform, Compressed Sensing}
\settopmatter{printacmref=false, printccs=false, printfolios=false	}
\setcopyright{none}
\maketitle

\input{Problem_statement_workshop}
\section{Notations}

This table below will introduce the notations we use in this paper.

\begin{center}
	
	\begin{tabular}{|c|c|}
		
		\hline
		
		Symbol     &  Notational Meaning \\
		
		\hline
		
		$N$           & Size of the string or database in symbols \\
		\hline
		$M = N^{\mu}$           & Length of the query in symbols \\
		\hline
        $L = N^\lambda$  &   Number of matches \\
        \hline
        $K$             & Maximum Hamming distance allowed between $\yv$ and $\xv$ \\
        \hline
		$G = N^\gamma$           & Number of blocks \\
		\hline
		$\tilde{N} = N^{1-\gamma}$   & Length of one block \\
		\hline
		$A = N^\alpha$      & Sub-sampling parameter \\
		\hline
		$B$       & Number of shifts  \\
		\hline
		$d$           & Number of stages in the FFAST algorithm \\
		\hline
	\end{tabular}
\end{center}	
	
	\vspace{15pt}


We denote vectors using the underbar, time domain signals using lowercase letters and the frequency domain signals using uppercase letter. For example $\xv = \{x[1],x[2], \cdots x[N] \}$ denotes a time domain signal with $i^{th}$ time component denoted by $x[i]$, and $\Xv= \mathcal{F}\xv$ denotes the Fourier coefficients of $\xv$. Matrices are denoted using uppercases letters. We differentiate a signal from a matrix by having a underbar for a signal vector. We denote the set ${0,1,2\cdots, N-1}$ by $[N]$.
	
	


\section{Description Of The Algorithm}
%Introduce the system model by explaining the FFAST (dual problem - IFFT) architecture and then describe the decoding algorithm - peeling (multiple matches case) and Product code approach (only one exact match case).      
In this section we describe our algorithm, with sample and time complexities that are sub-linear in $N$, to find the locations $\underline{\tau} = (\tau_1, \tau_2, \cdots \tau_L)$ in the string $\xv$, where the string (query) $\yv$ matches. This is achieved by computing the cross-correlation $\rv$ of $\xv$ and $\yv$, defined in Eqn.~\eqref{Eqn:DefCrossCorrelation}, and finding the positions where there is a significant peak. 
%To reiterate we define the cross-correlation $\rv = [r[0], r[1], \cdots, r[N]] $ of $\xv$ and $\yv$ as
%
%\begin{equation}\label{eqn:Rxy_def}
%r[m] \ =  \ \sum_{i=1}^{N} x[m+i] ~ y[i], \ 0 \leq m \leq N   
%\end{equation}

The main idea here is that $\RXYv$ is sparse (upto some noise) with dominant peaks at $L$ positions ($\tau$) where the strings match, and noise components at $N-L$ positions where the strings do not match. Consider the case of exact matching,
\begin{equation} \label{eqn:RXY_sparse}
r[m] \ = \left\{
\begin{array}{ll}
  &M , \ \ \  \text{if} \ m \in \mathcal{T} \\
  & n_m , \ \ \ m \in [N]-\mathcal{T}
\end{array} 
\right.  
\end{equation}
where, $ \mathcal{T}:=\{\tau_1, \tau_2, \cdots, \tau_L\}$,
% $M-K \leq v_m \leq M $ is the signal component and
 $n_m$ is the noise component that is induced due to correlation of two i.i.d. sequence of random variables each taking values from $\mathcal{A} := \{+1,-1\}$. The $\rv$ can also be computed as shown below:
\begin{equation}\label{eqn:Rxy_fourier}
  \rv = \underset{\text{ \RNum{1} } } {\mathcal{F}_{N}^{-1}} \ \{ \underset{\text{ \RNum{2} } }{  \mathcal{F}_{N}\{\xv\}}  \odot \ \underset{\text{ \RNum{3} } }{ \mathcal{F}_{N}\{\yv'\}}  \} 
\end{equation} 
where $\mathcal{F}_{N}\{ \cdot \}$ and $\mathcal{F}_{N}^{-1}\{ \cdot \}$ refers to $N$-point discrete Fourier transform and its inverse respectively, $\odot$ is the point-wise multiplication operation and ${ y'[n]} = { y^{*}[-n]}$. 

 We exploit this property of sparsity to compute $\RXYv$ by using only a subset of samples from $\Xv = \mathcal{F}\{\xv\} $ and $\Yv' = \mathcal{F}\{\yv'\} $, each sampled at positions $l \in \mathcal{S}$, where $\mathcal{S} = \mathcal{S}_{1,1} \cup \mathcal{S}_{1,2} \cup \cdots \cup \mathcal{S}_{i,j} \cup  \mathcal{S}_{d,B} \subset \{ 0,1,\cdots ,N-1 \}$ and $\mathcal{S}_{i,j}$, $1 \leq i \leq d $ and  $1 \leq j \leq B $ are disjoint sets of size $|\mathcal{S}_{i,j}| \approxeq N^{1-\alpha}$ with periodic sample points from $[N]$ given by 
 
 \begin{equation}
 \label{eqn:sampling_sets}\mathcal{S}_{i,j} = \{s_j,\ s_j + f_i,\ s_j + 2f_i,\ \cdots s_j + \lfloor{\frac{N}{f_i} }\rfloor f_i \}
 \end{equation}
   where $s_j$'s and $f_i$'s are constants chosen based on the requirements from Robust Sparse Inverse Fourier Transform (RSIDFT) framework described in Section~\ref{sec:RSIDFT}.

 As evident from Equation~\ref{eqn:Rxy_fourier}, our algorithm for computing $R_{XY}$ consists of three stages:
\begin{enumerate}
	\item[\RNum{1}] \textit{Computing the sketch of $\xv$}: 
	 We assume that the sketch of $\xv$, \ $ \Xv[l] = \mathcal{F}\{\xv\}$ is precomputed at positions $l \in \mathcal{S}$ and stored in a database.  
	\item[\RNum{2}] \textit{Computing the sketch of $\yv$}:
	 For every new query $\yv$, $ \Yv'[l]$ is computed at $l \in \mathcal{S}$. Naively, the FFT algorithm can be used to compute this with $O(N \log N)$ complexity. Since only a subset $\mathcal{S}$ of samples from $\Yv'$ is needed, this can be done by using a folding technique described below. 
	  The idea behind this folding technique is to induce aliasing in $\yv'$ and then taking a smaller point Fourier Transform to compute the sub-sampled version $\Yv'$. Aliasing is induced by folding the signal $\yv'$ into blocks of length $N^{1-\alpha}$ and adding them. The desired subsampling patterns in frequency domain are induced by multiplying $\yv'$ with suitable exponentials. Let us denote the aliased versions of $\yv'$ by $\underline{y^{a}_{i,j}}'$. Then, $\underline{y^{a}_{i,j}}'$ is given by
	  \begin{equation}
	  	{y^{a}_{i,j}}'[p] = \sum \limits_{m = 0}^{\lfloor{\frac{N}{f_i}}\rfloor} y'[p + mf_i] e^{j \frac{2 \pi s_j}{N} } 
	  \end{equation}
	  Taking $\frac{N}{f_i}$ point DFT of $\underline{y^{a}_{i,j}}'$ produces $\Yv'[l]$ sub-sampled at $l \in \mathcal{S}_{i,j}$. To obtain all the samples in $\mathcal{S}$, the folding procedure needs to be carried out $dB$ times, once for each $(i,j)$ pair, where $1 \leq i \leq d $ and  $1 \leq j \leq B $.  
	       
	\item[\RNum{3}] \textit{Computing sparse $\mathcal{F}^{-1}$}: 
	
	 Since $R_{XY}$ is sparse, we use a Robust Sparse Inverse  Discrete Fourier Transform(RSIDFT) framework to compute the $L$-sparse coefficients. The architecture of RSIDFT is similar to FFAST proposed in \cite{pawar2014robust}, but the decoding algorithm has some key modifications to handle the noise model induced in this problem.
	 
	 
	 \subsection{RSIDFT Framework} 	\label{sec:RSIDFT}
	
	  Let $ \Zv  =  \Xv \odot \Yv'$ be the input to RSIDFT framework. The RSIDFT framework computes the $L$ dominant coefficients of $\zv$ $= {\RXYv}$ by using only a subsampled version of $\Zv$, $Z[l]$ at positions $l \in \mathcal{S}$.\\
	  Let $N = P_1 \times P_2 \times \cdots P_d$ be the prime factorization of $N$, the length of signals $\bf x \ \text{and} \ y$. For a given $\alpha$, we choose distinct $f_i = \prod P_j$, $i \in [d]$ such that $f_i \approxeq N^{\alpha} $. 
	 
	 Consider the RSIDFT framework shown in Figure~\ref{fig:rsidft}. The framework consists of $d-$stages, each with a different sub-sampling factor $f_i$. In each stage, there are $B$ branches with shifts from $\underline{s}\ = [s_1, s_2, \cdots s_B] $, with $s_1 =0$ in the first branch, and the rest chosen randomly from $[A]$. We can also carefully choose the shifts to satisfy Mutual Coherence property and Restricted Isometric property described in the analysis section.
	 
	 \begin{figure}
	 	\begin{center}
	 		\includegraphics[height=7cm]{Figures/FFAST_Robust} 
	 	\end{center}	   
	 	\caption{ RSIDFT Framework to compute inverse Fourier Transform of a signal $\Rv$ that is sparse in time domain. }\label{fig:rsidft}
	\vspace{5 pt}
	 \end{figure}	
	        
	
	 
	 Given the inputs $\Zv$, in branch $j$ of $i$th stage RSIDFT sub-samples the signal $\Zv$ at $\mathcal{S}_{i,j} = \{s_j,\ s_j + f_i,\ s_j + 2f_i,\ \cdots s_j + \lfloor{\frac{N}{f_i} }\rfloor f_i$, to obtain $\Zv^{s}_{i,j}$. Sub-sampling is followed by a $\frac{N}{f_i}-$ point IDFT in each branch of stage $i$ to obtain $ \zv^{s}_{i,j}$. Notice that $ \zv^{s}_{i,j}$ is an aliased version of $\zv$. \\
	 Let $\zv^{b}_{i,p_i}$ be the observations that is formed by combining all $p_i$th coefficients of $\zv^{s}_{i,j}$ (belonging to stage $i$) together as a vector, where $1 \leq p_i \leq \frac{N}{f_i}$.
	 
	 \[ \zv^{b}_{i,p_i} = \begin{bmatrix}
	 z^{s}_{i,1}[p_i] \\
	 z^{s}_{i,2}[p_i] \\
	 \vdots\\
	 z^{s}_{i,B}[p_i]
	 \end{bmatrix}  \]
	 
	  A peeling decoder takes these observations $ \zv^{b}_{i,p_i}$ as inputs and computes the sparse $\RXYv$.
	 
	 \subsection{Peeling Decoder}
		
	 Each coefficient of $\underline{z^{b}_{i,p_i}}$ is a bin that has $N^{\alpha}$ coefficients from $\RXYv$ hashed into it. This can be seen using a Tanner graph with $\RXYv$ as the left nodes (bit nodes) and the aliased coefficients  $\underline{z^{b}_{i,p_i}}$ as the right nodes (check-nodes). Notice that each bit node has degree $d$ and each check-node degree is approximately $N^{\alpha}$. The peeling decoder has the following three steps in the decoding process.
		 \begin{itemize}
		 	\setlength{\itemindent}{.1in}
			 \item \textit{Bin Identification :} In this step a check-node is classified as a Zero-ton ($degree = 0$) or a Single-ton ($degree = 1)$ or a Multi-ton ($degree >1$). The classification is done based on a comparing the first observation,$z^{b}_{i,p_i}[1]$ (corresponding to zero shift), with a predefined threshold. The threshold is set differently for Exact Matching and Approximate Matching cases.\\
			 {\bf Exact Matching:} 
			                \[
			                \begin{array}{ll}
			                \text{Zero-ton:}& \ \ z^{b}_{i,p_i}[1] < M/2 \\
			                \text{Single-ton:}& \ \ M/2 \leq z^{b}_{i,p_i}[1] \leq 3M/2 \\ 
			                \text{Multi-ton:}& \ \ z^{b}_{i,p_i}[1] > 3M/2
			                \end{array}
			                \]   
			                
			 {\bf Approximate Matching:} 
			 \[
			 \begin{array}{ll}
			 \text{Zero-ton:}& \ \ z^{b}_{i,p_i}[1] < \frac{(1-3\eta/2)M}{2} \\ 
			 \text{Single-ton:}& \ \ \frac{(1-3\eta/2)M}{2} < z^{b}_{i,p_i}[1] < \frac{(1-\eta/2)3M}{2}   \\
			 \text{Double-ton:}& \ \ \frac{(1-\eta/2)3M}{2} < z^{b}_{i,p_i}[1] < \frac{(5-3\eta)M}{2}\\ 
			 \text{Multi-ton:}& \ \ z^{b}_{i,p_i}[1] > \frac{(5-3\eta)M}{2}\\
			 \end{array}
			 \] 
			   
 \item \textit{Position Identification :} Given that a check-node is classified as a Single-ton, we need to identify the position of the non-zero coefficient  hashed into it. This is done by correlating the observation vector $\underline{z^{b}_{i,p_i}}$ with each column $\underline{g_l}$ of  $G = [\underline{g_1} \ \underline{g_2} \cdots \underline{g_A}]$, where $\underline{g_l} =\begin{bmatrix}
+			 w_1^{l}, ~ w_2^{l}, ~ \cdots\, ~ w_B^{l}
+			 \end{bmatrix}^{T}$ with $w_k = e^{j \frac{2\pi s_k}{N}}$,  and then picking the column index  $p'$ that produced the maximum value.
\[ p' = \underset{l}{\argmax}\  \underline{s^T_l} ~ \underline{z^{b}_{i,p_i}}\]
			 
			 \item \textit{Peeling Process: } Although the main idea behind the peeling process is same for Exact Matching and the Approximate Matching scenarios, there is some minor differences in their implementation. The main idea here is to find the Singletons and then remove its contribution from other check-nodes where the bit-node corresponding to this Singleton participates.
			 
			 
			  {\bf Exact Matching:} Here we remove the identified Singleton's contribution from all the check-nodes it participates in.
			  
			  {\bf Approximate Matching:} Here we only remove the identified Singleton's contribution from a double-ton and do not alter multi-tons whose $degree > 2$.
			  		 
		 \end{itemize} 
	 
	 
\end{enumerate}

\section{Performance Analysis}
\def\vgap{2pt}
In this section, we will analyze the overall probability of error involved in finding the correct position of match. This can be done by analyzing the following three error events independently and then using a union bound to bound the total probability of error.

\begin{itemize}
	\item $\mathcal{E}_1${-\it Bin Classification}: Event that a bin is wrongly classified
	\item $\mathcal{E}_2${-\it Position Identification}: Event that a position is wrongly identified given the bin is correctly identified as a singleton  
	\item $\mathcal{E}_3${-\it Peeling Process} : Event that the peeling process fails to recover the $L$ significant correlation coefficients
\end{itemize}

\subsection{\bf Bin Classification}
\begin{lemma}
The probability of bin classification error at any bin $(i,j)$ can be upper bounded by
\begin{align*}
\mbb{P}[\mc{E}_1]\leq 6e^{-\frac{N^{\mu-\alpha}(1-6\eta)^2}{16}}
\end{align*}
\end{lemma}

\begin{proof}
\begin{align*}
\mbb{P}[\mc{E}_1]&=\mbb{P}[\msc{H}_z]\mbb{P}[\mc{E}_1|\widehat{\msc{H}}_{i,j}=\msc{H}_z]~+
						\quad \mbb{P}[\msc{H}_s]\mbb{P}[\mc{E}_1|\widehat{\msc{H}}_{i,j}=\msc{H}_s]~+
						\quad\mbb{P}[\msc{H}_d]\mbb{P}[\mc{E}_1|\widehat{\msc{H}}_{i,j}=\msc{H}_d \cup \msc{H}_m]\\
				&\leq \mbb{P}[\mc{E}_1|\widehat{\msc{H}}_{i,j}=\msc{H}_z]~+
						\quad \mbb{P}[\mc{E}_1|\widehat{\msc{H}}_{i,j}=\msc{H}_s]~+
						\quad \mbb{P}[\mc{E}_1|\widehat{\msc{H}}_{i,j}=\msc{H}_d \cup \msc{H}_m]\\
    			&\leq  e^{-\frac{N^{\mu-\alpha}(1-2\eta)^2}{8}}+2e^{-\frac{N^{\mu-\alpha}(1-4\eta)^2}{16}}+ 2e^{-\frac{N^{\mu-\alpha}(1-6\eta)^2}{16}}+e^{-\frac{N^{\mu-\alpha}(1-6\eta)^2}{16}}\\
    			&\leq 6e^{-\frac{N^{\mu-\alpha}(1-6\eta)^2}{16}}\\
						\end{align*}
						where the inequalities in the third line are due to Lemmas \ref{Lem:ZerotonClassif}, \ref{Lem:SingletonClassif}, \ref{Lem:DoubletonClassif} and \ref{Lem:MultitonClassif} respectively.
\end{proof}

\subsection{\bf Position Identification}
We will analyze the singleton identification in two separate cases:
\begin{itemize}
\item $\mc{E}_{21}$: Event where the position is identified incorrectly when the bin is classified  correctly a singleton
\item $\mc{E}_{22}$: In the case of approximate matching, event where the position is identified incorrectly when the bin is originally a double-ton and one of the non-zero variable nodes has already been peeled off
\end{itemize} 
\begin{lemma}
For the choice of $B=4c_1^2\log 5N$, the probability of error in identifying the position of a singleton at any bin $(i,j)$ can be upper bounded by
\begin{align*}
\mbb{P}[\mc{E}_{21}]\leq \exp\left\lbrace-\frac{N^{\mu-\alpha}(1-2\eta)^2(c_1^2-1)}{8(c_1^2+1)}\right\rbrace
\end{align*}
\end{lemma}
\begin{proof}
	
	Let $j_p$ be the variable node participating in the singleton $(i,j)$. Then the observation vector $\zv{i,j}$ is given by
	\begin{align*}
	\underline{z}_{i,j} &= \begin{bmatrix}
	\wv_{j_{1}},\wv_{j_2}, & \cdots   & \wv_{j_p}, &\cdots \ &\wv_{j_{f_i}}
	\end{bmatrix} \times
	\begin{bmatrix}
	n_{1} \\
	\vdots \\
	r[j_p]\\
	\vdots\\
	n_{j} \\
	\vdots\\
	n_{f_i}\\
	\end{bmatrix}\\
	&= r[j_p] ~ \wv_{j_p}+ \sum_{k \neq p}n_k \wv_{j_k} \\
	\end{align*}
	where for convenience we use a simpler notation $j_k=j+(k-1)\frac{N}{f_i}, \wv_{j_k}=\wv^{j_k}$ as defined in Eq. and $n_{l}=\sum\limits_{k=0}^{M-1}x[\theta_{\ell}+k]y[k]$ as defined in Eq. \eqref{Eqn:BinCombination}.
	
	The estimated position $\hat{p}$ is given by
	\begin{align}
	\label{Eqn:SingletonBinCombination}
	\hat{p}= \underset{l}{\argmax}~~ \frac{\wv_{j_l}^{\dagger}\underline{z}_{i,j}}{B}
	\end{align}
	where $\dagger$ denotes the conjugate transpose of the vector. Also note that $|| \wv_{j_k}||=B$ for any $j$ and $k$.  From Eq. \eqref{Eqn:SingletonBinCombination} we observe that the position is wrongly identified when $\exists p'$ such that
	\begin{align*}
	&r[j_p] + \frac{1}{B}\sum_{k \neq p} n_k 	\wv_{j_p}^{\dagger}\wv_{j_k} \leq \frac{r[j_p]}{B} ~ \wv_{j_{p'}}^{\dagger}\wv_{j_p}+ n_{p'}+\frac{1}{B}\sum_{k \neq p,p'}n_k\wv_{j_{p'}}^{\dagger} \wv_{j_k} \\
	&\leftrightarrow \sum_{k \neq p,p'}\alpha_k n_k+\beta n_{p'}\geq  r[j_p]\left(1-\frac{\wv_{j_{p'}}^{\dagger}\wv_{j_p}}{B}\right)\geq M(1-2\eta)(1-\mu_{\text{max}})
	\end{align*}
	where $\alpha_k$ and $\beta$ are constants and can be shown to be in the range $\alpha_k\in[-2\mu_\text{max},2\mu_\text{max}]$ and $\beta\in[1-\mu_\text{max},1+\mu_\text{max}]$. Now using the bound given Chernoff Lemma in Lem.~\ref{Lem:Chernoff2} we obtain
	\begin{align*}
	\mbb{P}[\mc{E}_{21}]&\leq \exp\left\lbrace-\frac{2M(1-2\eta)^2(1-\mu_{\text{max}})^2}{16f_i\mu^2_{\max}+4(1+\mu_{\max})^2}\right\rbrace\\
	&\leq\exp\left\lbrace-\frac{2M(1-2\eta)^2(1-\mu_{\text{max}})^2}{16(f_i\mu^2_{\max}+1)}\right\rbrace\\
	&\leq\exp\left\lbrace-\frac{2M(1-2\eta)^2(c_1-1)^2}{16(f_i+c_1^2)}\right\rbrace\\
	&\approx\exp\left\lbrace-\frac{N^{\mu-\alpha}(1-2\eta)^2(c_1^2-1)}{8(c_1^2+1)}\right\rbrace\\
	\end{align*}
	where for the choice of $B=4c_1^2\log 5N$, $\mu_{\max}\leq 2\sqrt{\frac{\log 5N}{B}}\leq 1/c_1$.
	
\end{proof}
\begin{lemma}
For the choice of $B=4c_1^2\log 5N$,the probability of error in identifying the position of second non-zero variable node at a double-ton at any bin $(i,j)$, given that the first position identification is correct, can be upper bounded by
\begin{align*}
\mbb{P}[\mc{E}_{22}]\leq  \exp\left\lbrace-\frac{N^{\mu - \alpha }~(c_1+1)^2}{8}\right\rbrace
\end{align*}
\end{lemma}
\begin{proof}

{\bf $\mc{E}_{22}$:}

Let $j_p$ and $j_{\tilde{p}}$ be the two variable nodes participating in the doubleton $(i,j)$. Then the observation vector $\zv{i,j}$ is given by 

\begin{align*}
\underline{z}_{i,j} &= \begin{bmatrix}
\wv_{j_{1}},\wv_{j_2}, & \cdots   & \wv_{j_p}, &\cdots \ &\wv_{j_{f_i}}
\end{bmatrix} \times
\begin{bmatrix}
n_{1} \\
\vdots \\
r[j_p]\\
\vdots\\
n_{j} \\
\vdots\\
r[j_{\tilde{p}}]\\
\vdots\\
n_{f_i}\\
\end{bmatrix}\\
&= r[j_p] ~ \wv_{j_p} + r[j_{\tilde{p}}] ~ \wv_{j_{\tilde{p}}} + \sum_{k \neq p}n_k \wv_{j_k} \\
\end{align*}
where for convenience we use a simpler notation $j_k=j+(k-1)\frac{N}{f_i}, \wv_{j_k}=\wv^{j_k}$ as defined in Eq. and $n_{l}=\sum\limits_{k=0}^{M-1}x[\theta_{\ell}+k]y[k]$ as defined in Eq. \eqref{Eqn:BinCombination}.

Let the contribution from $j_{\tilde{p}}$ be peeled off from the doubleton at some iteration, then we get
\[ \zv_{i,j} = r[j_p] ~ \wv_{j_p} \pm \eta M ~ \wv_{j_{\tilde{p}}} + \sum_{k \neq p}n_k \wv_{j_k}\]

Notice that $\eta M ~ \wv_{j_{\tilde{p}}}$ is an extra error term induced due to peeling off.
Now the estimated second position $\hat{p}$ is calculated using the Eq. \eqref{Eqn:SingletonBinCombination}. We can observe that the position is wrongly identified when $\exists p'$ such that
\[ \frac{\wv_{j_p}^{\dagger}\underline{z}_{i,j}}{B} \leq \frac{\wv_{j_{p'}}^{\dagger}\underline{z}_{i,j}}{B}\]
\begin{align*}
 &\implies r[j_p] + \frac{1}{B}\sum_{k \neq p, \tilde{p}} n_k 	\wv_{j_p}^{\dagger}\wv_{j_k} \pm \frac{\eta M}{B} \wv_{j_p}^{\dagger}\wv_{j_{\tilde{p}}}  \leq \frac{r[j_p]}{B} ~ \wv_{j_{p'}}^{\dagger}\wv_{j_p}+ n_{p'}+\frac{1}{B}\sum_{k \neq p,p',\tilde{p}}n_k\wv_{j_{p'}}^{\dagger} \wv_{j_k} \pm \frac{\eta M}{B} \wv_{j_{p'}}^{\dagger}\wv_{j_{\tilde{p}}}\\
&\leftrightarrow \sum_{k \neq p,p',\tilde{p}}\alpha_k n_k+ (\beta n_{p'} - M) \geq  r[j_p]\left(1-\frac{\wv_{j_{p'}}^{\dagger}\wv_{j_p}}{B}\right) - \frac{2 \eta M}{B} \wv_{j_{p'}}^{\dagger}\wv_{j_{\tilde{p}}} - M
\end{align*}
\[~~~~~~~~~~~~~~~~~~~\geq M(1-2\eta)(1-\mu_{\text{max}}) - M (2\eta \mu_{\text{max}} + 1) = -M(\mu_{\text{max}}+2\eta)                           
\]
where $\alpha_k$ and $\beta$ are constants and can be shown to be in the range $\alpha_k\in[-2\mu_\text{max},2\mu_\text{max}]$ and $\beta\in[1-\mu_\text{max},1+\mu_\text{max}]$. Now using the bound given by Chernoff Lemma in Lem.~\ref{Lem:Chernoff2} we obtain
\begin{align*}
\mbb{P}[\mc{E}_{22}]&\leq \exp\left\lbrace-\frac{2M(1+\mu_{\text{max}})^2}{16(f_i-3)\mu^2_{\max}+4{\mu_{\max}}^2}\right\rbrace\\
&\leq\exp\left\lbrace-\frac{2M(1+\mu_{\text{max}})^2}{16 f_i \mu^2_{\max}}\right\rbrace\\
&\leq\exp\left\lbrace-\frac{M(c_1+1)^2}{8 f_i}\right\rbrace\\
&\approx\exp\left\lbrace-\frac{N^{\mu - \alpha }~(c_1+1)^2}{8}\right\rbrace\\
\end{align*}
where for the choice of $B=4c_1^2\log 5N$, $\mu_{\max}\leq 2\sqrt{\frac{\log 5N}{B}} = 1/c_1$.

\end{proof}

%------------------------------------ Approximate Matching previous version --------------------------
%{\bf Approximate Matching:}
%\[
%\mathbf{Z_i} = \begin{bmatrix}
%\mathbf{s_1}       & \cdots   & \mathbf{s_p} &\cdots \ &\mathbf{s_A}
%\end{bmatrix} \times
%\begin{bmatrix}
%n_1 \\
%\vdots \\
%R_{XY}[p]\\
%\vdots\\
%n_j \\
%\vdots\\
%n_{A}\\
%\end{bmatrix}
%\]
%
%
%where $n_j \sim \mathcal{N}(0,M)$, $j \neq p$.
%
%\[\begin{array}{ll}
%Z_i[1] \ &= \ R_{XY}[p] + \sum_{j \neq p}n_j \\
%&= \ R_{XY}[p] + n 
%\end{array}
%\]
%where $n \sim \mathcal{N}(0,(N^\alpha-1)M)$. Once we identify a Singleton with probability of error $Pe_s \leq 2 e^{- \frac{(1-2\eta)^2N^{\mu-\alpha}}{8}}$, we can fix the value of $R_{XY}[p] = M(1-\eta/2)$ since we are only interested in finding the positions and not the exact value of correlations. 
%
%Now that we know $R_{XY}[p]$, 
%
%\[ \mathbf{Z_i} \ = \ R_{XY}[p] ~ \mathbf{s_p}+ \sum_{j \neq p}n_j \mathbf{s_j} \\
%\]
%
%%The estimated position $p'$ is given by
%%
%%\begin{align*}
%% p' = \underset{l}{\argmax}~~ \wv^{\dagger} \underline{z}_{i,j}
%% \end{align*}
%
%Also, let $p_1$ be the position of the previously peeled node from this check-node. There can only be at most one peeled edge as we restrict our peeling process to a doubleton and do not consider other multi-tons.
% 
%Let us consider two cases:
%
%{\textit{Case 1:} $p' = p$}
%\[
%\begin{array}{ll}
%c_1 \ &= \ \mathbf{s^T_{p'}} ~\mathbf{Z_i} \ = \ R_{XY}[p] {\bf ~s^T_{p'}~ s_p}\ + \ \sum_{j \neq p,p_1}n_j {\bf ~s^T_{p'}~ s_j} \ \pm \  \frac{\eta M}{2} {\bf ~s^T_{p'} ~ s_{p_1}} \\
%&\leq B ~ (M-\eta M/2) +  \sum_{j \neq p} 2 n_j ~ \mu_{max} \pm \frac{\eta M}{2} \mu_{max}
%\end{array} 
%\]
%
%{\textit{Case 2:} $p' \neq p$}
%\[
%\begin{array}{ll}
%c_2 \ &= \ \mathbf{s^T_{p'}} ~\mathbf{Z_i}\ = \ R_{XY}[p]  {\bf ~s^T_{p'}~ s_p}+ \sum_{j \neq p',p,p_1}n_j{\bf ~s^T_{p'}~ s_j} \ + \ n_{p'} {\bf ~s^T_{p'}~ s_p} \ \pm \ \frac{\eta M}{2} {\bf ~s^T_{p'}~ s_{p_1}}\\
%&\leq (M-\eta M/2) ~ \mu_{max} + \sum_{j \neq p',p} \ 2n_j ~ \mu_{max} \ + \ n_{p'} B \pm \frac{\eta M}{2} \mu_{max}
%\end{array} 
%\]
%
%\[
%\begin{array}{ll}
%c_1 - c_2 \ &=  \ R_{XY}[p](B - {\bf ~s^T_{p'}~ s_p}) \ - \ n_{p'}(B - {\bf ~s^T_{p'}~ s_p'}) ~ + ~ \sum_{j \neq p',p,p_1}n_j ({\bf ~s^T_{p}~ s_j} - {\bf ~s^T_{p'}~ s_j})  \pm \  \eta M {\bf ~s^T_{p'} ~ s_{p_1}}  \\
%&\leq (M-\eta M/2)(B-\mu_{max})  \ - \ n_{p'}(B - \mu_{max}) ~ + ~ \sum_{j \neq p',p}n_j \mu_{max} \pm \eta M \mu_{max}\\
%\end{array} 
%\]
%
%
%Notice that $c_1$ and $c_2$ are both Gaussian random variables given by
%
%\[ \begin{array}{ll}
%c_1 &\sim  \mathcal{N}(B ~ (M-\eta M/2) \pm \frac{\eta M}{2} \mu_{max} \ , \ 4N^\alpha M B \log(5N)) \\
%c_2 &\sim  \mathcal{N}(2M \sqrt{B\log(5N)}\ , \ M~B^2 + 4N^\alpha M B \log(5N))\\
%c_1 - c_2 &\sim  \mathcal{N}((M-\eta M/2)(B-\mu_{max}) \pm \eta M \mu_{max} \ , \ M((B-\mu_{max})^2 + (N^{\alpha}-2)\mu_{max}^2)
%\end{array}\]
%
%The probability that event $\mathcal{E}_2$ happens, given that $\mathcal{E}_1$ doesn't happen is given by
%\[ P(\mathcal{E}_2 / \bar{\mathcal{E}}_1 ) = Pr(c_2 > c_1) \]
%
%%If $c_1$ and $c_2$ are independent, then 
%
%%\[c1-c2 \sim \mathcal{N}(M(B-2\sqrt{B\log5n})\ ,\ 8MBN^\alpha\log5N +  B^2 ) \]
%
%\[ P(\mathcal{E}_2 / \bar{\mathcal{E}}_1 ) \leq Q \left( \sqrt{\frac{((M-\eta M/2)(B-\mu_{max}) \pm \eta M \mu_{max})^2}{M(B-\mu_{max})^2 + M(N^{\alpha}-2)\mu_{max}^2)}} \right) \]
%
%where $\mu_{max} = 2\sqrt{B \log 5 N^{\alpha}} $. If $B = O(\log 5 N^{\alpha})$, then the above equation reduces to
%
%\[ P(\mathcal{E}_2 / \bar{\mathcal{E}}_1 ) \leq Q \left( \sqrt{\frac{M(1-3\eta/2)^2}{1 + 4~ (N^{\alpha}-2)}} \right) \]
%
%If $\mu > \alpha$, the error vanishes.
\section{Sample and Computational Complexity}
\label{Sec:Complexity}
In this section, we will analyze the sketching complexity which is the  number of samples we access from the sketch of the signal $\xv$ stored in the database and the computational complexity as a function of the system parameters.

\subsection{\bf Sample Complexity}\label{subsec:SampleComplexity}
In each branch of the RSDIFT framework we down-sample the $N$ samples by a factor of $\frac{N}{f_i}$ to get $f_i\approx N^{\alpha}$ samples. We repeat this for a random shift in each branch for $B=O(\log N)$ branches in each stage thus resulting in a total of $O(N^{\alpha}\log N)$ samples per block per stage. We repeat this for $d = \frac{1}{1-\alpha}$ such stages resulting in a total of $dN^{\alpha}\log N$ samples per block. So, the total number of samples is given by
\begin{align*}
S&= O \left(dN^{\alpha}\log N\right) =   O(N^{1-\mu}\log N)
\end{align*}


\subsection{\bf Computational Complexity} \label{subsec:ComputationComplexity}

As described in Eq.~\ref{eqn:Rxy_fourier}, the computation of $\rv$ involves three steps:
\begin{enumerate}
	\item  Operation - \RNum{1}:
	Since we assume that the sketch of database $\xv$, $\mathcal{F}_{N}\{\xv\}$, is pre-computed, we do not include this in computational complexity.

	\item  Operation - \RNum{2}:
	As described in Sec.~\ref{subsec:skteches}, in each branch $(i,j)$, we use a folding based technique to compute the sketch of $\yv$, $\mathcal{F}_{N}\{\yv'\}$ at points in the set $\mc{S}_{i,j}$. The folding technique involves two steps: folding and adding (aliasing) which has a complexity of $O(M)$ computations , and computing $f_i$-point IDFTs that takes $O(N^\alpha \log N^{\alpha})$ computations. So, for a total of $dB$ branches the number of computations in this step is given by
	
	\begin{align*}
	 C_{\RNum{2}} \ &= ~  dB ~
	( \underset{\text{Folding} }{\underbrace{N^{\mu}}} + \ \
	\underset{\text{Shorter FFTs} }{\underbrace{N^{\alpha} \ \log N^{\alpha}}} \ )\\
	&= ~O(\max(N^{1-\mu}\log^2 N ,N^{\mu}\log N)).
	\end{align*}
	
	{\textit{Note:}} Folding and adding, for each shift, involves adding $N^{1-\alpha}$ vectors of length $N^{\alpha}$. We know that the length of the query is $M =N^{\mu}$, i.e., the number of non-zero elements in $\yv$ (zero-padded version of the query) is $M$ and hence we only need to compute $M$ additions instead of length of the vector $N$.

	\item  Operation - \RNum{3}:
	 Computing $\mathcal{F}_{N}^{-1}\{ \xv \}$ involves two parts:\\ RSIDFT framework and the decoder. The RSIDFT framework involves computing smaller $f_i$ point IDFTs, which takes approximately $O(N^{\alpha} \log N^{\alpha})$ computations in each branch. For a total of $dB$ branches, we get a complexity of $O(dB N^{\alpha}$ $\log N^{\alpha})$. In the decoding process, the dominant computation is from position identification. Each position identification process involves correlating the observation vector of length $B$ with $\frac{N}{f_i} \approx N^{1-\alpha}$ column vectors, which amounts to $B N^{1-\alpha}$ computations. There will be a maximum of $dL$ such position identifications, which gives a complexity of$O(dLBN^{1-\alpha} )$. Now, plugging in $\alpha = 1-\mu$ (condition for vanishing probability of error) the total number of computations involved in this step, $C_{\RNum{1}}$, is given by
	
	\begin{align*}	
	C_{\RNum{3}} \ &=  {d B}  \left (
	\underset{\text{Shorter IFFTs /block/stage} }{\underbrace{ O(N^{\alpha}  \log N^{\alpha})}} \hspace{-3pt}+ \underset{\text{Correlations} }{\underbrace{ L~N^{1-\alpha}}} \right )\\
	&=  O(\max(N^{1-\mu}\log^2 N ,N^{\mu+\lambda}\log N)).
	\end{align*}

\end{enumerate}

Thus, the total number of computations, $C = \max\{C_{\RNum{2}},C_{\RNum{3}}\} $, is given by
  \begin{align*}
  C ~ = O\left(\max\{N^{1-\mu}\log^2 N, N^{\mu+\lambda}\log N \}\right)
  \end{align*}

\section{Simulation Results} \label{sec:Simulation_Results}


\begin{figure*}[ht]
		\begin{tabular}{cc}
			\subfloat[$M=10^5(\mu=0.41), \tilde{N}=10^7, ~G=10^{5}$]{\resizebox{0.40\textwidth}{!}{% This file was created by matlab2tikz.
%
%The latest updates can be retrieved from
%  http://www.mathworks.com/matlabcentral/fileexchange/22022-matlab2tikz-matlab2tikz
%where you can also make suggestions and rate matlab2tikz.
%
\begin{tikzpicture}

\begin{axis}[%
width=2.521in,
height=1.566in,
at={(0.758in,0.481in)},
scale only axis,
xmin=100,
xmax=800,
xlabel={Sample/Computational Gain},
ymode=log,
ymin=1e-05,
ymax=0.1,
yminorticks=true,
ylabel={Prob of Missing a Match},
axis background/.style={fill=white}
]
\addplot [color=red,solid,mark=*,mark options={solid},forget plot]
  table[row sep=crcr]{%
167.8088	1e-07\\
218.0519	0.0005965\\
274.9922	0.0006834\\
338.1745	0.0012\\
484.384	0.027\\
750.2035	0.047\\
};
\end{axis}
\end{tikzpicture}% }}&
			\subfloat[$M=10^3(\mu=0.25),~ \tilde{N}=10^6, ~G=10^{6}$]{\resizebox{0.40\textwidth}{!}{% This file was created by matlab2tikz.
%
%The latest updates can be retrieved from
%  http://www.mathworks.com/matlabcentral/fileexchange/22022-matlab2tikz-matlab2tikz
%where you can also make suggestions and rate matlab2tikz.
%
\begin{tikzpicture}

\begin{axis}[%
width=2.521in,
height=1.566in,
at={(0.758in,0.481in)},
scale only axis,
xmin=2,
xmax=18,
xlabel={Sample/Computational Gain},
ymode=log,
ymin=1e-05,
ymax=1,
yminorticks=true,
ylabel={Prob of Missing a Match},
axis background/.style={fill=white}
]
\addplot [color=red,solid,mark=*,mark options={solid},forget plot]
  table[row sep=crcr]{%
2.0985	4e-07\\
3.2978	0.004\\
3.9974	0.004\\
4.7636	0.006\\
5.9963	0.017\\
7.3622	0.038\\
9.4944	0.054\\
17.4904	0.197\\
};
\end{axis}
\end{tikzpicture}% }}
		\end{tabular}
		
		\caption{Plots of probability of missing a match vs. sample gain for exact matching of a query of length $M$ from a equiprobable  binary \{+1,-1\} sequence of length $N= 10^{12}$, divided into $G$ blocks each of length $\tilde{N}$. The substring was simulated to repeat in $L=10^6$($\lambda=0.5$) locations uniformly at random.} \label{Fig:Simulation Results}
\end{figure*}

\subsection{Synthetic Dataset}
Simulations\footnote{Code available publicly in https://github.com/tjnagaraj/Pattern-Matching } were carried out to test the performance of RSIDFT framework for exact matching scenario on a database of length $N=10^{12}$ for two different query lengths $M=10^5$ ($\mu = 0.41$) and $M=10^3$ ($\mu = 0.25$). The database was generated as a equiprobable $\{+1,-1\}$ sequence of length $N$. A substring of length $M$ from the generated database is presented as a query. Also the chosen query was repeated at $L=10^6$ randomly chosen locations in the database.

The sample gain, defined as the ratio of $N$ to the number of samples used from the sketch of database, was varied and the probability of RSIDFT framework to miss a match ($P_e$), as defined below, was measured.
\[P_e = \frac{\text{\# of correctly identified locations}}{L} \]   
The plots of $P_e$ vs. sample gain, is presented in Fig.~\ref{Fig:Simulation Results} for two different query lengths: $M=10^5~(\mu=0.41)$ in Fig~\ref{Fig:Simulation Results}(a) and $M=10^3~(\mu=0.25)$ in Fig~\ref{Fig:Simulation Results}(b). As can be inferred from the plots we achieve a sample gain of 200-500 (depending on the tolerable error probability) for the query length corresponding to  $\mu=0.41$ and a sample gain of $2$-$8$ for $\mu=0.25$. This sample gain results from an average number of samples per branch $f_i \approx 9.25 \times10^7 $ ($\alpha=0.66$) for $\mu=0.41$, and  $f_i \approx 6.94\times10^9 $ ($\alpha=0.82$) for $\mu=0.25$. The trend in the results almost matches with the theoretical findings of $\alpha = 1-\mu$. We also notice a sharp threshold in the sample gain, below which the RSIDFT framework succeeds with very high probability. 

\subsection{Real Dataset}
We did some preliminary tests on a real world dataset to test the algorithm's performance for signals that do not satisfy our i.i.d model assumption. We simulated the performance of our algorithm for an audio clip. Vector $\xv$ corresponded to a 100 second clip sampled at 48KHz ($N=4800000$) and $\yv$ was a sub-string that corresponded to a 3 second substring. Our algorithm required only a sketch size of $48000$ FFT coefficients providing a 100 times reduction over linear time techniques. Although, the i.i.d assumption is violated, our algorithm still provides a good performance improvement over a linear scheme. 

 		
\appendix
\section{Chernoff Bounds}


%\begin{lemma}[Hoeffding tail bound]% for bounded random variables]
%\label{Lem:Chernoff}
%Let $X_1, X_2,\ldots, X_n$ be a sequence of independent random variables such that $X_i$ has mean $\mu_i$ and sub-Gaussian parameter $\sigma_i$. Then for any $\delta>0$:
%\begin{align*}
%\text{\textbf{Upper Tail}}: ~&\mbb{P}\left[\sum \left(X_i-\mu_i\right)\geq \delta\right]\leq \exp\left\lbrace-\frac{\delta^2}{2\sum \sigma_i^2}\right\rbrace\\
%\textbf{Lower Tail}: ~&\mbb{P}\left[\sum \left(X_i-\mu_i\right)\leq -\delta\right]\leq \exp\left\lbrace-\frac{\delta^2}{2\sum \sigma_i^2}\right\rbrace
%\end{align*}
%Note that for bounded random variables $X_i\in [a,b]$ the sub-Gaussian parameter is $\sigma_i=\frac{b-a}{2}$ whereupon the upper tail Hoeffding bound can be simplified to
%\begin{align}
%%\text{\textbf{Upper Tail}}: ~&
%\mbb{P}\left[\sum_{i=1}^{n} \left(X_i-\mu_i\right)\geq \delta\right]\leq \exp\left\lbrace-\frac{2\delta^2}{n(b-a)^2}\right\rbrace.
%%\textbf{Lower Tail}: ~&\mbb{P}\left[\sum_{i=1}^{n} \left(X_i-\mu_i\right)\leq -\delta\right]\leq \exp\left\lbrace-\frac{2\delta^2}{n\sum(b-a)^2}\right\rbrace
%\label{Eqn:HoeffdingBoundedRV}
%\end{align}
%Similarly the lower tail bound can be simplified.
%\end{lemma}


\begin{lemma}[Tail bounds for noise terms]
	\label{Lem:tailbounds}
Let us consider $r[\theta_0],r[\theta_1],\ldots ,r[\theta_{g_i-1}]$ where $\theta_j=\theta_0+jf_i$ and $\theta_j \notin \{\tau_1,\ldots, \tau_L\}$ is not one of the matching positions for any $j$. Then for any $\delta>0$:\\
{\bf Upper Tail:}
\begin{align*}
\mbb{P}\left[ \left(\frac{1}{M}\sum\limits_{j\in[g_i]}\sum\limits_{k\in[M]}x[\theta_j+k]y[k]\right)\geq \delta\right]\leq \exp\left\lbrace-\frac{M\delta^2}{2g_i}\right\rbrace
\end{align*}
{\bf Lower Tail:}
\begin{align*}
\mbb{P}\left[ \left(\frac{1}{M}\sum\limits_{j\in[g_i]}\sum\limits_{k\in[M]}x[\theta_j+k]y[k]\right)\geq \delta\right]\leq \exp\left\lbrace-\frac{M\delta^2}{2g_i}\right\rbrace\\
\end{align*}
Recall that $[g_i]$ is used to denote the set $\{0,1,\ldots,g_i-1\}$.
\end{lemma}
\begin{proof}
The detailed proof is provided in our longer version \cite{nagaraj2017pattern} (Lemma 7)
\end{proof}
%-------------------------------------------------
%\begin{proof}
%Since $\theta_j$ is not one of the matching positions for any $j$, $x[\theta_j+k]\neq y[k]$ and more importantly $x[\theta_j+k]\indep y[k] ~\forall j,k$. This implies that $x[\theta_j+k]y[k]=\pm 1$ with equal probability and $\mbb{E}[x[\theta_i+k]y[k]]=0$. Let the set of random variables corresponding to a position $\theta_j$ be $S_{j}\coleq \{x[\theta_j+k]y[k],k\in[M]\}$. It is clear that the random variables in the set $S_j$ are all independent with respect to each other due to our i.i.d assumption on the database $\xv$ and $\theta_j$ being a non-matching position. For the case of $\mu<\alpha$ we have $M<f_i$ for large enough $N$ thus resulting in non-overlapping parts of $\xv$  participating in the correlation coefficients $r[\theta_i]$ and $r[\theta_j]$. Hence it can be shown that $S_i\indep S_j~~\forall i,j$ and we can apply the bounds from Eqn. \eqref{Eqn:HoeffdingBoundedRV} achieve the required result.
%
%For the case of $\mu\geq\alpha$, $M>f_i$ for large enough $N$ which results in a coefficient $x[j]$ participating in multiple correlation coefficients $r[\theta_j]$. Therefore we pursue an alternate method of proof by defining 
%$$
%p_{j,l}\coleq x[\theta_j+l]\sum_{k\in[\frac{M}{f_i}]} y[l+kf_i] ~\text{ for } l\in[f_i],
%$$
%where $\theta_j=\theta_0+jf_i$. W.L.O.G we assume that $f_i$ divides $M$ evenly although the proof can be extended on similar lines for the case where $f_i$ does not divide $M$ evenly. Now we can show that the required sum
%\begin{align*}
%\sum\limits_{j\in[g_i]}\sum\limits_{l\in[M]}x[\theta_j+l]y[l]=\sum_{j\in[g_i]}\sum_{l\in[f_i]}p_{j,l}.
%\end{align*}
%From the above equivalent representation of the required sum, we need the following to achieve the required result:
%\begin{itemize}
%\item $p_{j,l}$ is sub-Gaussian with parameter $\sigma_i=\sqrt[•]{\frac{M}{f_i}}$ since, from Eqn. \eqref{Eqn:HoeffdingBoundedRV}, 
%\begin{align*}
%\mbb{P}\left[ p_{j,l}\leq \delta\right] \leq \exp\left\lbrace \frac{-\delta^2}{2\frac{M}{f_i}}\right\rbrace
%\end{align*} 
%\item The random variables $\{p_{j,l},~\forall j,l\}$ are independent of each other due to the i.i.d assumption on the database
%\end{itemize}
% Now we can apply the tail bounds for sum of $g_i f_i$ sub-Gaussian random variables each with sub-Gaussian parameter $\sqrt{\frac{M}{f_i}}$ from Lem. \ref{Lem:Chernoff} and arrive at the required result.
%\end{proof}
%---------------------------------------------------------------------


\section{Bin Classification Errors}
\label{Append:BinClassif}
We employ classification rules based only on the first element of the measurement vector at bin $(i,j)$ which can be given by
\begin{align}
Z[1]=\begin{cases}
\sum\limits_{\ell=0}^{g_{i}-1}\sum\limits_{k=0}^{M-1} n_{l,k}  & ~~\text{ if } ~~ \msc{H}=\msc{H}_z\label{Eqn:BinCombination}\\
\vspace{\vgap}
M_1+\sum\limits_{\ell=0}^{g_{i}-2}\sum\limits_{k=0}^{M-1} n_{l,k}  & ~~\text{ if } ~~ \msc{H}=\msc{H}_s\\
\vspace{\vgap}
M_1+M_2+\sum\limits_{\ell=0}^{g_{i}-3}\sum\limits_{k=0}^{M-1} n_{l,k}  & ~~\text{ if } ~~ \msc{H}=\msc{H}_d\\
\end{cases}
\end{align}
where $n_{l,k}=x[\theta_{\ell}+k]y[k]$ and $\theta_{\ell}\notin\{\tau_1,\tau_2,\ldots,\tau_L\}$. Also for the case of exact matching $M_1=M_2=M$ whereas in the case of approximate matching the values of $M_1,M_2\in[M(1-2\eta):M]$.

\begin{lemma}[zero-ton]
\label{Lem:ZerotonClassif}
Given that the bin $(i,j)$ is a zero-ton, the classification error can be bounded by
\begin{align*}
\mbb{P}[\mc{E}_1|\msc{H}_z]\leq e^{-\frac{N^{\mu+\alpha-1}(1-2\eta)^2}{8}}
\end{align*}
\end{lemma}
\begin{proof}
The above expression can be derived by observing that a bin is not classified as zero-ton if $\frac{Z[1]}{M}\geq\frac{1-2\eta}{2}$. Let us denote the probability of this event as $p_{z1}$ which can be bounded as:
\begin{align*}
p_{z1}=&\mbb{P}\left[\frac{Z[1]}{M}\geq\frac{1-2\eta}{2}\right]\\
&\leq e^{-\frac{Mg_i(1-2\eta)^2}{8g_i^{2}}} \approx e^{-\frac{N^{\mu+\alpha-1}(1-2\eta)^2}{8}}
\end{align*} 
where the second bound is due to Eqn. \eqref{Eqn:BinCombination} and Lemma.~\ref{Lem:tailbounds}. The approximation in the third line is from our design that all the $g_i$ are chosen such that $g_i\approx N^{1-\alpha}$ and $M=N^{\mu}.$
\end{proof}

\begin{lemma}[singleton]
\label{Lem:SingletonClassif}
Given that the bin $(i,j)$ is a singleton, the classification error can be bounded by
\begin{align*}
\mbb{P}[\mc{E}_1|\msc{H}_s]\leq 2e^{-\frac{N^{\mu+\alpha-1}(1-4\eta)^2}{16}}
\end{align*}
\end{lemma}
%----------------------------------------------------------------------------------
%\begin{proof}
%We observe that a bin is not classified as singleton if $\frac{Z[1]}{M}\leq\frac{1-2\eta}{2}$ or $\frac{Z[1]}{M}\geq\frac{3-4\eta}{2}$. Let us denote the probability of the two events as $p_{s1}$ and $p_{s2}$ respectively which can be bounded as:
%\begin{align*}
%p_{s1}&=\mbb{P}\left[\frac{1}{M}\sum\limits_{\ell=0}^{g_{i}-2}\sum\limits_{k=0}^{M-1} n_{l,k}\leq\frac{1-2\eta}{2}-\frac{M_1}{M}\right]\\
%&\leq \mbb{P}\left[\frac{1}{M}\sum\limits_{\ell=0}^{g_{i}-2}\sum\limits_{k=0}^{M-1} n_{l,k}\leq-\frac{1-2\eta}{2}\right]\\
%&\leq e^{-\frac{Mg_i(1-2\eta)^2}{16g_i^{2}}}\\
%&\approx e^{-\frac{N^{\mu+\alpha-1}(1-2\eta)^2}{16}}
%\end{align*} 
%where we used  {\it lower tail} of Lemma \ref{Lem:Chernoff} and $g_i\approx N^{1-\alpha}$ and the lower bound on $\frac{M_1•}{M}\geq (1-2\eta)$. Similarly $p_{s2}$ can be upper bounded by:
%% $p_{s2|\leq e^{-\frac{N^{\mu-\alpha}(1-2\eta)^2}{16}}$
%\begin{align*}
%p_{s2}&=\mbb{P}\left[\frac{1}{M}\sum\limits_{\ell=0}^{g_{i}-2}\sum\limits_{k=0}^{M-1} n_{l,k}\geq\frac{3-4\eta}{2}-\frac{M_1}{M}\right]\\
%&\leq \mbb{P}\left[\frac{1}{Mg_i}\sum\limits_{\ell=0}^{g_{i}-2}\sum\limits_{k=0}^{M-1} n_{l,k}\geq-\frac{1-4\eta}{2g_i}\right]\\
%&\approx e^{-\frac{N^{\mu+\alpha-1}(1-4\eta)^2}{8}}
%\end{align*} 
%Thus the overall probability of error for classifying a singleton can be obtained by combining $p_{s1}$ and $p_{s2}$.
%\end{proof}
%---------------------------------------------------------------------------------
\begin{lemma}[double-ton]
\label{Lem:DoubletonClassif}
Given that the bin $(i,j)$ is a double-ton, the classification error can be bounded by
\begin{align*}
\mbb{P}[\mc{E}_1|\msc{H}_d]\leq 2e^{-\frac{N^{\mu+\alpha-1}(1-6\eta)^2}{16}}
\end{align*}
\end{lemma}
%---------------------------------------------------------------------------------
%\begin{proof}
%We observe that a bin is not classified as double-ton if $\frac{Z[1]}{M}\leq\frac{3-4\eta}{2}$ or $\frac{Z[1]}{M}\geq\frac{5-6\eta}{2}$. Let us denote the probability of these two events as $p_{d1}$ and $p_{d2}$ respectively which can be bounded similar to Lemma ~\ref{Lem:SingletonClassif}.
%\begin{align*}
%p_{d1}&=\mbb{P}\left[\frac{1}{M}\sum\limits_{\ell=0}^{g_{i}-3}\sum\limits_{k=0}^{M-1} n_{l,k}\leq\frac{3-4\eta}{2}-\frac{M_1+M_2}{M}\right]\\
%&\leq \mbb{P}\left[\frac{1}{M}\sum\limits_{\ell=0}^{g_{i}-3}\sum\limits_{k=0}^{M-1} n_{l,k}\leq-\frac{1-4\eta}{2}\right]\\
%&\leq e^{-\frac{M(g_i-2)(1-4\eta)^2}{16(g_i-2)^{2}}}\\
%&\approx e^{-\frac{N^{\mu+\alpha-1}(1-4\eta)^2}{16}}.
%\end{align*} 
%Similarly $p_{d2}$ can be bounded as 
%\begin{align*}
%p_{d2}&=\mbb{P}\left[\frac{1}{M}\sum\limits_{\ell=0}^{g_{i}-3}\sum\limits_{k=0}^{M-1} n_{l,k}\geq\frac{5-6\eta}{2}-\frac{M_1+M_2}{M}\right]\\
%&\leq \mbb{P}\left[\frac{1}{M}\sum\limits_{\ell=0}^{g_{i}-3}\sum\limits_{k=0}^{M-1} n_{l,k}\geq\frac{1-6\eta}{2}\right]\\
%&\leq e^{-\frac{M(g_i-2)(1-6\eta)^2}{8(g_i-2)^{2}}}\\
%&\approx e^{-\frac{N^{\mu+\alpha-1}(1-6\eta)^2}{8}}
%\end{align*} 
%where we use the lower bounds $M_1,M_2\leq M$.
%\end{proof}
%--------------------------------------------------------------------------------

\begin{lemma}[multi-ton]
\label{Lem:MultitonClassif}
Given that the bin $(i,j)$ is a multi-ton, the classification error can be bounded by
\begin{align*}
\mbb{P}[\mc{E}_1|\msc{H}_m]\leq e^{-\frac{N^{\mu+\alpha-1}(1-6\eta)^2}{16}}
\end{align*}
\end{lemma}

\begin{proof}
	The proof for Lemmas ~\ref{Lem:SingletonClassif},~\ref{Lem:DoubletonClassif}, ~\ref{Lem:MultitonClassif} follows similar lines as in Lemma~\ref{Lem:ZerotonClassif}. For a detailed proof refer to our longer version \cite{nagaraj2017pattern} (Lemmas 9,10,11).
\end{proof}
%-------------------------------------------------------------------------------
%\begin{proof}
%We observe that a bin is not classified as multi-ton if $\frac{Z[1]}{M}\leq\frac{5-6\eta}{2}$. Let us denote the probability of this event as $p_{m1}$ which can be bounded as:
%\begin{align*}
%p_{m1}&=\mbb{P}\left[\frac{1}{M}\sum\limits_{\ell=0}^{g_{i}-3}\sum\limits_{k=0}^{M-1} n_{l,k}\leq\frac{5-6\eta}{2}-\frac{M_m}{M}\right]\\
%&\leq \mbb{P}\left[\frac{1}{M}\sum\limits_{\ell=0}^{g_{i}-m}\sum\limits_{k=0}^{M-1} n_{l,k}\leq-\frac{1-6\eta}{2}\right]\\
%&\leq e^{-\frac{M(g_i-m)(1-4\eta)^2}{16(g_i-m)^{2}}}\\
%&\leq e^{-\frac{M(1-6\eta)^2}{16 n_i}}\\
% &\approx e^{-\frac{N^{\mu+\alpha-1}(1-6\eta)^2}{16}}.
%\end{align*} 
%\end{proof}
%-------------------------------------------------------------------------------
\section{Position Identification}
\label{Append:PositionIdentif}
We will analyze the singleton identification in two separate cases:
\begin{itemize}
\item $\mc{E}_{21}$: Event where the position is identified incorrectly when the bin is classified  correctly a singleton
\item $\mc{E}_{22}$: In the case of approximate matching, event where the position is identified incorrectly when the bin is originally a double-ton and one of the non-zero variable nodes has already been peeled off
\end{itemize}

\begin{definition}[Mutual Incoherence]
	The mutual incoherence $\mu_{\text{max}}( \mb{W})$ of a matrix $\mb{W} = [\wv_1 ~ \wv_2 ~ \cdots \wv_i \cdots \wv_N ]$ is defined as 
	
	\[\mu_{\text{max}}(\mb{W}) \defeq \max \limits_{\forall i \neq j} \frac{|\wv_i^{\dagger} \wv_j |}{||\wv_i || . ||\wv_j ||} \]
\end{definition}

\begin{lemma}[Mutual Incoherence Bound for sub-sampled IDFT matrix  [\cite{pawar2014robust},Proposition~A.1]
\label{lemma:MutualCoherence}
	The mutual incoherence $\mu_{\text{max}}$ $(\mb{W_{i,k}})$ of the sensing matrix $\mb{W}_{i,k}$ (defined in Eq.~\ref{Eqn:Sensing Matrix}), with $B$ shifts, is upper bounded by
	\[ \mu_{\text{max}} < 2\sqrt{\frac{\log(5N)}{B}} \] 
	
\end{lemma}
\begin{proof}
	The proof follows similar lines as the proof for Lemma V.3. in \cite{pawar2014robust}.
\end{proof}
 
\begin{lemma} \label{Lem:Pos1}
For some constant $c_1 \in \mathbb{R}$ and the choice of $B=4c_1^2\log 5N$,  the probability of error in identifying the position of a singleton at any bin $(i,j)$ can be upper bounded by
\begin{align*}
\mbb{P}[\mc{E}_{21}]\leq \exp\left\lbrace-\frac{N^{\mu+\alpha-1}(1-2\eta)^2(c_1^2-1)}{8(c_1^2+1)}\right\rbrace
\end{align*}
\end{lemma}
%-------------------------------------------------------------------------------
%\begin{proof}
%	
%	Let $j_p$ be the variable node participating in the singleton $(i,j)$. Then the observation vector $\zv_{i,j}$ is given by
%	\begin{align*}
%	\underline{z}_{i,j} &= \begin{bmatrix}
%	\wv_{j_{1}},\wv_{j_2}, & \cdots   & \wv_{j_p}, &\cdots \ &\wv_{j_{g_i}}
%	\end{bmatrix} \times
%	\begin{bmatrix}
%	n_{1} \\
%	\vdots \\
%	r[j_p]\\
%	\vdots\\
%	n_{j} \\
%	\vdots\\
%	n_{g_i}\\
%	\end{bmatrix}\\
%	&= r[j_p] ~ \wv_{j_p}+ \sum_{k \neq p}n_k \wv_{j_k} \\
%	\end{align*}
%	where for convenience we use a simpler notation $j_k=j+(k-1)\frac{N}{f_i}, \wv_{j_k}=\wv^{j_k}$ as defined in Eq. and $n_{l}=\sum\limits_{k=0}^{M-1}x[\theta_{\ell}+k]y[k]$ as defined in Eq. \eqref{Eqn:BinCombination}.
%	
%	The estimated position $\hat{p}$ is given by
%	\begin{align}
%	\label{Eqn:SingletonBinCombination}
%	\hat{p}= \underset{l}{\argmax}~~ \frac{\wv_{j_l}^{\dagger}\underline{z}_{i,j}}{B}
%	\end{align}
%	where $\dagger$ denotes the conjugate transpose of the vector. Also note that $|| \wv_{j_k}||=B$ for any $j$ and $k$.  From Eq. \eqref{Eqn:SingletonBinCombination} we observe that the position is wrongly identified when $\exists p'$ such that
%	\begin{align*}
%	&r[j_p] + \frac{1}{B}\sum_{k \neq p} n_k 	\wv_{j_p}^{\dagger}\wv_{j_k} \leq \frac{r[j_p]}{B} ~ \wv_{j_{p'}}^{\dagger}\wv_{j_p}+ n_{p'}+\frac{1}{B}\sum_{k \neq p,p'}n_k\wv_{j_{p'}}^{\dagger} \wv_{j_k} \\
%	&\leftrightarrow \sum_{k \neq p,p'}\alpha_k n_k+\beta n_{p'}\geq  r[j_p]\left(1-\frac{\wv_{j_{p'}}^{\dagger}\wv_{j_p}}{B}\right)\geq M(1-2\eta)(1-\mu_{\text{max}})
%	\end{align*}
%	where $\alpha_k$ and $\beta$ are constants and can be shown to be in the range $\alpha_k\in[-2\mu_\text{max},2\mu_\text{max}]$ and $\beta\in[1-\mu_\text{max},1+\mu_\text{max}]$. Now using the bound given Chernoff Lemma in Lem.~\ref{Lem:tailbounds} we obtain
%	\begin{align*}
%	\mbb{P}[\mc{E}_{21}]&\leq \exp\left\lbrace-\frac{2M(1-2\eta)^2(1-\mu_{\text{max}})^2}{16g_i\mu^2_{\max}+4(1+\mu_{\max})^2}\right\rbrace\\
%	&\leq\exp\left\lbrace-\frac{2M(1-2\eta)^2(1-\mu_{\text{max}})^2}{16(g_i\mu^2_{\max}+1)}\right\rbrace\\
%	&\leq\exp\left\lbrace-\frac{2M(1-2\eta)^2(c_1-1)^2}{16(g_i+c_1^2)}\right\rbrace\\
%	&\approx\exp\left\lbrace-\frac{N^{\mu+\alpha-1}(1-2\eta)^2(c_1^2-1)}{8(c_1^2+1)}\right\rbrace\\
%	\end{align*}
%	The second inequality follows by the definition of  $\mu_{\text{max}} \leq 1$.  We choose $B=4c_1^2\log 5N$, and substituting $\mu_{\max}\leq 2\sqrt{\frac{\log 5N}{B}} = 1/c_1$ (Lemma \ref{lemma:MutualCoherence}) we get the third inequality.
%\end{proof}

\begin{lemma}\label{Lem:Pos2}
For some constant $c_1 \in \mathbb{R}$ and the choice of $B=4c_1^2\log 5N$, the probability of error in identifying the position of second non-zero variable node at a double-ton at any bin $(i,j)$, given that the first position identification is correct, can be upper bounded by
	\begin{align*}
		\mbb{P}[\mc{E}_{22}]\leq \exp\left\lbrace-\frac{N^{\mu+\alpha-1} ~ (c_1(1 - 2\eta) - 1)^2}{8(1+ c_1^2)}\right\rbrace
	\end{align*}
\end{lemma}

\begin{proof}
	The proof for Lemmas ~\ref{Lem:Pos1} and ~\ref{Lem:Pos2} can be found in our longer version\cite{nagaraj2017pattern} (Lemma 14,15).
\end{proof}
%---------------------------------------------------------------------------------
%\begin{proof}
%	
%	{\bf $\mc{E}_{22}$:}
%	
%	Let $j_p$ and $j_{\tilde{p}}$ be the two variable nodes participating in the doubleton $(i,j)$. Then the observation vector $\zv_{i,j}$ is given by 
%	
%	\begin{align*}
%		\underline{z}_{i,j} &= \begin{bmatrix}
%			\wv_{j_{1}},\wv_{j_2}, & \cdots   & \wv_{j_p}, &\cdots \ &\wv_{j_{g_i}}
%		\end{bmatrix} \times
%		\begin{bmatrix}
%			n_{1} \\
%			\vdots \\
%			r[j_p]\\
%			\vdots\\
%			n_{j} \\
%			\vdots\\
%			r[j_{\tilde{p}}]\\
%			\vdots\\
%			n_{g_i}\\
%		\end{bmatrix}\\
%		&= r[j_p] ~ \wv_{j_p} + r[j_{\tilde{p}}] ~ \wv_{j_{\tilde{p}}} + \sum_{k \neq p}n_k \wv_{j_k} \\
%	\end{align*}
%	
%	Let the contribution from $j_{\tilde{p}}$ be peeled off from the doubleton at some iteration, then we get
%	\[ \zv_{i,j} = r[j_p] ~ \wv_{j_p} + \frac{e_1}{B} ~ \wv_{j_{\tilde{p}}} + \sum_{k \neq p}n_k \wv_{j_k}\]
%	
%	where $e_1 \in[-\eta M, \eta M]$ is an extra error term induced due to peeling off.
%	
%	Now the estimated second position $\hat{p}$ is calculated using Eq. \eqref{Eqn:SingletonBinCombination}. We can observe that the position is wrongly identified when $\exists p'$ such that
%	\[ \frac{\wv_{j_p}^{\dagger}\underline{z}_{i,j}}{B} \leq \frac{\wv_{j_{p'}}^{\dagger}\underline{z}_{i,j}}{B}\]
%	\begin{align*}
%		&\implies r[j_p] + \frac{1}{B}\sum_{k \neq p, \tilde{p}} n_k 	\wv_{j_p}^{\dagger}\wv_{j_k} + \frac{e_1}{B} \wv_{j_p}^{\dagger}\wv_{j_{\tilde{p}}} \\ & \qquad \leq \frac{r[j_p]}{B} ~ \wv_{j_{p'}}^{\dagger}\wv_{j_p}+ n_{p'}+\frac{1}{B}\sum_{k \neq p,p',\tilde{p}}n_k\wv_{j_{p'}}^{\dagger} \wv_{j_k} + \frac{e_1}{B} \wv_{j_{p'}}^{\dagger}\wv_{j_{\tilde{p}}}
%	\end{align*}
%	\begin{align*}
%		&\leftrightarrow \sum_{k \neq p,p',\tilde{p}}\alpha_k n_k+ \beta n_{p'}  \geq  r[j_p]\left(1-\frac{\wv_{j_{p'}}^{\dagger}\wv_{j_p}}{B}\right) - \frac{2 \eta M}{B} \wv_{j_{p'}}^{\dagger}\wv_{j_{\tilde{p}}}
%	\end{align*}
%	\[~~\geq M(1-2\eta)(1-\mu_{\text{max}}) - 2 \eta M \mu_{\text{max}} = M(1 - 2\eta - \mu_{\text{max}})                         
%	\]
%	where $\alpha_k$ and $\beta$ are constants and can be shown to be in the range $\alpha_k\in[-2\mu_\text{max},2\mu_\text{max}]$ and $\beta\in[1-\mu_\text{max},1+\mu_\text{max}]$. Now using the bound given by Chernoff Lemma in Lem.~\ref{Lem:tailbounds} we obtain
%	\begin{align*}
%		\mbb{P}[\mc{E}_{22}]&\leq \exp\left\lbrace-\frac{2M(1 - 2\eta - \mu_{\text{max}})^2}{16g_i\mu^2_{\max}+4(1+\mu_{\max})^2}\right\rbrace\\
%		&\leq\exp\left\lbrace-\frac{2M(1 - 2\eta - \mu_{\text{max}})^2}{16(g_i\mu^2_{\max}+1)}\right\rbrace\\
%		&\leq\exp\left\lbrace-\frac{M(c_1(1 - 2\eta) - 1)^2}{8(g_i+c_1^2)}\right\rbrace\\
%		&\leq\exp\left\lbrace-\frac{N^{\mu+\alpha-1} ~ (c_1(1 - 2\eta) - 1)^2}{8(1+ c_1^2)}\right\rbrace\\
%	\end{align*}
%	where for the choice of $B=4c_1^2\log 5N$, $\mu_{\max}\leq 2\sqrt{\frac{\log 5N}{B}} = 1/c_1$.
%	
%\end{proof}
%\appendices

\begin{lemma}[Chernoff Bound for bounded random variables]
\label{Lem:Chernoff}
Let $X_1, X_2,\ldots, X_n$ be a sequence of independent random variables such that $X_i \in\{-1, +1\}$ for all $i$ and $E[X_i]=\mu$ for all $i$. Then for any $\delta>0$:
\begin{align*}
\text{\textbf{Upper Tail}}: &\mbb{P}\left[\frac{1}{n}\sum \left(X_i-\mu\right)\geq \delta\right]\leq e^{-\frac{n\delta^2}{2}}\\
\textbf{Lower Tail}: &\mbb{P}\left[\frac{1}{n}\sum \left(X_i-\mu\right)\leq -\delta\right]\leq e^{-\frac{n\delta^2}{4}}
\end{align*}
\end{lemma}

\begin{lemma}[Variant of Hoeffding Bound]
\label{Lem:Chernoff2}
Let $X_1, X_2,\ldots, X_n$ be a sequence of independent random variables such that $X_i \in[a_i, b_i]$ and $E[X_i]=\mu$ for all $i$. Then for any $\delta>0$:
\begin{align*}
\text{\textbf{Upper Tail}}: &\mbb{P}\left[\sum \left(X_i-\mu\right)\geq \delta\right]\leq \exp\left\lbrace-\frac{2\delta^2}{\sum(b_i-a_i)^2}\right\rbrace
\end{align*}
\end{lemma}

\begin{remark}
\label{Lem:CorrelationCoefficient}
Let us consider $r[\theta_1],\ldots ,r[\theta_{f_i}]$ where $\theta_i \notin \{\tau_1,\ldots, \tau_L\}$ are not one of the matching positions. Then we can show that $\mbb{P}[x[\theta_i+k]y[k]=+1]$ with probability 1/2 and $\mbb{E}[x[\theta_i+k]y[k]]=0$. We also need to show that  the set of random variables  $\{x[\theta_i+k]y[k]: i\in\{1,2,\ldots f_i\},k\in[M]\}$ are independent. I was able to show this for the case of $M=3$. I don't see a simple way of extending this to the case of general $M$.
\end{remark}

\section{Bin Classification Errors}
\label{Append:BinClassif}
We employ classification rules based only on the first element of the measurement vector at bin $(i,j)$ which can be given by
\begin{align}
Z[1]=\begin{cases}
\sum\limits_{\ell=0}^{f_{i}-1}\sum\limits_{k=0}^{M-1} n_{l,k}  & ~~\text{ if } ~~ \msc{H}=\msc{H}_z\label{Eqn:BinCombination}\\
\vspace{\vgap}
M_1+\sum\limits_{\ell=0}^{f_{i}-2}\sum\limits_{k=0}^{M-1} n_{l,k}  & ~~\text{ if } ~~ \msc{H}=\msc{H}_s\\
\vspace{\vgap}
M_1+M_2+\sum\limits_{\ell=0}^{f_{i}-3}\sum\limits_{k=0}^{M-1} n_{l,k}  & ~~\text{ if } ~~ \msc{H}=\msc{H}_d\\
\end{cases}
\end{align}
where $n_{l,k}=x[\theta_{\ell}+k]y[k]$ and $\theta_{\ell}\notin\{\tau_1,\tau_2,\ldots,\tau_L\}$. Also for the case of exact matching $M_1=M_2=M$ whereas in the case of approximate matching the values of $M_1,M_2\in[M(1-2\eta):M]$.

\begin{lemma}[zero-ton]
\label{Lem:ZerotonClassif}
Given that the bin $(i,j)$ is a zero-ton, the classification error can be bounded by
\begin{align*}
\mbb{P}[\mc{E}_1|\msc{H}_z]\leq e^{-\frac{N^{\mu-\alpha}(1-2\eta)^2}{8}}
\end{align*}
\end{lemma}
\begin{proof}
The above expression can be derived by observing that a bin is not classified as zero-ton if $\frac{Z[1]}{M}\geq\frac{1-2\eta}{2}$. Let us denote the probability of this event as $p_{z1}$ which can be bounded as:
\begin{align*}
p_{z1}=&\mbb{P}\left[\frac{Z[1]}{Mf_i}\geq\frac{1-2\eta}{2f_i}\right]\\
&\leq e^{-\frac{Mf_i(1-2\eta)^2}{8f_i^{2}}}\\
&\approx e^{-\frac{N^{\mu-\alpha}(1-2\eta)^2}{8}}
\end{align*} 
where the second bound is due to Eqn. \eqref{Eqn:BinCombination} and Lemma.~\ref{Lem:Chernoff}. The approximation in the third line is from our design that all the $f_i$ are chosen such that $f_i\approx N^{\alpha}$ and $M=N^{\mu}.$
\end{proof}

\begin{lemma}[singleton]
\label{Lem:SingletonClassif}
Given that the bin $(i,j)$ is a singleton, the classification error can be bounded by
\begin{align*}
\mbb{P}[\mc{E}_1|\msc{H}_s]\leq 2e^{-\frac{N^{\mu-\alpha}(1-4\eta)^2}{16}}
\end{align*}
\end{lemma}
\begin{proof}
We observe that a bin is not classified as singleton if $\frac{Z[1]}{M}\leq\frac{1-2\eta}{2}$ or $\frac{Z[1]}{M}\geq\frac{3-4\eta}{2}$. Let us denote the probability of the two events as $p_{s1}$ and $p_{s2}$ respectively which can be bounded as:
\begin{align*}
p_{s1}&=\mbb{P}\left[\frac{1}{M}\sum\limits_{\ell=0}^{f_{i}-2}\sum\limits_{k=0}^{M-1} n_{l,k}\leq\frac{1-2\eta}{2}-\frac{M_1}{M}\right]\\
&\leq \mbb{P}\left[\frac{1}{M}\sum\limits_{\ell=0}^{f_{i}-2}\sum\limits_{k=0}^{M-1} n_{l,k}\leq-\frac{1-2\eta}{2}\right]\\
&\leq e^{-\frac{Mf_i(1-2\eta)^2}{16f_i^{2}}}\\
&\approx e^{-\frac{N^{\mu-\alpha}(1-2\eta)^2}{16}}
\end{align*} 
where we used  {\it lower tail} of Lemma \ref{Lem:Chernoff} and $f_i\approx N^{\alpha}$ and the lower bound on $\frac{M_1•}{M}\geq (1-2\eta)$. Similarly $p_{s2}$ can be upper bounded by:
% $p_{s2|\leq e^{-\frac{N^{\mu-\alpha}(1-2\eta)^2}{16}}$
\begin{align*}
p_{s2}&=\mbb{P}\left[\frac{1}{M}\sum\limits_{\ell=0}^{f_{i}-2}\sum\limits_{k=0}^{M-1} n_{l,k}\geq\frac{3-4\eta}{2}-\frac{M_1}{M}\right]\\
&\leq \mbb{P}\left[\frac{1}{Mf_i}\sum\limits_{\ell=0}^{f_{i}-2}\sum\limits_{k=0}^{M-1} n_{l,k}\geq-\frac{1-4\eta}{2f_i}\right]\\
&\approx e^{-\frac{N^{\mu-\alpha}(1-4\eta)^2}{8}}
\end{align*} 
Thus the overall probability of error for classifying a singleton can be obtained by combining $p_{s1}$ and $p_{s2}$.
\end{proof}

\begin{lemma}[double-ton]
\label{Lem:DoubletonClassif}
Given that the bin $(i,j)$ is a double-ton, the classification error can be bounded by
\begin{align*}
\mbb{P}[\mc{E}_1|\msc{H}_d]\leq 2e^{-\frac{N^{\mu-\alpha}(1-6\eta)^2}{16}}
\end{align*}
\end{lemma}
\begin{proof}
We observe that a bin is not classified as double-ton if $\frac{Z[1]}{M}\leq\frac{3-4\eta}{2}$ or $\frac{Z[1]}{M}\geq\frac{5-6\eta}{2}$. Let us denote the probability of these two events as $p_{d1}$ and $p_{d2}$ respectively which can be bounded similar to Lemma ~\ref{Lem:SingletonClassif}.
\begin{align*}
p_{d1}&=\mbb{P}\left[\frac{1}{M}\sum\limits_{\ell=0}^{f_{i}-3}\sum\limits_{k=0}^{M-1} n_{l,k}\leq\frac{3-4\eta}{2}-\frac{M_1+M_2}{M}\right]\\
&\leq \mbb{P}\left[\frac{1}{M}\sum\limits_{\ell=0}^{f_{i}-3}\sum\limits_{k=0}^{M-1} n_{l,k}\leq-\frac{1-4\eta}{2}\right]\\
&\leq e^{-\frac{M(f_i-2)(1-4\eta)^2}{16(f_i-2)^{2}}}\\
&\approx e^{-\frac{N^{\mu-\alpha}(1-4\eta)^2}{16}}.
\end{align*} 
Similarly $p_{d2}$ can be bounded as 
\begin{align*}
p_{d2}&=\mbb{P}\left[\frac{1}{M}\sum\limits_{\ell=0}^{f_{i}-3}\sum\limits_{k=0}^{M-1} n_{l,k}\geq\frac{5-6\eta}{2}-\frac{M_1+M_2}{M}\right]\\
&\leq \mbb{P}\left[\frac{1}{M}\sum\limits_{\ell=0}^{f_{i}-3}\sum\limits_{k=0}^{M-1} n_{l,k}\geq\frac{1-6\eta}{2}\right]\\
&\leq e^{-\frac{M(f_i-2)(1-6\eta)^2}{8(f_i-2)^{2}}}\\
&\approx e^{-\frac{N^{\mu-\alpha}(1-6\eta)^2}{8}}
\end{align*} 
where we use the lower bounds $M_1,M_2\leq M$.
\end{proof}

\begin{lemma}[multi-ton]
\label{Lem:MultitonClassif}
Given that the bin $(i,j)$ is a multi-ton, the classification error can be bounded by
\begin{align*}
\mbb{P}[\mc{E}_1|\msc{H}_m]\leq e^{-\frac{N^{\mu-\alpha}(1-6\eta)^2}{16}}
\end{align*}
\end{lemma}
\begin{proof}
We observe that a bin is not classified as multi-ton if $\frac{Z[1]}{M}\leq\frac{5-6\eta}{2}$. Let us denote the probability of this event as $p_{m1}$ which can be bounded as:
\begin{align*}
p_{m1}&=\mbb{P}\left[\frac{1}{M}\sum\limits_{\ell=0}^{f_{i}-3}\sum\limits_{k=0}^{M-1} n_{l,k}\leq\frac{5-6\eta}{2}-\frac{M_m}{M}\right]\\
&\leq \mbb{P}\left[\frac{1}{M}\sum\limits_{\ell=0}^{f_{i}-m}\sum\limits_{k=0}^{M-1} n_{l,k}\leq-\frac{1-6\eta}{2}\right]\\
&\leq e^{-\frac{M(f_i-m)(1-4\eta)^2}{16(f_i-m)^{2}}}\\
&\leq e^{-\frac{M(1-6\eta)^2}{16 f_i}}\\
 &\approx e^{-\frac{N^{\mu-\alpha}(1-6\eta)^2}{16}}.
\end{align*} 
\end{proof}

\section{Position Identification}
\label{Append:PositionIdentif}
We will analyze the singleton identification in two separate cases:
\begin{itemize}
\item $\mc{E}_{21}$: Event where the position is identified incorrectly when the bin is classified  correctly a singleton
\item $\mc{E}_{22}$: In the case of approximate matching, event where the position is identified incorrectly when the bin is originally a double-ton and one of the non-zero variable nodes has already been peeled off
\end{itemize}

\begin{definition}[Mutual Incoherence]
	The mutual incoherence $\mu_{\text{max}}( \mb{W})$ of a matrix $\mb{W} = [\wv_1 ~ \wv_2 ~ \cdots \wv_i \cdots \wv_N ]$ is defined as 
	
	\[\mu_{\text{max}}(\mb{W}) \defeq \max \limits_{\forall i \neq j} \frac{|\wv_i^{\dagger} \wv_j |}{||\wv_i || . ||\wv_j ||} \]
	
\end{definition}
\begin{lemma}[Mutual Incoherence Bound for sub-sampled IDFT matrix] \label{lemma:MutualCoherence}
	The mutual incoherence $\mu_{\text{max}} (\mb{W_{i,k}})$ of the sensing matrix $\mb{W}_{i,k}$ (defined in Eq.~\ref{Eqn:Sensing Matrix}), with $B$ shifts, is upper bounded by
	
	\[ \mu_{\text{max}} < 2\sqrt{\frac{\log(5N)}{B}} \] 
	
\end{lemma}
\begin{proof}
	The proof follows similar lines as the proof for Lemma V.3. in \cite{pawar2014robust}.
\end{proof}
 
\begin{lemma}
For the choice of $B=4c_1^2\log 5N$, the probability of error in identifying the position of a singleton at any bin $(i,j)$ can be upper bounded by
\begin{align*}
\mbb{P}[\mc{E}_{21}]\leq \exp\left\lbrace-\frac{N^{\mu-\alpha}(1-2\eta)^2(c_1^2-1)}{8(c_1^2+1)}\right\rbrace
\end{align*}
\end{lemma}
\begin{proof}
	
	Let $j_p$ be the variable node participating in the singleton $(i,j)$. Then the observation vector $\zv_{i,j}$ is given by
	\begin{align*}
	\underline{z}_{i,j} &= \begin{bmatrix}
	\wv_{j_{1}},\wv_{j_2}, & \cdots   & \wv_{j_p}, &\cdots \ &\wv_{j_{f_i}}
	\end{bmatrix} \times
	\begin{bmatrix}
	n_{1} \\
	\vdots \\
	r[j_p]\\
	\vdots\\
	n_{j} \\
	\vdots\\
	n_{f_i}\\
	\end{bmatrix}\\
	&= r[j_p] ~ \wv_{j_p}+ \sum_{k \neq p}n_k \wv_{j_k} \\
	\end{align*}
	where for convenience we use a simpler notation $j_k=j+(k-1)\frac{N}{f_i}, \wv_{j_k}=\wv^{j_k}$ as defined in Eq. and $n_{l}=\sum\limits_{k=0}^{M-1}x[\theta_{\ell}+k]y[k]$ as defined in Eq. \eqref{Eqn:BinCombination}.
	
	The estimated position $\hat{p}$ is given by
	\begin{align}
	\label{Eqn:SingletonBinCombination}
	\hat{p}= \underset{l}{\argmax}~~ \frac{\wv_{j_l}^{\dagger}\underline{z}_{i,j}}{B}
	\end{align}
	where $\dagger$ denotes the conjugate transpose of the vector. Also note that $|| \wv_{j_k}||=B$ for any $j$ and $k$.  From Eq. \eqref{Eqn:SingletonBinCombination} we observe that the position is wrongly identified when $\exists p'$ such that
	\begin{align*}
	&r[j_p] + \frac{1}{B}\sum_{k \neq p} n_k 	\wv_{j_p}^{\dagger}\wv_{j_k} \leq \frac{r[j_p]}{B} ~ \wv_{j_{p'}}^{\dagger}\wv_{j_p}+ n_{p'}+\frac{1}{B}\sum_{k \neq p,p'}n_k\wv_{j_{p'}}^{\dagger} \wv_{j_k} \\
	&\leftrightarrow \sum_{k \neq p,p'}\alpha_k n_k+\beta n_{p'}\geq  r[j_p]\left(1-\frac{\wv_{j_{p'}}^{\dagger}\wv_{j_p}}{B}\right)\geq M(1-2\eta)(1-\mu_{\text{max}})
	\end{align*}
	where $\alpha_k$ and $\beta$ are constants and can be shown to be in the range $\alpha_k\in[-2\mu_\text{max},2\mu_\text{max}]$ and $\beta\in[1-\mu_\text{max},1+\mu_\text{max}]$. Now using the bound given Chernoff Lemma in Lem.~\ref{Lem:Chernoff2} we obtain
	\begin{align*}
	\mbb{P}[\mc{E}_{21}]&\leq \exp\left\lbrace-\frac{2M(1-2\eta)^2(1-\mu_{\text{max}})^2}{16f_i\mu^2_{\max}+4(1+\mu_{\max})^2}\right\rbrace\\
	&\leq\exp\left\lbrace-\frac{2M(1-2\eta)^2(1-\mu_{\text{max}})^2}{16(f_i\mu^2_{\max}+1)}\right\rbrace\\
	&\leq\exp\left\lbrace-\frac{2M(1-2\eta)^2(c_1-1)^2}{16(f_i+c_1^2)}\right\rbrace\\
	&\approx\exp\left\lbrace-\frac{N^{\mu-\alpha}(1-2\eta)^2(c_1^2-1)}{8(c_1^2+1)}\right\rbrace\\
	\end{align*}
	The second inequality follows by the definition of  $\mu_{\text{max}} \leq 1$.  We choose $B=4c_1^2\log 5N$, and substituting $\mu_{\max}\leq 2\sqrt{\frac{\log 5N}{B}} = 1/c_1$ (Lemma \ref{lemma:MutualCoherence}) we get the third inequality.
	
\end{proof}
\begin{lemma}
	For the choice of $B=4c_1^2\log 5N$,the probability of error in identifying the position of second non-zero variable node at a double-ton at any bin $(i,j)$, given that the first position identification is correct, can be upper bounded by
	\begin{align*}
		\mbb{P}[\mc{E}_{22}]\leq \exp\left\lbrace-\frac{N^{\mu - \alpha} ~ (c_1(1 - 2\eta) - 1)^2}{8(1+ c_1^2)}\right\rbrace
	\end{align*}
\end{lemma}
\begin{proof}
	
	{\bf $\mc{E}_{22}$:}
	
	Let $j_p$ and $j_{\tilde{p}}$ be the two variable nodes participating in the doubleton $(i,j)$. Then the observation vector $\zv_{i,j}$ is given by 
	
	\begin{align*}
		\underline{z}_{i,j} &= \begin{bmatrix}
			\wv_{j_{1}},\wv_{j_2}, & \cdots   & \wv_{j_p}, &\cdots \ &\wv_{j_{f_i}}
		\end{bmatrix} \times
		\begin{bmatrix}
			n_{1} \\
			\vdots \\
			r[j_p]\\
			\vdots\\
			n_{j} \\
			\vdots\\
			r[j_{\tilde{p}}]\\
			\vdots\\
			n_{f_i}\\
		\end{bmatrix}\\
		&= r[j_p] ~ \wv_{j_p} + r[j_{\tilde{p}}] ~ \wv_{j_{\tilde{p}}} + \sum_{k \neq p}n_k \wv_{j_k} \\
	\end{align*}
	
	Let the contribution from $j_{\tilde{p}}$ be peeled off from the doubleton at some iteration, then we get
	\[ \zv_{i,j} = r[j_p] ~ \wv_{j_p} + \frac{e_1}{B} ~ \wv_{j_{\tilde{p}}} + \sum_{k \neq p}n_k \wv_{j_k}\]
	
	where $e_1 \in[-\eta M, \eta M]$ is an extra error term induced due to peeling off.
	
	Now the estimated second position $\hat{p}$ is calculated using Eq. \eqref{Eqn:SingletonBinCombination}. We can observe that the position is wrongly identified when $\exists p'$ such that
	\[ \frac{\wv_{j_p}^{\dagger}\underline{z}_{i,j}}{B} \leq \frac{\wv_{j_{p'}}^{\dagger}\underline{z}_{i,j}}{B}\]
	\begin{align*}
		&\implies r[j_p] + \frac{1}{B}\sum_{k \neq p, \tilde{p}} n_k 	\wv_{j_p}^{\dagger}\wv_{j_k} + \frac{e_1}{B} \wv_{j_p}^{\dagger}\wv_{j_{\tilde{p}}}  \leq \frac{r[j_p]}{B} ~ \wv_{j_{p'}}^{\dagger}\wv_{j_p}+ n_{p'}+\frac{1}{B}\sum_{k \neq p,p',\tilde{p}}n_k\wv_{j_{p'}}^{\dagger} \wv_{j_k} + \frac{e_1}{B} \wv_{j_{p'}}^{\dagger}\wv_{j_{\tilde{p}}}\\
		&\leftrightarrow \sum_{k \neq p,p',\tilde{p}}\alpha_k n_k+ \beta n_{p'}  \geq  r[j_p]\left(1-\frac{\wv_{j_{p'}}^{\dagger}\wv_{j_p}}{B}\right) - \frac{2 \eta M}{B} \wv_{j_{p'}}^{\dagger}\wv_{j_{\tilde{p}}}
	\end{align*}
	\[~~\geq M(1-2\eta)(1-\mu_{\text{max}}) - 2 \eta M \mu_{\text{max}} = M(1 - 2\eta - \mu_{\text{max}})                         
	\]
	where $\alpha_k$ and $\beta$ are constants and can be shown to be in the range $\alpha_k\in[-2\mu_\text{max},2\mu_\text{max}]$ and $\beta\in[1-\mu_\text{max},1+\mu_\text{max}]$. Now using the bound given by Chernoff Lemma in Lem.~\ref{Lem:Chernoff2} we obtain
	\begin{align*}
		\mbb{P}[\mc{E}_{22}]&\leq \exp\left\lbrace-\frac{2M(1 - 2\eta - \mu_{\text{max}})^2}{16f_i\mu^2_{\max}+4(1+\mu_{\max})^2}\right\rbrace\\
		&\leq\exp\left\lbrace-\frac{2M(1 - 2\eta - \mu_{\text{max}})^2}{16(f_i\mu^2_{\max}+1)}\right\rbrace\\
		&\leq\exp\left\lbrace-\frac{M(c_1(1 - 2\eta) - 1)^2}{8(f_i+c_1^2)}\right\rbrace\\
		&\leq\exp\left\lbrace-\frac{N^{\mu - \alpha} ~ (c_1(1 - 2\eta) - 1)^2}{8(1+ c_1^2)}\right\rbrace\\
	\end{align*}
	where for the choice of $B=4c_1^2\log 5N$, $\mu_{\max}\leq 2\sqrt{\frac{\log 5N}{B}} = 1/c_1$.
	
\end{proof}

\begin{lemma}
\end{lemma}

\begin{lemma}
\end{lemma}


\bibliographystyle{ACM-Reference-Format}
\bibliography{../../../bib/journal_abbr,../../../bib/sparseestimation}

\end{document} 
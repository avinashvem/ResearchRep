\section{Notations}

This table below will introduce the notations we use in this paper.
\begin{table}[h!]
\label{Table:Notations}
\caption{Parameters and various quantities involved in describing the algorithm}
\begin{center}
	\begin{tabular}{|c|c|} 	
		\hline		
		\textit{Symbol}				        &  \textit{Meaning} \\		
		\hline
		$N$           				& Size of the string or database in symbols \\
		\hline
		$M = N^{\mu}$       & Length of the query in symbols \\
		\hline
        $L = N^\lambda$    &   Number of matches \\
        \hline
        $K$             &$\max_{\tau}d_{H}(\xv[\tau:\tau+M-1],\yv)$\\
        \hline
	    $\eta$             &$\frac{K}{M}$\\
	    \hline
		$G = N^\gamma$    & Number of blocks \\
		\hline
		$\tilde{N} = N^{1-\gamma}$   & Length of one block \\
		\hline
		$f_i = N^\alpha$     & Length of smaller point IDFT at each branch\\
		\hline
		$n_i = N/f_i$     	   &  Sub-sampling parameter \\
		\hline
		$B$   					    & Number of shifts also referred to as branches  \\
		\hline
		$d$           				& Number of stages in the FFAST algorithm \\
		\hline
	\end{tabular}
\end{center}	
\end{table}	
We denote signals and vectors using the standard vector notation of arrow over the letter, time domain signals using lowercase letters and the frequency domain signals using uppercase letters. For example $\xv = (x[0], x[1], \ldots, x[N] )$ denotes a time domain signal with $i^{\text{th}}$ time component denoted by $x[i]$, and $\Xv= \mathcal{F}_{N}\{\xv\}$ denotes the $N$-point Fourier coefficients of $\xv$. We denote matrices using boldface upper case letters. We denote the set $\{0,1,2\cdots, N-1\}$ by $[N]$. Note that in Table.~\ref{Table:Notations} we suppose the values of $M$ and $L$ to be $N^{\mu}$ and $N^{\lambda}$ respectively but it is not necessary that these parameters have this form. All our results can be extended straight forwardly to any $M$ and $L$ as long as $M=O(N^{\mu})$ and $L=O(N^{\lambda})$.
	
	


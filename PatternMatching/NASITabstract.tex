The Peeling Decoder : Theory and Applications

The peeling decoder is a simple greedy decoder that can be used to decode classes of codes defined on graphs, such as low density parity check (LDPC) codes and low density generator matrix codes, on the erasure channel.  This deceptively simple decoder suffices to design capacity achieving coding schemes for the erasure channel.  In addition, the peeling decoder can also be used to design optimal universal rateless codes as shown by Luby in the design of LT codes. In part I of this two-part tutorial, we will explain the main theoretical ideas behind the analysis of the peeling decoder and the design of optimal fixed rate and rateless codes for the erasure channel. We will also discuss how the peeling decoder can be used to decode generalized LDPC codes and product codes when used with non-erasure channels. We will conclude part I with a discussion of the relationship between channel coding and syndrome source coding for the compression of sparse sources.

The peeling decoder has been applied successfully not only to decoding codes on the erasure channel, but also in a variety of applications outside of main stream coding theory.  In Part II of this tutorial, we will discuss applications of the peeling decoder to massive uncoordinated multiple access schemes, sparse Fourier transform computation and if time permits, to compressed sensing and group testing problems. Remarkably, the peeling decoder can be used to design optimal multiple access schemes in some cases and order-optimal algorithms for sparse transform computation and some sparse support recovery problems. 
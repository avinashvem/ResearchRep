\section{Sample and Computational Complexity}
In this section, we will analyze the sample complexity which is the  number of samples we access from the sketch of the signal $\xv$ stored in the database and the computational complexity as a function of the system parameters.

\subsection{\bf Sample Complexity} As stated in the problem statement, we exploit the sparseness in IDFT computations to compute the correlations using less number of samples from the database. In each branch of the FFAST algorithm we down-sample the $N$ samples by a factor of $N^{1-\alpha}$ to get $N^{\alpha}$ samples. We also do this for each chosen shift and we get a total of  $O(N^{\alpha}\log N)$ samples per block per stage. We repeat this for $d = \frac{1}{1-\alpha}$ such stages to get a total complexity of $dN^{\alpha}\log N$ samples per block. So, the total number of samples is given by  

\begin{align*}
\text{Total \# of samples required (S)} &=  d \ \underset{\text{Samples per block per stage} }{\underbrace{N^{\alpha}\log N}}\\
   &=   O(N^{\max(\mu,1-\mu)}\log N)
\end{align*}



\subsection{\bf Computational Complexity}

As described in Eq.~\ref{eqn:Rxy_fourier}, the computation of $\rv$ involves three steps:

\begin{enumerate} 
	\item  Operation - \RNum{1}:
	 Computing $\mathcal{F}_{N}^{-1}\{ \xv \}$ involves two parts: RSIDFT framework and the Decoder. The RSIDFT framework involves computing smaller $f_i$ point IDFTs, which takes approximately $O(N^{\alpha} \log N^{\alpha})$ computations in each branch. For a total of $d~B$ branches, we get a complexity of $O(d ~ B ~ N^{\alpha} \log N^{\alpha})$. In the decoding process, the dominant computation is from position identification. Each position identification process involves correlating the observation vector of length $B$ with approximately $N^{1-\alpha}$ column vectors, which amounts to $B N^{1-\alpha}$ computations. There will a maximum of $d L$ such position identifications, which gives a complexity of $O(d ~B ~L ~N^{1-\alpha} )$.   So, the total number of computations involved in this step, $C_{\RNum{1}}$, is given by
	\begin{align*}
	C_{\RNum{1}} \ &=  {d ~ B} \ \left (
	\underset{\text{Shorter IFFTs /block/stage} }{\underbrace{ O(N^{\alpha} \ \log N^{\alpha})}}  ~ + ~ L~N^{1-\alpha} \right )& \\
	\vspace{2pt}
	\ & =	
	\begin{cases}
	O(N^{\max(1-\mu, \mu + \lambda) } \log^2 N ), &  	 \mu < 0.5\\
	O(N^{\max(\mu, 1- \mu + \lambda )} \log^2 N ), &  	 \mu > 0.5
	\end{cases}
	\end{align*}
	
	
	\item  Operation - \RNum{2}:
	Since we assume that the sketch of database $\xv$, $\mathcal{F}_{N}\{\xv\}$, is pre-computed, we do not include this in computational complexity.
	
	\item  Operation - \RNum{3}:
	
	As described in Sec.~\ref{subsec:skteches}, in each branch $(i,j)$, we use a folding based technique to compute the sketch of $\yv$, $\mathcal{F}_{N}\{\yv\}$ at points in the set $\mc{S}_{i,j}$. The folding technique involves two steps: folding and adding (aliasing) which has a complexity of $O(M)$ computations , and computing $N^{\alpha}$ point DFTs that takes $O(N^\alpha \log N^{\alpha})$ computations. So, for a total of $d~B$ branches the number of computations in this step is given by
	
	\begin{align*}
	 C_{\RNum{3}} \ &= \  d ~ B ~ 
	( \underset{\text{\# of additions (aliasing)} }{\underbrace{N^{\mu}}} + \ \
	\underset{\text{Shorter FFTs} }{\underbrace{N^{\alpha} \ \log N^{\alpha}}} \ ) \\
	&=  O(N^{max(\mu,1- \mu)} \log^2 N))
	\end{align*}
	
	
	{\textit{Note:}} Folding and adding, for each shift, involves adding $N^{1-\alpha}$ vectors of length $N^{\alpha}$. We know that the length of the query is $M =N^{\delta}$, i.e., the number of non-zero elements in $Y$ (zero-padded version of the query) is $M$ and hence we only need to compute $M$ additions instead of the length of the vector.
	
\end{enumerate}

Total number of computations, $C = C_{\RNum{1}}+C_{\RNum{3}} $, is given by

   
  \begin{align*}
  C \ =
  \begin{cases}
  O(N^{\max(1-\mu, \mu + \lambda) } \log^2 N ), &  	 \mu < 0.5\\
  O(N^{\max(\mu, 1- \mu + \lambda )} \log^2 N ), &  	 \mu > 0.5
  \end{cases}
  \end{align*}
  



%\subsection{Sample Complexity}
%As stated in the problem statement, we exploit the sparseness in IDFT computations to compute the correlations using less number of samples from the database.
%
%
%
%
%In each branch of the FFAST algorithm, executed over a block of samples, we down-sample the $\tilde{N}$ samples by a factor of $N^{\alpha}$ to get $N^{1-\gamma-\alpha}$ samples. We also do this for each chosen shift and we get a total of  $N^{1-\gamma-\alpha+\beta}$ samples per block per stage. We repeat this for $d = \frac{1-\gamma}{\alpha}$ such stages to get a total complexity of $dN^{1-\gamma-\alpha+\beta}$ samples per block. So, the total number of samples is given by  
%
%\[
%\text{Total \# of samples required (S)} \  = \  \underset{\text{$d$ stages} }{\underbrace{\frac{1-\gamma}{\alpha}}} \ \underset{\text{Samples per block per stage} }{\underbrace{N^{1-\gamma-\alpha+\beta}}} \ \underset{\text{ \# of blocks} }{\underbrace{N^{\gamma}}} \ \ 
%   =  \ \ \frac{1-\gamma}{\alpha} \ N^{1-\alpha+\beta}
%\]  
%
%
%
%\subsection{Computational Complexity}
%
%Consider the operations involved in the computation of correlation of $X$ and $Y$, $R_{XY}$, as shown below.
%
%\[  R_{XY} = \underset{\text{ \RNum{1} } } {\underbrace{IDFT}} \ \{ \underset{\text{ \RNum{2} } }{\underbrace{DFT\{ X \} } }  \times \ \underset{\text{ \RNum{3} } }{\underbrace{DFT\{ Y \} } }  \}  \]
%
%
%In each block of samples from the database, the process of computing the correlation involves three sets of operations: two $\tilde{N}$-point DFT operations ($\RNum{2}, \RNum{3}$) sampled at $N^{1-\gamma-\alpha+\beta}$ points(based on sampling requirements of FFAST architecture) and one sparse-IDFFT operation ($\RNum{1}$). We assume that the database already contains sampled version of DFT \{ $X$ \} stored and hence we exclude those computations from the complexity calculations. So, it is enough that we analyze the computations involved in operations ($\RNum{1}, \RNum{3}$) on each block to get the overall computational complexity.
%
%\begin{enumerate}
%	\item  Operation - \RNum{3}: 
%	
%	Notice that the operation $ DFT \{ Y \} $ is just a one time computation and can be reused in each block since the query is same for all blocks.
%	Since we only need to compute the sampled version of the DFT\{$Y$\} in each branch, we can do some clever implementations to reduce the complexity. We can create aliasing on $Y$ and then take a shorter length ($N^{1-\gamma-\alpha}$-point) DFT to obtain the sampled version of $\tilde{N}$-point DFT of $Y$.
%	
%	This method of computing sampled DFT\{$Y$\} involves two operations - folding and adding (aliasing), and then a $N^{1-\gamma-\alpha}$-point DFT. The complexity involved in computing down-sampled DFT\{$Y$\}, $C_{\RNum{3}}$, is given by
%	
%	 \[ C_{\RNum{3}} \ = \  \underset{\text{$d$ stages} }{\underbrace{\frac{1-\gamma}{\alpha}}} \
%	  ( \underset{\text{\# of additions (aliasing)} }{\underbrace{N^{\delta+\beta}}} + \ \
%	  \underset{\text{Shorter FFTs} }{\underbrace{N^{1-\gamma-\alpha+\beta} \ \log N^{1-\gamma-\alpha}}} \ )  
%	 \]
%	
%	Folding and adding, for each shift, involves adding $N^{\alpha}$ vectors of length $N^{1-\gamma-\alpha}$. We know that the length of the query is $L =N^{\delta}$, i.e., the number of non-zero elements in $Y$ (zero-padded version of the query) is $L$ and hence we only need to compute $L$ additions instead of the length of the vector.
%	
%	\item  Operation - \RNum{1}:
%	
%	Taking advantage of the fact that the correlations are sparse (with some noise), we use a FFAST based architecture to reduce the complexity involved in computing IDFT. The number of computations involved in this process,$C_{\RNum{1}}$, is given by
%	
%	 \[C_{\RNum{1}} \ = \ \underset{\text{ \# of blocks} }{\underbrace{N^{\gamma}}} \underset{\text{$d$ stages} }{\underbrace{\frac{1-\gamma}{\alpha}}} \
%	 \underset{\text{Shorter IFFTs /block/stage} }{\underbrace{N^{1-\gamma-\alpha+\beta} \ \log N^{1-\gamma-\alpha}}} 
%	 \ = \  \frac{1-\gamma}{\alpha} \ N^{1-\alpha+\beta} \ \log N^{1-\gamma-\alpha} 
%	 \]    
%\end{enumerate}
%
%Total number of computations, $C = C_{\RNum{3}}+C_{\RNum{1}} $, is given by
%
%  \[ C \ = \ \  \frac{1-\gamma}{\alpha} \ ( N^{\delta+\beta} \ + \ N^{1-\gamma-\alpha+\beta} \ \log N^{1-\gamma-\alpha} \ + \ N^{1-\alpha+\beta} \ \log N^{1-\gamma-\alpha})
%  \]
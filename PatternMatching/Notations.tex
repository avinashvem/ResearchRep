\section{Notations}

This table below will introduce the notations we use in this paper.

\begin{center}
	
	\begin{tabular}{|c|c|}
		
		\hline
		
		Symbol     &  Notational Meaning \\
		
		\hline
		
		$N$           & Size of the string or database in symbols \\
		\hline
		$M = N^{\mu}$           & Length of the query in symbols \\
		\hline
        $L = N^\lambda$  &   Number of matches \\
        \hline
        $K$             & Maximum Hamming distance allowed between $\yv$ and $\xv$ \\
        \hline
		$G = N^\gamma$           & Number of blocks \\
		\hline
		$\tilde{N} = N^{1-\gamma}$   & Length of one block \\
		\hline
		$A = N^\alpha$      & Sub-sampling parameter \\
		\hline
		$B$       & Number of shifts  \\
		\hline
		$d$           & Number of stages in the FFAST algorithm \\
		\hline
	\end{tabular}
\end{center}	
	
	\vspace{15pt}


We denote vectors using the underbar, time domain signals using lowercase letters and the frequency domain signals using uppercase letter. For example $\xv = \{x[1],x[2], \cdots x[N] \}$ denotes a time domain signal with $i^{th}$ time component denoted by $x[i]$, and $\Xv= \mathcal{F}\xv$ denotes the Fourier coefficients of $\xv$. Matrices are denoted using uppercases letters. We differentiate a signal from a matrix by having a underbar for a signal vector. We denote the set ${0,1,2\cdots, N-1}$ by $[N]$.
	
	

